\documentclass[10pt,a4paper]{book}
\usepackage[utf8]{inputenc}
\usepackage[french]{babel}
\usepackage[T1]{fontenc}
\usepackage{amsmath}
\usepackage{amsfonts}
\usepackage{amssymb}
\author{Hexanome: 4105\\\\Alexis Andra\\Jolan Cornevin\\Mohamed Haidara\\Alexis Papin\\Robin Royer\\Maximilian Schiedermeier\\David Wobrock}
\title{Compte-rendu : Devéloppement Orienté Objet}
\usepackage{graphicx}
\usepackage{xcolor}
\usepackage[hidelinks]{hyperref}
\usepackage{natbib}
\usepackage{listings}
\usepackage{color}
\usepackage{longtable}
\usepackage{url}
\definecolor{lightgray}{rgb}{.95,.95,.95}
\definecolor{darkgray}{rgb}{.4,.4,.4}
\definecolor{purple}{rgb}{0.65, 0.12, 0.82}
\usepackage{geometry}
\geometry{
 total={ 210mm,297mm},
 left=20mm,
 right=20mm,
 top=15mm,
 bottom=15mm
 }

\lstdefinelanguage{JavaScript}{
  keywords={typeof, new, true, false, catch, function, return, null, catch, switch, var, if, in, while, do, else, case, break, done, fi, elif, for},
  keywordstyle=\color{blue}\bfseries,
  ndkeywords={class, export, boolean, throw, implements, import, this},
  ndkeywordstyle=\color{darkgray}\bfseries,
  identifierstyle=\color{black},
  sensitive=false,
  comment=[l]{//},
  morecomment=[s]{/*}{*/},
  commentstyle=\color{purple}\ttfamily,
  stringstyle=\color{black}\ttfamily,
  morestring=[b]',
  morestring=[b]"
}
\lstset{
   language=JavaScript,
   frame=single,
   extendedchars=true,
   basicstyle=\footnotesize\ttfamily,
   showstringspaces=false,
   showspaces=false,
   numbers=left,
   numberstyle=\footnotesize,
   numbersep=9pt,
   tabsize=2,
   breaklines=true,
   showtabs=false,
   captionpos=b,
   backgroundcolor=\color{lightgray},
}

\makeindex
\begin{document}
\maketitle
\tableofcontents 
\chapter{Capture et analyse des besoins}
\section{Planning prévisionnel du projet}
Le planning prévisionnel du projet s'est déroulé en trois étapes. Tout d'abord, nous avons identifié les stages nécessaires pour réaliser le projet. Nous avons transformé les missions en tâches réalisables qui sont exprimées par la liste suivante :
\begin{itemize}
	\item{Planning previsionnel du projet (3.0 h)}
	\item{Analyse du modèle (3.0h)}
	\item{Conception du modèle (2.5 h)}
	\item{Analyse des \textit{CU}s, conception des \textit{diagrammes de séquences} (3.5 h)}
	\item{Description textuelle des \textit{CU}s (4.0 h)}
	\item{Implémentation des classes représentant les données dans les fichiers \textit{XMLs} (1.5 h)\footnote{La planification d'une heure et demi peut apparaître insuffisante pour implémenter un modèle. Au moment de la conception de l'architecture, le modèle était considéré comme un ensemble de \textit{JavaBeans}. La réalisation des algorithmes était prévue dans le package contrôleur.}}
	\item{Conception d'IHM (prototype) (8.0 h)}
	\item{Conception d'IHM (précise) (8.0 h)}
	\item{Spécification d'\textit{événements utilisateur} (4.0 h)}
	\item{Conception du diagramme \textit{états-transitions} (6.0h)}
	\item{Implémentation du sérialiseur / désérialiseur (8.0 h)}
	\item{Spécification des interfaces du package contrôleur (5.0 h)}
	\item{Spécification des interfaces du package vue (3.0 h)}
	\item{Spécification des interfaces du package modèle (5.0 h)}
	\item{Conception de diagramme de classes du package contrôleur (6.0 h)}
	\item{Conception de diagramme de classes du package vue (3.0 h)}
	\item{Conception de diagramme de classes du package modèle (6.0 h)}
	\item{Implémentation des observateurs qui notifient la vue (4.0 h)}
	\item{Réalisation de l'IHM (package vue) (20 h)}
	\item{Mock implémentation des classes du package contrôleur (3.5 h)}
	\item{Conception et réalisation des tests unitaires (20 h)}	
	\item{Conception et réalisation des tests fonctionnels (20 h)}
	\item{Implémentation du package contrôleur (30 h)}
	\item{Implémentation du package vue (20 h)}
	\item{Implémentation du package modèle (10 h)}
	\item{Rétro-génération des diagrammes de classes à partir du code (3.0 h)}
	\item{Redaction du glossaire (2.0 h)}
\end{itemize}
~\\Deuxièmement, nous avons essayé de trouver toutes les dépendances entre les missions prévues. Le but de cette démarche était de pouvoir paralléliser au plus les développements. Le graphique suivant montre le déroulement prévisionnel du projet et également qui était le responsable prévu de chaque étape.

\begin{figure}[ht!]
    \centering
    \scalebox{.65}{\input{der.pdf_tex}}
    \caption{Dépendances et déroulement prévisionnel du projet}
\end{figure}

~\\Dans la troisième et dernière étape, nous avons estimé le temps nécessaire pour réaliser ces tâches en utilisant redmine.
\begin{figure}
    \centering
    \includegraphics[scale=0.25]{redmine.png}
    \caption{Capture d'écran de la gestion du projet avec redmine} 
\end{figure}

\section{Modèle du domaine}
\begin{figure}[h!]
    \centering
    \includegraphics[scale=0.35]{dia-mod.jpg}
    \caption{Diagramme du modèle}
\end{figure}
\section{Glossaire}
\subsection{Entités}
\begin{itemize}
\item \textbf{Plan de ville XML :} Fichier XML décrivant le \textit{plan de ville}. \vskip1mm
\item \textbf{Plan de ville :} Abstraction de la carte d'une ville. \vskip1mm
\item \textbf{Intersection :} Coordonnées d'intersection entre des \textit{tronçons}. \vskip1mm
\item \textbf{Coordonnée x d'intersection :} Coordonnée x d'\textit{intersection}. \vskip1mm
\item \textbf{Coordonnée x d'intersection :} Coordonnée y d'\textit{intersection}. \vskip1mm
\item \textbf{Tronçon :} Tronçon de rue entre deux \textit{intersections}. \vskip1mm
\item \textbf{Longueur du tronçon :} Longueur d'un \textit{tronçon}. \vskip1mm
\item \textbf{Vitesse moyenne de circulation du tronçon :} Vitesse moyenne de circulation sur un \textit{tronçon}. \vskip1mm
\item \textbf{Demandes de livraisons XML :} Fichier XML décrivant des \textit{demandes de livraisons}. \vskip1mm
\item \textbf{Demandes de livraisons :} Abstraction des \textit{livraisons} à effectuer. \vskip1mm
\item \textbf{Tournée :} Tournée de \textit{livraisons} se composant d'un \textit{itinéraire}, d'une \textit{durée de la tournée} et de tout le détail pour chaque livraison. \vskip1mm
\item \textbf{Itinéaire :} Suite ordonnée de \textit{livraisons}. \vskip1mm
\item \textbf{Durée de la tournée :} Durée pour effectuer l'ensemble des \textit{livraisons}. \vskip1mm
\item \textbf{Entrepot :} \textit{Intersection} de départ et de fin de la \textit{tournée}. \vskip1mm
\item \textbf{Feuille de route :} Fichier texte spécifiant la \textit{tournée}. \vskip1mm
\item \textbf{Livraison :} Livraison à effectuer. \vskip1mm
\item \textbf{Adresse de livraison :} \textit{Interection} où effectuer la \textit{livraison}. \vskip1mm
\item \textbf{Fenêtre de livraison :} Plage horaire pour effectuer une \textit{livraison}. \vskip1mm
\item \textbf{Début de fenêtre de livraison :} Heure de début de \textit{fenêtre de livraison}. \vskip1mm
\item \textbf{Fin de fenêtre de livraison :} Heure de fin de \textit{fenêtre de livraison}. \vskip1mm
\item \textbf{Temps de livraison :} \textit{Temps de trajet de livraison} + \textit{temps d'arrêt de livraison}. \vskip1mm
\item \textbf{Chemin de livraison :} Liste des \textit{tronçons} entre deux \textit{livraisons}. \vskip1mm
\item \textbf{Temps de trajet de livraison :} Temps pour parcourir le \textit{chemin de livraison}. \vskip1mm
\item \textbf{Temps d'arrêt de livraison :} Temps pour décharger la \textit{livraison} (10 min ou plus si le livreur est en avance sur la \textit{fenêtre de livraison}). \vskip1mm
\item \textbf{Horaire de passage de livraison :} Heure au début du \textit{temps d'arrêt de la livraison}. \vskip1mm
\item \textbf{Horaire de départ de livraison :} Heure à la fin du \textit{temps d'arrêt de livraison}. (=\textit{horaire de passage de livraison} + \textit{temps d'arrêt de livraison}) \vskip1mm
\item \textbf{Livraison en retard :} \textit{Livraison} effectuée en dehors de la \textit{fenêtre de livraison}. \vskip1mm
\item \textbf{Livraison en retard :} \textit{Livraison} effectuée en dehors de la \textit{fenêtre de livraison}. \vskip1mm
\end{itemize}
\subsection{Actions}
\begin{itemize}
\item \textbf{Annuler :} Annule la dernière action effectuée. \vskip1mm
\item \textbf{Rétablir :} Rétablit la dernière action annulée. \vskip1mm
\item \textbf{Ajouter livraisons :} Permet d'entrer dans le mode d'ajout de \textit{livraisons} à la \textit{tournée}. \vskip1mm
\item \textbf{Echanger livraisons :} Permet d'entrer dans le mode d'échange de \textit{livraisons} dans la \textit{tournée}. \vskip1mm
\item \textbf{Supprimer livraisons :} Permet d'entrer dans le mode de suppression de \textit{livraisons} dans le \textit{tournée}. \vskip1mm
\item \textbf{Calculer tournée :} Génere la \textit{tournée} à partir des \textit{demandes de livraisons}. \vskip1mm
\item \textbf{Générer la feuille de route :} Génere la \textit{feuille de route} à partir de la \textit{tournée}. \vskip1mm
\item \textbf{Ouvrir plan de la ville :} Permet de sélectionner et charger le \textit{plan de ville XML} et ainsi de visualiser le \textit{plan de ville}. \vskip1mm
\item \textbf{Ouvrir demandes de livraisons :} Permet de sélectionner et charger les \textit{demandes de livraisons XML} et ainsi de visualiser les \textit{demandes de livraisons}. \vskip1mm
\item \textbf{Quitter :} Permet de quitter l'application. \vskip1mm
\end{itemize}
\section{Diagramme de cas d'utilisation}
\begin{figure}[h!]
    \centering
    \includegraphics[scale=0.28]{cu.png}
    \caption{Diagramme de cas d'utilisation}
\end{figure}
\section{Description textuelle structurée des cas d'utilisation}
\subsection{Charger le plan}
\begin{itemize}
	\item{Pré-conditions}
	\item{Scénario principale}
	\begin{itemize}
		\item{Le système demande de charger le plan de la ville XML}
		\item{L’utilisateur sélectionne un plan dans son système de fichiers}
		\item{Le plan de la ville est affiché}
	\end{itemize}
	\item{Scénario alternatif}
	\begin{itemize}
		\item{Le plan de la ville XML n’est pas au bon format ou bien formé. $\rightarrow$ Le système indique que le plan de la ville XML n’est pas valide et retourne à l’étape 1.}
	\end{itemize}
	\item{Liste des appels utilisateurs}
	\begin{itemize}
		\item{Choisir le plan de ville XML}
	\end{itemize}
\end{itemize}
\subsection{Charger les livraisons}
\begin{itemize}
	\item{Pré-conditions}
	\begin{itemize}
		\item{Le plan de la ville xml doit être charge et le plan de la ville est affiche}
	\end{itemize}
	\item{Scénario principale}
	\begin{itemize}
		\item{Le système demande la saisie d’un plan de livraison xml}
		\item{L’utilisateur sélectionne, dans son système de fichier, un plan de livraison xml}
		\item{Le système affiche les adresses de livraisons sur le plan de ville.}
	\end{itemize}
	\item{Scénario alternatif}
	\begin{itemize}
		\item{Le système indique que le plan de livraisons xml est incorrecte ou mal-forme $\rightarrow$ retour a l’etape 1}
	\end{itemize}
	\item{Liste des appels utilisateurs}
	\begin{itemize}
		\item{Choisir le plan de livraisons XML}
	\end{itemize}
\end{itemize}
\subsection{Calculer la tournée}
\begin{itemize}
	\item{Pré-conditions}
	\begin{itemize}
		\item{Le plan de livraisons XML est charge et le plan de livraison est affiche sur le plan de la ville}
	\end{itemize}
	\item{Scénario principale}
	\begin{itemize}
		\item{L’utilisateur demande le calcul de la tournee.}
		\item{Le systeme calcul la tournee optimale.}
		\item{Le systeme affiche la tournee sur le plan et affiche la liste des livraisons(heure de passage, fenetre de livraisons et retard en rouge).}
	\end{itemize}
	\item{Scénario alternatif}
	\item{Liste des appels utilisateurs}
	\begin{itemize}
		\item{Demande de calcul de tournées}
	\end{itemize}
\end{itemize}
\subsection{Modifier la tournée (Supprimer une livraison)}
\begin{itemize}
	\item{Pré-conditions}
	\begin{itemize}
		\item{La tournée doit être calculée}
	\end{itemize}
	\item{Scénario principale}
	\begin{itemize}
		\item{L'utilisateur entre en mode suppression}
		\item{Le système demande à l'utilisateur quelle livraison il souhaite supprimer}
		\item{L'utilisateur indique au système quelle livraison il veut supprimer}
		\item{Le système modifie la tournée afin de prendre en compte la suppression en compte}
		\item{Le système affiche les modifications à l'utilisateur}
	\end{itemize}
	\item{Scénario alternatif}
	\item{Liste des appels utilisateurs}
	\begin{itemize}
		\item{Supprimer livraison}
	\end{itemize}
\end{itemize}
\subsection{Modifier la tournée (Échanger deux livraisons)}
\begin{itemize}
	\item{Pré-conditions}
		\begin{itemize}
		\item{La tournée doit être calculée}
	\end{itemize}
	\item{Scénario principale}
	\begin{itemize}
		\item{L’utilisateur entre en mode échange}
		\item{Le système demande à l’utilisateur de choisir les deux livraisons à échanger}
		\item{L’utilisateur sélectionne deux livraisons à échanger}
		\item{Le système modifie la tournée afin de prendre l’échange en compte}
		\item{Le système affiche les modifications à l’utilisateur}
	\end{itemize}
	\item{Scénario alternatif}
	\item{Liste des appels utilisateurs}
	\begin{itemize}
		\item{Échanger livraisons}
	\end{itemize}
\end{itemize}
\subsection{Modifier la tournée (Ajouter une livraison)}
\begin{itemize}
	\item{Pré-conditions}
		\begin{itemize}
		\item{La tournée doit être calculée}
	\end{itemize}
	\item{Scénario principale}
	\begin{itemize}
		\item{L’utilisateur entre en mode ajout}
		\item{Le système demande à l’utilisateur l’intersection pour le lieu de livraison}
		\item{L’utilisateur choisit  où il veut que la livraison soit effectuée}
		\item{Le système demande à l’utilisateur à quel moment la livraison doit être effectuée}
		\item{L’utilisateur indique au systeme la livraison précédente à celle qu’il désire ajouter}
		\item{Le système modifie la tournée afin de prendre l’ajout en compte}
		\item{Le système affiche les modifications à l’utilisateur}
	\end{itemize}
	\item{Scénario alternatif}
	\item{Liste des appels utilisateurs}
	\begin{itemize}
		\item{Ajouter livraison}
	\end{itemize}
\end{itemize}
\subsection{Modifier la tournée (Annuler)}
\begin{itemize}
	\item{Pré-conditions}
		\begin{itemize}
		\item{Il y a eu une modification}
	\end{itemize}
	\item{Scénario principale}
	\begin{itemize}
		\item{L’utilisateur indique au système qu’il souhaite annuler}
		\item{Le système effectue l’opération inverse à la modification}
		\item{Le système modifie la tournée afin de prendre l’annulation en compte}
		\item{Le système affiche la modification à l’utilisateur}
	\end{itemize}
	\item{Scénario alternatif}
	\item{Liste des appels utilisateurs}
	\begin{itemize}
		\item{Annuler}
	\end{itemize}
\end{itemize}
\subsection{Modifier la tournée (Rétablir)}
\begin{itemize}
	\item{Pré-conditions}
		\begin{itemize}
		\item{Avoir annulé quelque chose}
	\end{itemize}
	\item{Scénario principale}
	\begin{itemize}
		\item{L’utilisateur indique au système qu’il souhaite rétablir}
		\item{Le système effectue l’opération qui a été annulée}
		\item{Le système modifie la tournée afin de prendre en compte le rétablissement}
		\item{Le système affiche les modifications à l’utilisateur}
	\end{itemize}
	\item{Scénario alternatif}
	\item{Liste des appels utilisateurs}
	\begin{itemize}
		\item{Rétablir}
	\end{itemize}
\end{itemize}
\subsection{Générer la feuille de route}
\begin{itemize}
	\item{Pré-conditions}
	\begin{itemize}
		\item{Avoir calulé la tournée}
	\end{itemize}
	\item{Scénario principale}
	\begin{itemize}
		\item{L’utilisateur demande la génération d’une feuille de route}
		\item{Le système ouvre une boîte de dialogue pour sélectionner le répertoire d’enregistrement}
		\item{L’utilisateur choisit le répertoire et le nom du fichier à enregistrer}
		\item{Le système génère le texte de la feuille de route}
		\item{Le système écrit le texte dans le fichier choisit préalablement}
		\item{Le système retourne à l’affichage du plan}
	\end{itemize}
	\item{Scénario alternatif}
	\begin{itemize}
		\item{L’écriture disque échoue $\rightarrow$
Le système indique l’erreur à l'utilisateur et retourne à affichage du plan}
	\end{itemize}
	\item{Liste des appels utilisateurs}
	\begin{itemize}
		\item{Générer la feuille de route}
	\end{itemize}
\end{itemize}
\chapter{Conception}
\section{Liste des événements utilisateur et diagramme états-transitions}
\subsection{Liste des événements utilisateur}
\begin{itemize}
	\item{Clic bouton}
	\begin{itemize}
		\item{Charger le fichier du plan de la ville}
		\item{Charger le fichier des demandes de livraisons}
		\item{Ajouter livraisons}
		\item{Echanger livraisons}
		\item{Supprimer livraison}
		\item{Generer feuille de route}
	\end{itemize}
	\item{Clic gauche}
	\begin{itemize}
		\item{Livraison (liste+plan)}
		\item{Intersection (plan)}	
	\end{itemize}
	\item{Fichier}
	\begin{itemize}
		\item{Ouvrir > plan de la ville}
		\item{Ouvrir demandes de livraisons}
	\end{itemize}
	\item{Edition}	
	\begin{itemize}
		\item{Annuler}
		\item{Rétablir}
	\end{itemize}
	\item{Ctrl}
	\begin{itemize}
		\item{Z : Annuler}
		\item{Y : Rétablir}
		\item{P : Ouvrir plan de la ville}
		\item{O : Ouvrir demandes de livraisons}
	\end{itemize}
	\item{Clic droit}
	\begin{itemize}
		\item{Vue graphique}
	\end{itemize}	
\end{itemize}
\newpage
\subsection{Diagramme états-transitions}
\begin{figure}[h!]
    \centering
    \includegraphics[scale=0.35]{diag-et-trans.png}
    \caption{Diagramme des états et transitions}
\end{figure}
\newpage
\section{Diagrammes de packages et de classes}
\paragraph{Précisions sur les diagrammes}
Il s'agit des diagrammes de conception. Ils ne correspondent pas forcement à l'implémentation finalement réalisé.
\subsection{Packages}
\begin{figure}[h!]
    \centering
    \includegraphics[scale=0.2]{packages.png}
    \caption{Packages}
\end{figure}
\newpage
\subsection{Controleur}
\begin{figure}[h!]
	%\advance\leftskip-3cm
    \includegraphics[scale=0.30]{controleur.png}
    \caption{Diagramme des classes du \textit{controleur}}
\end{figure}
\newpage
\subsection{Modele}
\begin{figure}[h!]
    \centering
    \includegraphics[scale=0.28]{modele.png}
    \caption{Diagramme des classes du \textit{modele}}
\end{figure}
\newpage
\subsection{Vue}
\begin{figure}[h!]
    \centering
    \includegraphics[scale=0.4]{vue.png}
    \caption{Diagramme des classes du \textit{vue}}
\end{figure}
\section{Choix architecturaux et design patterns utilisés}
Étant donné la complexité du projet, il a fallu penser en amont à une architecture capable de faciliter le développement, d'assurer une maintenabilité du code et ainsi produire un programme de qualité. 

Dans cette optique, il paraissait évident d'utiliser certains design patterns ou patrons de conception notamment ceux du \textit{Gang of four}. Dans cette partie vous trouverez ainsi une description des patrons et les choix architecturaux associés.
\begin{itemize}

\item \textbf{Modèle-Vue-Contrôleur ou MVC}

L'architecture globale du projet est basée sur le patron de conception MVC. En effet ce patron permet d'avoir une bonne séparation entre la logique applicative et l'interface graphique de l'application. Elle offre ainsi un cadre normalisé pour structurer une application d'une certaine complexité comme celle qu'on a eu à développer. Avec cette structure on pourra facilement changer d'interface graphique en gardant le \textit{Modèle} qui constitue le cœur (algorithmique) de l'application.

Notre contrôleur se contente de gérer les différents événements envoyés par la vue. Il n'effectue aucun traitement. Le traitement est géré par les différents états de l'application à travers le pattern État expliqué un peu plus bas. Le contrôleur en tant que tel est divisé deux classes : une partie reçoit les événements puis renvoie la gestion à l'état, l'autre partie appelée \textit{ControleurDonnees} se charge d’accéder au modèle et contient par exemple l'historique des commandes exécutées. C'est cette dernière qui est utilisée par les états pour accéder au modèle.

La vue fait appel au contrôleur grâce à une \textbf{façade} \footnote{\url{https://fr.wikipedia.org/wiki/Fa\%C3\%A7ade\_(patron\_de\_conception)}} (une interface plus précisément) qui n'expose que les méthodes nécessaires à la vue. Ainsi on peut cacher certains détails du contrôleur à la vue ou facilement changer de contrôleur sans impacter la vue notamment pour un mode test ou un mode \textit{mock} \footnote{\url{https://fr.wikipedia.org/wiki/Mock\_(programmation\_orient\%C3\%A9e\_objet)}}.
Le lancement de l'application passe par la vue qui se charge créer le contrôleur par l’intermédiaire de la classe \textit{FenetrePrincipale}.

\item{\textbf{État}} 
	
Pour définir les différents comportements de notre application, on a utilisé le design pattern Etat qui permet déléguer le traitement d'un évènement à une autre classe. Dans une logique de programmer pour des interfaces, on a défini une interface commune à tous les états possibles de notre application. Chaque état implémentant cette interface, rédefini les évènements qui ont un effet sur lui. L'ajout d'un nouvel état c'est à dire d'une nouvelle fonctionnalité devient ainsi assez facile. Par contre, ajouter un nouvel évènement utilisateur serait un peu fastidieux dans la mesure où il faudrait que tous les états redéfinissent la méthode associée à ce nouvel évènement meme s'ils n'effectueront aucun traitement.

Dans notre cas, au vu des différentes fonctionnalités de notre application, il paraissait évident d'utiliser un tel patron de conception.	


\item \textbf{Observateur/Observable}
	
Ce patron établit une relation un à plusieurs entre des objets, où lorsqu'un objet change, plusieurs autres objets sont avisés du changement. Dans ce patron, un objet le sujet (l'observable) tient une liste des objets dépendants (les observateurs) qui seront avertis des modifications apportées au sujet. Quand une modification est apportée, le sujet émet un message aux différents observateurs. Le message peut contenir une description détaillée du changement. 

Dans ce patron, un objet observable comporte une méthode pour inscrire des observateurs. Chaque observateur comporte une méthode \textit{notify}. Lorsqu'un message est émis,  la méthode \textit{notify} de chaque observateur est appelée.

Pour assurer une communication entre notre modèle et la vue, on a utilisé ce patron de conception.

\item \textbf{Commande}

Ce patron encapsule une demande dans un objet, permettant de paramétrer, mettre en file d'attente, journaliser et annuler des demandes ou actions utilisateurs. Dans ce patron un objet commande correspond à une opération à effectuer et qui aura un effet sur le modèle. L'interface de cet objet comporte une méthode \textit{exécuter} et \textit{annuler}. Pour chaque opération, l'application va créer un objet différent qui implémente cette interface. L'opération est lancée lorsque la méthode exécuter est utilisée. Chaque commande est responsable d'annuler les modifications qu'elle à apportei adu mcodèle et de notifier la ve.  Utiliser un tel pattern permet de faciliter grandement le mécanisme \textit{Undo/redo} présent dans tous programmes de nos jours.

Il est à noter que l'annulation d'une commande ne change pas l'état de l'application.

\item \textbf{Singleton}

Pour la \textit{déserialisation} des fichiers XML, on mis en place une classe singleton dont l'instanciation plusieurs fois n'aurait pas eu de sens. 

\end{itemize}
\newpage
\section{Diagramme de séquence du calcul de la tournée à partir d'une demande de livraison}
\begin{figure}[h!]
    \centering
    \includegraphics[scale=0.28]{tournee2.png}
    \caption{DSS calcul de la tournee}
\end{figure}
\chapter{Implémentation et tests}
\section{Code du prototype et des tests unitaires}
Pour consulter le code du prototype et des tests veuillez regarder directement le dossier \textit{src} fourni avec cette documentation.
\section{Documentation JavaDoc du code}
Pour consulter le javadoc du prototype, veuillez regarder directement le dossier \textit{src} fourni avec cette documentation.
\section{Diagramme de packages et de classes rétro-générés à partir du code}
\subsection{Controleur}
\begin{figure}[h!]
    \centering
    \includegraphics[scale=0.38]{DcCommande.png}
    \caption{Package Commande}
\end{figure}
\newpage
\begin{figure}[h!]
    \centering
    \includegraphics[scale=0.35]{DcEtat.png}
    \caption{Package Etat}
\end{figure}
\begin{figure}[h!]
    \centering
    \includegraphics[scale=0.35]{DcObservateur.png}
    \caption{Package Observeurs}
\end{figure}
\subsection{Modele}
\begin{figure}[h!]
    \centering
    \includegraphics[scale=0.40]{DcPersistance.png}
    \caption{Package Persistance}
\end{figure}
\newpage
\subsection{Vue}
\begin{figure}[h!]
    \centering
    \includegraphics[scale=0.28]{DcVue.png}
    \caption{Package Vue}
\end{figure}
\chapter{Bilan}
\section{Planning effectif du projet}
Le planning prévisionnel nous a beaucoup aidé dans une réalisation bien pensée ce projet. La connaissance des dépendances entre les modules nous a permi d'éviter qu'un développeur est bloqué et de réaliser les tâches dans le bon ordre.\\
\begin{longtable}{|l|l|r|l|}
\hline
\textbf{Developpeur}&\multicolumn{2}{c|}{\textbf{Tache}}&\textbf{Total}\\
\hline

\hline
Alexis A. & Conception CUs & 4h & 20h\\
\cline{2-3}
& Diagramme état-transition & 4h & ~\\
\cline{2-3}
& Diagramme séquence calcul tournée & 4h & ~\\
\cline{2-3}
& Développement calcul tournée & 8h & ~\\
\hline

\hline
Jolan & Implementation JUnit & 4h & 48h\\
\cline{2-3}
& Implementation du Dijkstra & 25h & ~\\
\cline{2-3}
& Développement modèle & 10 & ~\\
\cline{2-3}
& Conception diagramme modèle & 2h & ~\\
\cline{2-3}
& Conception diagramme séquence & 3h & ~\\
\cline{2-3}
& Correctinos des bugs & 4h & ~\\
\hline

\hline
Mohamed & Conception diagramme modèle & 2h & 57h\\
\cline{2-3}
& Développement Vue & 29h & ~\\
\cline{2-3}
& Développement Contrôleur & 16h & ~\\
\cline{2-3}
& Documentation & 4h & ~\\
\cline{2-3}
& Correction des bugs & 5h & ~\\
\cline{2-3}
& Redaction Design Patterns & 1h & ~\\
\hline

\hline
Alexis P. & Conception CUs & 4h & 20h\\
\cline{2-3}
& Diagramme état-transition & 4h & ~\\
\cline{2-3}
& Diagramme séquence calcul tournée & 4h & ~\\
\cline{2-3}
& Développement calcul tournée & 8h & ~\\
\hline

\hline
Robin & Conception IHM & 6h & 14h\\
\cline{2-3}
& JUnit modèle & 8h & ~\\
\hline

\hline
Max & Planning previsionel & 4h & 71h\\
\cline{2-3}
& Conception architecture et interfaces & 9h & ~\\
\cline{2-3}
& Development Controleur & 18h & ~\\
\cline{2-3}
& Development Modele & 26h & ~\\
\cline{2-3}
& Gestion des taches & 3h & ~\\
\cline{2-3}
& Gestion et réalisation des livrables & 8h & ~\\
\cline{2-3}
& Development JUnit Modele & 3h & ~\\
\hline

\hline
David & Conception CUs & 4h & 52h\\
\cline{2-3}
& Diagramme Sequence & 3h & ~\\
\cline{2-3}
& Développement vue & 20h & ~\\
\cline{2-3}
& Documentation & 8h & ~\\
\cline{2-3}
& JUnit Controleur & 5h & ~\\
\cline{2-3}
& Correction des bugs & 10h & ~\\
\cline{2-3}
& Redaction rapport & 2h & ~\\
\hline
\end{longtable}
~\\Comme le relevé plus haut le montre, le planning prévisionnel était suffisamment précis en ce qui concerne la planification et conception. Par contre, nous n'avons pas réussi à réaliser les implémentations dans le temps prévu. Il y a plusieurs facteurs qui ont contribué à ce retardement. Parmi eux voici les plus importants:
\begin{itemize}
	\item{On était forcé de profondément changer l'architecture dans un moment où la plupart des classes étaient déjà implémentées.}
	\item{Même si pas demandé, nous étions motivés de trouver la meilleure solution possible. Pour ça nous avons parfois implémentés plusieurs alternatives. L'implémentation de l'annulation des commandes est un bon exemple, il sera présenté plus tard dans ce rapport.}
	\item{Les interfaces et classes fournies, qui étaient censées nous simplifier le calcul du chemin optimal, nous ont posés quelques problèmes de compréhension. Il nous était pas évident comment il fallait les utiliser pour en profiter dans le cadre du projet.}
\end{itemize}
\section{Bilan humain et technique}
\subsection{Bilan humain}
Coordonner le projet et notamment distribuer les tâches dans l'équipe était un devoir délicat. D'un côté, un but était la parallélisation des fils de développement. D'autre part, il y avait beaucoup de dépendances entre les étapes prévus. C'est pour ça qu'au début du projet, la priorité était d'identifier une procédure raisonnable qui permettait d'identifier les étapes critiques. Comme je n'ais jamais été en charge d'une équipe aussi nombreuse et puissante je dois admettre que j'avais d'abord sous-estimé ce défi.\\
Le résultat de cet effort nous a bien permis de profiter au maximum des ressources humaines. Néanmoins, il était important de tenir toute l'équipe au courant en ce que regarde les développements effectués en parallèle. En particulier, la convention de bien commenter son code, en combinaison avec une communication fréquente, nous a permis de maîtriser cette mission.\\
Personnellement, je trouvais éprouvant de trouver une bonne granularité pour la distribution des tâches dans l'équipe. Si une charge était trop petite, on risque de perdre du temps en expliquant le contexte. Aurait-elle pu être réalisée plus rapidement directement par la personne qui s'en est déjà occupée ? Sinon, quand les tâches sont trop complexes, on risque qu'un développeur sera bloqué et exclu des événements qui se passent à côté. Comme la section précédente l'a relevé, le nombre d'heures passées sur le projet par développeur n'est pas parfaitement équilibré. Cependant on ne doit pas oublier que les chiffres ne correspondent pas toujours précisément au travail réalisé. Notamment le temps passé sur la planification, communication et réflexion n'est pas facile à mesurer. C'est pour cela qu'il me semble plus avéré d'évaluer l'équipe comme une entité indivisible.\\
Parlant de l'équipe, on peut résumer qu'on était enchanté de pouvoir observer la réalisation de nos conceptions. Même s'il y avait parfois des défis qu'on n'avait pas prévu, l'équipe était toujours motivé pour discuter et trouver la meilleure solution possible.
\subsection{Bilan technique}
Pour réaliser ce projet, nous avons profité d'une grande variété des techniques. Dans le but de permettre à chaque développeur de travailler avec son IDE préféré, nous avons décidé de ne pas garder les fichiers de gestion d'IDE dans le répertoire git. Grâce à cela, nous avons réussi à travailler avec les IDEs \textit{NetBeans}, \textit{Eclipse} et \textit{IntelliJ} en même temps. 

Concernant les fichiers de conception, nous avons également utilisé plusieurs outils :
\begin{itemize}
	\item{La conception des diagrammes de classes était réalisée avec le logiciel \textit{UMLet} \footnote{http://www.umlet.com/}. Les fichiers utilisés par ce logiciel étaient également partagés dans le répertoire git.}
	\item{Pour tous les autres diagrammes nous avons profité des outils en ligne, particulièrement \textit{Draw.io}, dont une intégration est disponible avec \textit{Google Drive}.}
	\item{Pour automatiquement charger les bibliothèques dont nous avons besoin et automatiser les tests, nous avons intégré \textit{Apache Maven}. Son intégration était très confortable pour l'équipe parce qu'on n'a jamais perdu du temps en résolvant les dépendances soulevées par un collègue. A coté de Maven on a utilisé un outils d'intégration continue \href{https://travis-ci.org/}{\textbf{Travis-ci}}} qui se marie très bien avec Maven et qui permet de lancer les tests à chaque commit que l'on faisait. Avec cet outils on pouvait très facilement détecter les régressions dans le code et voir les changements ayant apporté la régression.
	\item{Les tests fonctionnels et unitaires étaient réalises en utilisant le framework \textit{JUnit}. Bien que les tests ont grandement contribué à trouver et résoudre des erreurs, nous aurions encore pu intensifier leur intégration. Nous avons suivi le conseil de ne jamais laisser un développeur tester son propre code. C'est en général une bonne pratique, mais nous avons parfois pas établi assez de communication entre un auteur d'une classe et le testeur. Ça peut causer que les tests ne passent pas, car ils n'utilisent pas les interfaces de la manière prévue.}
\end{itemize}

\paragraph{Précisions sur l'implémentation}
Il existe quelques problèmes dont nous sommes conscients et quelques détails que nous souhaitons spécifier ici.
\paragraph{}
Tout d'abord, nous étions très heureux d'avoir pu utiliser JavaFX, le dernier framework de développement de client lourd de Java, bien que les ordinateurs du département n'avaient pas Java 8 d'installé. Nous sommes réjouis de voir que le département fait des efforts pour nous permettre d'utiliser les technologies du moment (JavaFX, Androïd dans le cours d'IHM, ...).
\paragraph{}
Il est possible que lors du redimensionnement de la fenêtre alors que le plan est déjà chargé, il peut problème d'affichage. Ce problème survient essentiellement quand on met la fenêtre en pleine écran directement. De plus, après des agrandissements et des zooms, nous avons remarqué qu'il peut y avoir un décalage entre la position de la souris et la position des intersections sur la partie graphique. Les intersections passent en surbrillance seulement si le curseur est un peu en dessous à droite de la position du rond blanc sur le plan.
\paragraph{}
Pour le patron de conception Etat (\textit{State}), nous avons préféré recréer un objet de l'état correspond à chaque modification, au lieu d'avoir une instance constante (\textit{final}) et statique de chaque état que nous souhaitons réutiliser. Ce choix a été fait car nous voulions avoir un comportement unique à chaque passage dans un état. Ce comportement se situe dans le constructeur de chaque état.
\paragraph{}
Notre patron de conception Commande n'est pas totalement respecté. Dû à des contraintes de temps trop importante et la focalisation sur d'autres fonctionnalités, l'annulation et le rétablissement d'une commande se fait par la restitution du modèle à ce moment. Notre patron Commande est en réalité un mélange entre Commande et Mememto, stockant l'état d'un objet à un certain moment. Pour la copie profonde du modèle, nous n'utilisons pas le patron de conception Prototype, déconseillé par la communauté \footnote{http://stackoverflow.com/questions/64036/how-do-you-make-a-deep-copy-of-an-object-in-java}. Nous sommes conscients que cette manière de faire est beaucoup plus gourmande en mémoire et également en temps processeur. Une amélioration sera vraisemblablement proposée lors du prochain sprint. Afin que la mémoire ne croît pas à l'infini pendant la durée de vie de l'application, nous stockons seulement les dix dernières commandes (copie du modèle du coup) exécutées et les dix dernières annulées.
\paragraph{}
Lors de la création d'une livraison, l'identifiant du client est initialisé à -1.
\paragraph{}
Nous souhaitons, dans un futur proche, améliorer le rapport généré. Celui-ci ne signale pas les retards, ni les attentes endurés par le chauffeur. Il est pour le moment minimaliste. Nous aurions aimé avoir un rapport en HTML avec un peu de style CSS pour avoir un résultat plus attractif.
\paragraph{}
Lors de l'échange de deux livraisons successives, une erreur se produit. Nous n'avons pas eu le temps de corriger ce disfonctionnement de l'application.

\paragraph{}
Finalement, on peut résumer qu'il s'agissait d'un projet complexe et que nous n'avions pas beaucoup de temps pour le réaliser. Nous pensons avoir choisi les bons outils. Ceux-ci nous ont permis d'avancer rapidement et créer un produit fiable.
\appendix{}
\end{document}