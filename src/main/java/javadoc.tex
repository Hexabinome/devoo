\documentclass[11pt,a4paper]{report}
\usepackage{color}
\usepackage{ifthen}
\usepackage{ifpdf}
%\usepackage[headings]{fullpage}
\usepackage{listings}
\lstset{language=Java,breaklines=true}
\ifpdf \usepackage[pdftex, pdfpagemode={UseOutlines},bookmarks,colorlinks,linkcolor={blue},plainpages=false,pdfpagelabels,citecolor={red},breaklinks=true]{hyperref}
  \usepackage[pdftex]{graphicx}
  \pdfcompresslevel=9
  \DeclareGraphicsRule{*}{mps}{*}{}
\else
  \usepackage[dvips]{graphicx}
\fi

\newcommand{\entityintro}[3]{%
  \hbox to \hsize{%
    \vbox{%
      \hbox to .2in{}%
    }%
    {\bf  #1}%
    \dotfill\pageref{#2}%
  }
  \makebox[\hsize]{%
    \parbox{.4in}{}%
    \parbox[l]{5in}{%
      \vspace{1mm}%
      #3%
      \vspace{1mm}%
    }%
  }%
}
\newcommand{\refdefined}[1]{
\expandafter\ifx\csname r@#1\endcsname\relax
\relax\else
{$($in \ref{#1}, page \pageref{#1}$)$}\fi}
\date{\today}
\title{Java Doc Projet DevOO}
\author{Hexanome: 4105}
\chardef\textbackslash=`\\
\begin{document}
\maketitle
\sloppy
\addtocontents{toc}{\protect\markboth{Contents}{Contents}}
\tableofcontents
\chapter*{Class Hierarchy}{
\thispagestyle{empty}
\markboth{Class Hierarchy}{Class Hierarchy}
\addcontentsline{toc}{chapter}{Class Hierarchy}
\section*{Classes}
{\raggedright
\hspace{0.0cm} $\bullet$ java.lang.Object {\tiny \refdefined{java.lang.Object}} \\
\hspace{1.0cm} $\bullet$ DevOO {\tiny \refdefined{.DevOO}} \\
\hspace{1.0cm} $\bullet$ controleur.Controleur {\tiny \refdefined{controleur.Controleur}} \\
\hspace{1.0cm} $\bullet$ controleur.ControleurDonnees {\tiny \refdefined{controleur.ControleurDonnees}} \\
\hspace{1.0cm} $\bullet$ java.lang.Enum {\tiny \refdefined{java.lang.Enum}} \\
\hspace{2.0cm} $\bullet$ vue.ObjetVisualisable.CouleurTexte {\tiny \refdefined{vue.ObjetVisualisable.CouleurTexte}} \\
\hspace{1.0cm} $\bullet$ javafx.application.Application {\tiny \refdefined{javafx.application.Application}} \\
\hspace{2.0cm} $\bullet$ vue.FenetrePrincipale {\tiny \refdefined{vue.FenetrePrincipale}} \\
\hspace{1.0cm} $\bullet$ javafx.scene.Node {\tiny \refdefined{javafx.scene.Node}} \\
\hspace{2.0cm} $\bullet$ javafx.scene.Parent {\tiny \refdefined{javafx.scene.Parent}} \\
\hspace{3.0cm} $\bullet$ javafx.scene.layout.Region {\tiny \refdefined{javafx.scene.layout.Region}} \\
\hspace{4.0cm} $\bullet$ javafx.scene.control.Control {\tiny \refdefined{javafx.scene.control.Control}} \\
\hspace{5.0cm} $\bullet$ javafx.scene.control.Labeled {\tiny \refdefined{javafx.scene.control.Labeled}} \\
\hspace{6.0cm} $\bullet$ javafx.scene.control.ButtonBase {\tiny \refdefined{javafx.scene.control.ButtonBase}} \\
\hspace{7.0cm} $\bullet$ javafx.scene.control.Button {\tiny \refdefined{javafx.scene.control.Button}} \\
\hspace{8.0cm} $\bullet$ vue.BoutonObservateur {\tiny \refdefined{vue.BoutonObservateur}} \\
\hspace{2.0cm} $\bullet$ javafx.scene.shape.Shape {\tiny \refdefined{javafx.scene.shape.Shape}} \\
\hspace{3.0cm} $\bullet$ javafx.scene.text.Text {\tiny \refdefined{javafx.scene.text.Text}} \\
\hspace{4.0cm} $\bullet$ vue.ObserveurMessageChamps {\tiny \refdefined{vue.ObserveurMessageChamps}} \\
\hspace{1.0cm} $\bullet$ vue.ObjetVisualisable {\tiny \refdefined{vue.ObjetVisualisable}} \\
\hspace{2.0cm} $\bullet$ vue.DetailFenetre {\tiny \refdefined{vue.DetailFenetre}} \\
\hspace{2.0cm} $\bullet$ vue.DetailLivraison {\tiny \refdefined{vue.DetailLivraison}} \\
\hspace{1.0cm} $\bullet$ vue.VueGraphiqueAideur {\tiny \refdefined{vue.VueGraphiqueAideur}} \\
\hspace{1.0cm} $\bullet$ vue.VuePrincipale {\tiny \refdefined{vue.VuePrincipale}} \\
\hspace{1.0cm} $\bullet$ vue.VueTextuelle {\tiny \refdefined{vue.VueTextuelle}} \\
}
\section*{Interfaces}
\hspace{0.0cm} $\bullet$ controleur.ControleurInterface {\tiny \refdefined{controleur.ControleurInterface}} \\
}
\chapter{Package controleur}{
\label{controleur}\hypertarget{controleur}{}
\hskip -.05in
\hbox to \hsize{\textit{ Package Contents\hfil Page}}
\vskip .13in
\hbox{{\bf  Interfaces}}
\entityintro{ControleurInterface}{controleur.ControleurInterface}{Joue le rôle de façade pour le controleur.}
\vskip .13in
\hbox{{\bf  Classes}}
\entityintro{Controleur}{controleur.Controleur}{Implémente l'interface contrôleur.}
\entityintro{ControleurDonnees}{controleur.ControleurDonnees}{Cette classe contient les données nécessaires pour la gestion des états.}
\vskip .1in
\vskip .1in
\section{\label{controleur.ControleurInterface}Interface ControleurInterface}{
\hypertarget{controleur.ControleurInterface}{}\vskip .1in 
Joue le rôle de façade pour le controleur. La vue ne vera que cette façade pour appeler les méthodes du controleur\vskip .1in 
\subsection{Declaration}{
\begin{lstlisting}[frame=none]
public interface ControleurInterface
\end{lstlisting}
\subsection{All known subinterfaces}{Controleur\small{\refdefined{controleur.Controleur}}}
\subsection{All classes known to implement interface}{Controleur\small{\refdefined{controleur.Controleur}}}
\subsection{Method summary}{
\begin{verse}
\hyperlink{controleur.ControleurInterface.ajouterActivationFonctionnalitesObservateur(controleur.observateur.ActivationFonctionnalitesObservateur)}{{\bf ajouterActivationFonctionnalitesObservateur(ActivationFonctionnalitesObservateur)}} Ajoute un observateur pour l'activation des fonctionnalités principales de l'application\\
\hyperlink{controleur.ControleurInterface.ajouterActivationOuvrirDemandeObservateur(controleur.observateur.ActivationOuvrirDemandeObservateur)}{{\bf ajouterActivationOuvrirDemandeObservateur(ActivationOuvrirDemandeObservateur)}} Ajoute un observateur des changement du plan\\
\hyperlink{controleur.ControleurInterface.ajouterActivationOuvrirPlanObservateur(controleur.observateur.ActivationOuvrirPlanObservateur)}{{\bf ajouterActivationOuvrirPlanObservateur(ActivationOuvrirPlanObservateur)}} Ajoute un observateur du chargement du plan\\
\hyperlink{controleur.ControleurInterface.ajouterAnnulerCommandeObservateur(controleur.observateur.AnnulerCommandeObservateur)}{{\bf ajouterAnnulerCommandeObservateur(AnnulerCommandeObservateur)}} Ajoute un observateur à l'annulation d'une commande\\
\hyperlink{controleur.ControleurInterface.ajouterMessageObservateur(controleur.observateur.MessageObservateur)}{{\bf ajouterMessageObservateur(MessageObservateur)}} Ajoute un observateur des messages envoyés\\
\hyperlink{controleur.ControleurInterface.ajouterModeleObservateur(controleur.observateur.ModeleObservateur)}{{\bf ajouterModeleObservateur(ModeleObservateur)}} Ajoute un observateur au changement du modèle\\
\hyperlink{controleur.ControleurInterface.ajouterPlanChargeObserveur(controleur.observateur.PlanChargeObservateur)}{{\bf ajouterPlanChargeObserveur(PlanChargeObservateur)}} \\
\hyperlink{controleur.ControleurInterface.ajouterRetablirCommandeObservateur(controleur.observateur.RetablirCommandeObservateur)}{{\bf ajouterRetablirCommandeObservateur(RetablirCommandeObservateur)}} Ajoute un observateur au rétablissement d'une commande\\
\hyperlink{controleur.ControleurInterface.ajouterTourneeObservateur(controleur.observateur.ActivationFonctionnalitesObservateur)}{{\bf ajouterTourneeObservateur(ActivationFonctionnalitesObservateur)}} Ajoute un observateur à la tournée\\
\hyperlink{controleur.ControleurInterface.chargerLivraisons(java.io.File)}{{\bf chargerLivraisons(File)}} Cette methode essaye de convertir un fichier XML dans sa représentation d'objets.\\
\hyperlink{controleur.ControleurInterface.chargerPlan(java.io.File)}{{\bf chargerPlan(File)}} Cette methode essaye de convertir un fichier XML dans sa représentation d'objets.\\
\hyperlink{controleur.ControleurInterface.clicAnnuler()}{{\bf clicAnnuler()}} Appel lors d'un clic sur Annuler\\
\hyperlink{controleur.ControleurInterface.clicCalculTournee()}{{\bf clicCalculTournee()}} Appel lors du clic sur le calcul de la tournée\\
\hyperlink{controleur.ControleurInterface.clicDroit()}{{\bf clicDroit()}} Appel lors du clic droit\\
\hyperlink{controleur.ControleurInterface.clicOutilAjouter()}{{\bf clicOutilAjouter()}} Appel lors du clic pour passer dans le mode d'ajout\\
\hyperlink{controleur.ControleurInterface.clicOutilEchanger()}{{\bf clicOutilEchanger()}} Appel lors du clic pour passer dans le mode d'échange\\
\hyperlink{controleur.ControleurInterface.clicOutilSupprimer()}{{\bf clicOutilSupprimer()}} Appel lors du clic pour passer dans le mode de suppression\\
\hyperlink{controleur.ControleurInterface.clicRetablir()}{{\bf clicRetablir()}} Appel lors d'un clic sur Rétablir\\
\hyperlink{controleur.ControleurInterface.clicSurLivraison(int)}{{\bf clicSurLivraison(int)}} Appel quand il y a un clic sur une livraison\\
\hyperlink{controleur.ControleurInterface.clicSurPlan(int)}{{\bf clicSurPlan(int)}} Appel quand il y a un clic sur plan\\
\hyperlink{controleur.ControleurInterface.genererFeuilleDeRoute(java.io.File)}{{\bf genererFeuilleDeRoute(File)}} Génère la feuille de route\\
\hyperlink{controleur.ControleurInterface.getModele()}{{\bf getModele()}} Retourn le modèle\\
\hyperlink{controleur.ControleurInterface.getPlanDeVille()}{{\bf getPlanDeVille()}} Retourne le plande la ville\\
\end{verse}
}
\subsection{Methods}{
\vskip -2em
\begin{itemize}
\item{ 
\index{ajouterActivationFonctionnalitesObservateur(ActivationFonctionnalitesObservateur)}
\hypertarget{controleur.ControleurInterface.ajouterActivationFonctionnalitesObservateur(controleur.observateur.ActivationFonctionnalitesObservateur)}{{\bf  ajouterActivationFonctionnalitesObservateur}\\}
\begin{lstlisting}[frame=none]
void ajouterActivationFonctionnalitesObservateur(observateur.ActivationFonctionnalitesObservateur observeur)\end{lstlisting} %end signature
\begin{itemize}
\item{
{\bf  Description}

Ajoute un observateur pour l'activation des fonctionnalités principales de l'application
}
\item{
{\bf  Parameters}
  \begin{itemize}
   \item{
\texttt{observeur} -- }
  \end{itemize}
}%end item
\end{itemize}
}%end item
\item{ 
\index{ajouterActivationOuvrirDemandeObservateur(ActivationOuvrirDemandeObservateur)}
\hypertarget{controleur.ControleurInterface.ajouterActivationOuvrirDemandeObservateur(controleur.observateur.ActivationOuvrirDemandeObservateur)}{{\bf  ajouterActivationOuvrirDemandeObservateur}\\}
\begin{lstlisting}[frame=none]
void ajouterActivationOuvrirDemandeObservateur(observateur.ActivationOuvrirDemandeObservateur planObserveur)\end{lstlisting} %end signature
\begin{itemize}
\item{
{\bf  Description}

Ajoute un observateur des changement du plan
}
\item{
{\bf  Parameters}
  \begin{itemize}
   \item{
\texttt{planObserveur} -- }
  \end{itemize}
}%end item
\end{itemize}
}%end item
\item{ 
\index{ajouterActivationOuvrirPlanObservateur(ActivationOuvrirPlanObservateur)}
\hypertarget{controleur.ControleurInterface.ajouterActivationOuvrirPlanObservateur(controleur.observateur.ActivationOuvrirPlanObservateur)}{{\bf  ajouterActivationOuvrirPlanObservateur}\\}
\begin{lstlisting}[frame=none]
void ajouterActivationOuvrirPlanObservateur(observateur.ActivationOuvrirPlanObservateur chargementPlanObserveur)\end{lstlisting} %end signature
\begin{itemize}
\item{
{\bf  Description}

Ajoute un observateur du chargement du plan
}
\item{
{\bf  Parameters}
  \begin{itemize}
   \item{
\texttt{chargementPlanObserveur} -- }
  \end{itemize}
}%end item
\end{itemize}
}%end item
\item{ 
\index{ajouterAnnulerCommandeObservateur(AnnulerCommandeObservateur)}
\hypertarget{controleur.ControleurInterface.ajouterAnnulerCommandeObservateur(controleur.observateur.AnnulerCommandeObservateur)}{{\bf  ajouterAnnulerCommandeObservateur}\\}
\begin{lstlisting}[frame=none]
void ajouterAnnulerCommandeObservateur(observateur.AnnulerCommandeObservateur annulerCommandeObserveur)\end{lstlisting} %end signature
\begin{itemize}
\item{
{\bf  Description}

Ajoute un observateur à l'annulation d'une commande
}
\item{
{\bf  Parameters}
  \begin{itemize}
   \item{
\texttt{annulerCommandeObserveur} -- }
  \end{itemize}
}%end item
\end{itemize}
}%end item
\item{ 
\index{ajouterMessageObservateur(MessageObservateur)}
\hypertarget{controleur.ControleurInterface.ajouterMessageObservateur(controleur.observateur.MessageObservateur)}{{\bf  ajouterMessageObservateur}\\}
\begin{lstlisting}[frame=none]
void ajouterMessageObservateur(observateur.MessageObservateur obs)\end{lstlisting} %end signature
\begin{itemize}
\item{
{\bf  Description}

Ajoute un observateur des messages envoyés
}
\item{
{\bf  Parameters}
  \begin{itemize}
   \item{
\texttt{obs} -- }
  \end{itemize}
}%end item
\end{itemize}
}%end item
\item{ 
\index{ajouterModeleObservateur(ModeleObservateur)}
\hypertarget{controleur.ControleurInterface.ajouterModeleObservateur(controleur.observateur.ModeleObservateur)}{{\bf  ajouterModeleObservateur}\\}
\begin{lstlisting}[frame=none]
void ajouterModeleObservateur(observateur.ModeleObservateur observeur)\end{lstlisting} %end signature
\begin{itemize}
\item{
{\bf  Description}

Ajoute un observateur au changement du modèle
}
\item{
{\bf  Parameters}
  \begin{itemize}
   \item{
\texttt{observeur} -- }
  \end{itemize}
}%end item
\end{itemize}
}%end item
\item{ 
\index{ajouterPlanChargeObserveur(PlanChargeObservateur)}
\hypertarget{controleur.ControleurInterface.ajouterPlanChargeObserveur(controleur.observateur.PlanChargeObservateur)}{{\bf  ajouterPlanChargeObserveur}\\}
\begin{lstlisting}[frame=none]
void ajouterPlanChargeObserveur(observateur.PlanChargeObservateur planChargeObservateur)\end{lstlisting} %end signature
}%end item
\item{ 
\index{ajouterRetablirCommandeObservateur(RetablirCommandeObservateur)}
\hypertarget{controleur.ControleurInterface.ajouterRetablirCommandeObservateur(controleur.observateur.RetablirCommandeObservateur)}{{\bf  ajouterRetablirCommandeObservateur}\\}
\begin{lstlisting}[frame=none]
void ajouterRetablirCommandeObservateur(observateur.RetablirCommandeObservateur retablirCommandeObserveur)\end{lstlisting} %end signature
\begin{itemize}
\item{
{\bf  Description}

Ajoute un observateur au rétablissement d'une commande
}
\item{
{\bf  Parameters}
  \begin{itemize}
   \item{
\texttt{retablirCommandeObserveur} -- }
  \end{itemize}
}%end item
\end{itemize}
}%end item
\item{ 
\index{ajouterTourneeObservateur(ActivationFonctionnalitesObservateur)}
\hypertarget{controleur.ControleurInterface.ajouterTourneeObservateur(controleur.observateur.ActivationFonctionnalitesObservateur)}{{\bf  ajouterTourneeObservateur}\\}
\begin{lstlisting}[frame=none]
void ajouterTourneeObservateur(observateur.ActivationFonctionnalitesObservateur tourneeObserveur)\end{lstlisting} %end signature
\begin{itemize}
\item{
{\bf  Description}

Ajoute un observateur à la tournée
}
\item{
{\bf  Parameters}
  \begin{itemize}
   \item{
\texttt{tourneeObserveur} -- }
  \end{itemize}
}%end item
\end{itemize}
}%end item
\item{ 
\index{chargerLivraisons(File)}
\hypertarget{controleur.ControleurInterface.chargerLivraisons(java.io.File)}{{\bf  chargerLivraisons}\\}
\begin{lstlisting}[frame=none]
void chargerLivraisons(java.io.File fichierLivraisons) throws java.lang.Exception\end{lstlisting} %end signature
\begin{itemize}
\item{
{\bf  Description}

Cette methode essaye de convertir un fichier XML dans sa représentation d'objets.
}
\item{
{\bf  Parameters}
  \begin{itemize}
   \item{
\texttt{fichierLivraisons} -- Objet File qui représente le fichier XML}
  \end{itemize}
}%end item
\item{{\bf  Throws}
  \begin{itemize}
   \item{\vskip -.6ex \texttt{java.lang.Exception} -- Lance une exception s'il y a une erreur lors du chargement des objets}
  \end{itemize}
}%end item
\end{itemize}
}%end item
\item{ 
\index{chargerPlan(File)}
\hypertarget{controleur.ControleurInterface.chargerPlan(java.io.File)}{{\bf  chargerPlan}\\}
\begin{lstlisting}[frame=none]
void chargerPlan(java.io.File fichierPlan) throws java.lang.Exception\end{lstlisting} %end signature
\begin{itemize}
\item{
{\bf  Description}

Cette methode essaye de convertir un fichier XML dans sa représentation d'objets.
}
\item{
{\bf  Parameters}
  \begin{itemize}
   \item{
\texttt{fichierPlan} -- Objet File qui représente le fichier XML}
  \end{itemize}
}%end item
\item{{\bf  Throws}
  \begin{itemize}
   \item{\vskip -.6ex \texttt{java.lang.Exception} -- Lance une exception s'il y a une erreur lors du chargement des objets}
  \end{itemize}
}%end item
\end{itemize}
}%end item
\item{ 
\index{clicAnnuler()}
\hypertarget{controleur.ControleurInterface.clicAnnuler()}{{\bf  clicAnnuler}\\}
\begin{lstlisting}[frame=none]
void clicAnnuler()\end{lstlisting} %end signature
\begin{itemize}
\item{
{\bf  Description}

Appel lors d'un clic sur Annuler
}
\end{itemize}
}%end item
\item{ 
\index{clicCalculTournee()}
\hypertarget{controleur.ControleurInterface.clicCalculTournee()}{{\bf  clicCalculTournee}\\}
\begin{lstlisting}[frame=none]
void clicCalculTournee()\end{lstlisting} %end signature
\begin{itemize}
\item{
{\bf  Description}

Appel lors du clic sur le calcul de la tournée
}
\end{itemize}
}%end item
\item{ 
\index{clicDroit()}
\hypertarget{controleur.ControleurInterface.clicDroit()}{{\bf  clicDroit}\\}
\begin{lstlisting}[frame=none]
void clicDroit()\end{lstlisting} %end signature
\begin{itemize}
\item{
{\bf  Description}

Appel lors du clic droit
}
\end{itemize}
}%end item
\item{ 
\index{clicOutilAjouter()}
\hypertarget{controleur.ControleurInterface.clicOutilAjouter()}{{\bf  clicOutilAjouter}\\}
\begin{lstlisting}[frame=none]
void clicOutilAjouter()\end{lstlisting} %end signature
\begin{itemize}
\item{
{\bf  Description}

Appel lors du clic pour passer dans le mode d'ajout
}
\end{itemize}
}%end item
\item{ 
\index{clicOutilEchanger()}
\hypertarget{controleur.ControleurInterface.clicOutilEchanger()}{{\bf  clicOutilEchanger}\\}
\begin{lstlisting}[frame=none]
void clicOutilEchanger()\end{lstlisting} %end signature
\begin{itemize}
\item{
{\bf  Description}

Appel lors du clic pour passer dans le mode d'échange
}
\end{itemize}
}%end item
\item{ 
\index{clicOutilSupprimer()}
\hypertarget{controleur.ControleurInterface.clicOutilSupprimer()}{{\bf  clicOutilSupprimer}\\}
\begin{lstlisting}[frame=none]
void clicOutilSupprimer()\end{lstlisting} %end signature
\begin{itemize}
\item{
{\bf  Description}

Appel lors du clic pour passer dans le mode de suppression
}
\end{itemize}
}%end item
\item{ 
\index{clicRetablir()}
\hypertarget{controleur.ControleurInterface.clicRetablir()}{{\bf  clicRetablir}\\}
\begin{lstlisting}[frame=none]
void clicRetablir()\end{lstlisting} %end signature
\begin{itemize}
\item{
{\bf  Description}

Appel lors d'un clic sur Rétablir
}
\end{itemize}
}%end item
\item{ 
\index{clicSurLivraison(int)}
\hypertarget{controleur.ControleurInterface.clicSurLivraison(int)}{{\bf  clicSurLivraison}\\}
\begin{lstlisting}[frame=none]
void clicSurLivraison(int livraisonId)\end{lstlisting} %end signature
\begin{itemize}
\item{
{\bf  Description}

Appel quand il y a un clic sur une livraison
}
\item{
{\bf  Parameters}
  \begin{itemize}
   \item{
\texttt{livraisonId} -- L'identifiant de la livraison}
  \end{itemize}
}%end item
\end{itemize}
}%end item
\item{ 
\index{clicSurPlan(int)}
\hypertarget{controleur.ControleurInterface.clicSurPlan(int)}{{\bf  clicSurPlan}\\}
\begin{lstlisting}[frame=none]
void clicSurPlan(int intersectionId)\end{lstlisting} %end signature
\begin{itemize}
\item{
{\bf  Description}

Appel quand il y a un clic sur plan
}
\item{
{\bf  Parameters}
  \begin{itemize}
   \item{
\texttt{intersectionId} -- L'identifiant de l'intersection cliqué}
  \end{itemize}
}%end item
\end{itemize}
}%end item
\item{ 
\index{genererFeuilleDeRoute(File)}
\hypertarget{controleur.ControleurInterface.genererFeuilleDeRoute(java.io.File)}{{\bf  genererFeuilleDeRoute}\\}
\begin{lstlisting}[frame=none]
void genererFeuilleDeRoute(java.io.File fichier) throws controleur.commande.CommandeException\end{lstlisting} %end signature
\begin{itemize}
\item{
{\bf  Description}

Génère la feuille de route
}
\item{
{\bf  Parameters}
  \begin{itemize}
   \item{
\texttt{fichier} -- Le fichier dans lequel on devra écrire la feuille de route}
  \end{itemize}
}%end item
\item{{\bf  Throws}
  \begin{itemize}
   \item{\vskip -.6ex \texttt{controleur.commande.CommandeException} -- Une erreur lors de l'exécution de la commande de génération}
  \end{itemize}
}%end item
\end{itemize}
}%end item
\item{ 
\index{getModele()}
\hypertarget{controleur.ControleurInterface.getModele()}{{\bf  getModele}\\}
\begin{lstlisting}[frame=none]
modele.donneesxml.ModeleLecture getModele()\end{lstlisting} %end signature
\begin{itemize}
\item{
{\bf  Description}

Retourn le modèle
}
\item{{\bf  Returns} -- 
Le modèle actuel en lecture 
}%end item
\end{itemize}
}%end item
\item{ 
\index{getPlanDeVille()}
\hypertarget{controleur.ControleurInterface.getPlanDeVille()}{{\bf  getPlanDeVille}\\}
\begin{lstlisting}[frame=none]
modele.donneesxml.PlanDeVille getPlanDeVille()\end{lstlisting} %end signature
\begin{itemize}
\item{
{\bf  Description}

Retourne le plande la ville
}
\item{{\bf  Returns} -- 
Le plan de la ville actuellement chargé 
}%end item
\end{itemize}
}%end item
\end{itemize}
}
}
\section{\label{controleur.Controleur}Class Controleur}{
\hypertarget{controleur.Controleur}{}\vskip .1in 
Implémente l'interface contrôleur. Point d'entrée principal pour toutes les intéractions avec le package vue.\vskip .1in 
\subsection{Declaration}{
\begin{lstlisting}[frame=none]
public class Controleur
 extends java.lang.Object implements ControleurInterface\end{lstlisting}
\subsection{Constructor summary}{
\begin{verse}
\hyperlink{controleur.Controleur()}{{\bf Controleur()}} Constructeur public du contrôleur\\
\end{verse}
}
\subsection{Method summary}{
\begin{verse}
\hyperlink{controleur.Controleur.ajouterActivationFonctionnalitesObservateur(controleur.observateur.ActivationFonctionnalitesObservateur)}{{\bf ajouterActivationFonctionnalitesObservateur(ActivationFonctionnalitesObservateur)}} \\
\hyperlink{controleur.Controleur.ajouterActivationOuvrirDemandeObservateur(controleur.observateur.ActivationOuvrirDemandeObservateur)}{{\bf ajouterActivationOuvrirDemandeObservateur(ActivationOuvrirDemandeObservateur)}} \\
\hyperlink{controleur.Controleur.ajouterActivationOuvrirPlanObservateur(controleur.observateur.ActivationOuvrirPlanObservateur)}{{\bf ajouterActivationOuvrirPlanObservateur(ActivationOuvrirPlanObservateur)}} \\
\hyperlink{controleur.Controleur.ajouterAnnulerCommandeObservateur(controleur.observateur.AnnulerCommandeObservateur)}{{\bf ajouterAnnulerCommandeObservateur(AnnulerCommandeObservateur)}} \\
\hyperlink{controleur.Controleur.ajouterMessageObservateur(controleur.observateur.MessageObservateur)}{{\bf ajouterMessageObservateur(MessageObservateur)}} \\
\hyperlink{controleur.Controleur.ajouterModeleObservateur(controleur.observateur.ModeleObservateur)}{{\bf ajouterModeleObservateur(ModeleObservateur)}} \\
\hyperlink{controleur.Controleur.ajouterPlanChargeObserveur(controleur.observateur.PlanChargeObservateur)}{{\bf ajouterPlanChargeObserveur(PlanChargeObservateur)}} \\
\hyperlink{controleur.Controleur.ajouterRetablirCommandeObservateur(controleur.observateur.RetablirCommandeObservateur)}{{\bf ajouterRetablirCommandeObservateur(RetablirCommandeObservateur)}} \\
\hyperlink{controleur.Controleur.ajouterTourneeObservateur(controleur.observateur.ActivationFonctionnalitesObservateur)}{{\bf ajouterTourneeObservateur(ActivationFonctionnalitesObservateur)}} \\
\hyperlink{controleur.Controleur.chargerLivraisons(java.io.File)}{{\bf chargerLivraisons(File)}} \\
\hyperlink{controleur.Controleur.chargerPlan(java.io.File)}{{\bf chargerPlan(File)}} \\
\hyperlink{controleur.Controleur.clicAnnuler()}{{\bf clicAnnuler()}} \\
\hyperlink{controleur.Controleur.clicCalculTournee()}{{\bf clicCalculTournee()}} \\
\hyperlink{controleur.Controleur.clicDroit()}{{\bf clicDroit()}} \\
\hyperlink{controleur.Controleur.clicOutilAjouter()}{{\bf clicOutilAjouter()}} \\
\hyperlink{controleur.Controleur.clicOutilEchanger()}{{\bf clicOutilEchanger()}} \\
\hyperlink{controleur.Controleur.clicOutilSupprimer()}{{\bf clicOutilSupprimer()}} \\
\hyperlink{controleur.Controleur.clicRetablir()}{{\bf clicRetablir()}} \\
\hyperlink{controleur.Controleur.clicSurLivraison(int)}{{\bf clicSurLivraison(int)}} \\
\hyperlink{controleur.Controleur.clicSurPlan(int)}{{\bf clicSurPlan(int)}} \\
\hyperlink{controleur.Controleur.genererFeuilleDeRoute(java.io.File)}{{\bf genererFeuilleDeRoute(File)}} \\
\hyperlink{controleur.Controleur.getModele()}{{\bf getModele()}} \\
\hyperlink{controleur.Controleur.getPlanDeVille()}{{\bf getPlanDeVille()}} \\
\end{verse}
}
\subsection{Constructors}{
\vskip -2em
\begin{itemize}
\item{ 
\index{Controleur()}
\hypertarget{controleur.Controleur()}{{\bf  Controleur}\\}
\begin{lstlisting}[frame=none]
public Controleur()\end{lstlisting} %end signature
\begin{itemize}
\item{
{\bf  Description}

Constructeur public du contrôleur
}
\end{itemize}
}%end item
\end{itemize}
}
\subsection{Methods}{
\vskip -2em
\begin{itemize}
\item{ 
\index{ajouterActivationFonctionnalitesObservateur(ActivationFonctionnalitesObservateur)}
\hypertarget{controleur.Controleur.ajouterActivationFonctionnalitesObservateur(controleur.observateur.ActivationFonctionnalitesObservateur)}{{\bf  ajouterActivationFonctionnalitesObservateur}\\}
\begin{lstlisting}[frame=none]
void ajouterActivationFonctionnalitesObservateur(observateur.ActivationFonctionnalitesObservateur observeur)\end{lstlisting} %end signature
\begin{itemize}
\item{
{\bf  Description copied from \hyperlink{controleur.ControleurInterface}{ControleurInterface}{\small \refdefined{controleur.ControleurInterface}} }

Ajoute un observateur pour l'activation des fonctionnalités principales de l'application
}
\item{
{\bf  Parameters}
  \begin{itemize}
   \item{
\texttt{observeur} -- }
  \end{itemize}
}%end item
\end{itemize}
}%end item
\item{ 
\index{ajouterActivationOuvrirDemandeObservateur(ActivationOuvrirDemandeObservateur)}
\hypertarget{controleur.Controleur.ajouterActivationOuvrirDemandeObservateur(controleur.observateur.ActivationOuvrirDemandeObservateur)}{{\bf  ajouterActivationOuvrirDemandeObservateur}\\}
\begin{lstlisting}[frame=none]
void ajouterActivationOuvrirDemandeObservateur(observateur.ActivationOuvrirDemandeObservateur planObserveur)\end{lstlisting} %end signature
\begin{itemize}
\item{
{\bf  Description copied from \hyperlink{controleur.ControleurInterface}{ControleurInterface}{\small \refdefined{controleur.ControleurInterface}} }

Ajoute un observateur des changement du plan
}
\item{
{\bf  Parameters}
  \begin{itemize}
   \item{
\texttt{planObserveur} -- }
  \end{itemize}
}%end item
\end{itemize}
}%end item
\item{ 
\index{ajouterActivationOuvrirPlanObservateur(ActivationOuvrirPlanObservateur)}
\hypertarget{controleur.Controleur.ajouterActivationOuvrirPlanObservateur(controleur.observateur.ActivationOuvrirPlanObservateur)}{{\bf  ajouterActivationOuvrirPlanObservateur}\\}
\begin{lstlisting}[frame=none]
void ajouterActivationOuvrirPlanObservateur(observateur.ActivationOuvrirPlanObservateur chargementPlanObserveur)\end{lstlisting} %end signature
\begin{itemize}
\item{
{\bf  Description copied from \hyperlink{controleur.ControleurInterface}{ControleurInterface}{\small \refdefined{controleur.ControleurInterface}} }

Ajoute un observateur du chargement du plan
}
\item{
{\bf  Parameters}
  \begin{itemize}
   \item{
\texttt{chargementPlanObserveur} -- }
  \end{itemize}
}%end item
\end{itemize}
}%end item
\item{ 
\index{ajouterAnnulerCommandeObservateur(AnnulerCommandeObservateur)}
\hypertarget{controleur.Controleur.ajouterAnnulerCommandeObservateur(controleur.observateur.AnnulerCommandeObservateur)}{{\bf  ajouterAnnulerCommandeObservateur}\\}
\begin{lstlisting}[frame=none]
void ajouterAnnulerCommandeObservateur(observateur.AnnulerCommandeObservateur annulerCommandeObserveur)\end{lstlisting} %end signature
\begin{itemize}
\item{
{\bf  Description copied from \hyperlink{controleur.ControleurInterface}{ControleurInterface}{\small \refdefined{controleur.ControleurInterface}} }

Ajoute un observateur à l'annulation d'une commande
}
\item{
{\bf  Parameters}
  \begin{itemize}
   \item{
\texttt{annulerCommandeObserveur} -- }
  \end{itemize}
}%end item
\end{itemize}
}%end item
\item{ 
\index{ajouterMessageObservateur(MessageObservateur)}
\hypertarget{controleur.Controleur.ajouterMessageObservateur(controleur.observateur.MessageObservateur)}{{\bf  ajouterMessageObservateur}\\}
\begin{lstlisting}[frame=none]
void ajouterMessageObservateur(observateur.MessageObservateur obs)\end{lstlisting} %end signature
\begin{itemize}
\item{
{\bf  Description copied from \hyperlink{controleur.ControleurInterface}{ControleurInterface}{\small \refdefined{controleur.ControleurInterface}} }

Ajoute un observateur des messages envoyés
}
\item{
{\bf  Parameters}
  \begin{itemize}
   \item{
\texttt{obs} -- }
  \end{itemize}
}%end item
\end{itemize}
}%end item
\item{ 
\index{ajouterModeleObservateur(ModeleObservateur)}
\hypertarget{controleur.Controleur.ajouterModeleObservateur(controleur.observateur.ModeleObservateur)}{{\bf  ajouterModeleObservateur}\\}
\begin{lstlisting}[frame=none]
void ajouterModeleObservateur(observateur.ModeleObservateur observeur)\end{lstlisting} %end signature
\begin{itemize}
\item{
{\bf  Description copied from \hyperlink{controleur.ControleurInterface}{ControleurInterface}{\small \refdefined{controleur.ControleurInterface}} }

Ajoute un observateur au changement du modèle
}
\item{
{\bf  Parameters}
  \begin{itemize}
   \item{
\texttt{observeur} -- }
  \end{itemize}
}%end item
\end{itemize}
}%end item
\item{ 
\index{ajouterPlanChargeObserveur(PlanChargeObservateur)}
\hypertarget{controleur.Controleur.ajouterPlanChargeObserveur(controleur.observateur.PlanChargeObservateur)}{{\bf  ajouterPlanChargeObserveur}\\}
\begin{lstlisting}[frame=none]
void ajouterPlanChargeObserveur(observateur.PlanChargeObservateur planChargeObservateur)\end{lstlisting} %end signature
}%end item
\item{ 
\index{ajouterRetablirCommandeObservateur(RetablirCommandeObservateur)}
\hypertarget{controleur.Controleur.ajouterRetablirCommandeObservateur(controleur.observateur.RetablirCommandeObservateur)}{{\bf  ajouterRetablirCommandeObservateur}\\}
\begin{lstlisting}[frame=none]
void ajouterRetablirCommandeObservateur(observateur.RetablirCommandeObservateur retablirCommandeObserveur)\end{lstlisting} %end signature
\begin{itemize}
\item{
{\bf  Description copied from \hyperlink{controleur.ControleurInterface}{ControleurInterface}{\small \refdefined{controleur.ControleurInterface}} }

Ajoute un observateur au rétablissement d'une commande
}
\item{
{\bf  Parameters}
  \begin{itemize}
   \item{
\texttt{retablirCommandeObserveur} -- }
  \end{itemize}
}%end item
\end{itemize}
}%end item
\item{ 
\index{ajouterTourneeObservateur(ActivationFonctionnalitesObservateur)}
\hypertarget{controleur.Controleur.ajouterTourneeObservateur(controleur.observateur.ActivationFonctionnalitesObservateur)}{{\bf  ajouterTourneeObservateur}\\}
\begin{lstlisting}[frame=none]
void ajouterTourneeObservateur(observateur.ActivationFonctionnalitesObservateur tourneeObserveur)\end{lstlisting} %end signature
\begin{itemize}
\item{
{\bf  Description copied from \hyperlink{controleur.ControleurInterface}{ControleurInterface}{\small \refdefined{controleur.ControleurInterface}} }

Ajoute un observateur à la tournée
}
\item{
{\bf  Parameters}
  \begin{itemize}
   \item{
\texttt{tourneeObserveur} -- }
  \end{itemize}
}%end item
\end{itemize}
}%end item
\item{ 
\index{chargerLivraisons(File)}
\hypertarget{controleur.Controleur.chargerLivraisons(java.io.File)}{{\bf  chargerLivraisons}\\}
\begin{lstlisting}[frame=none]
void chargerLivraisons(java.io.File fichierLivraisons) throws java.lang.Exception\end{lstlisting} %end signature
\begin{itemize}
\item{
{\bf  Description copied from \hyperlink{controleur.ControleurInterface}{ControleurInterface}{\small \refdefined{controleur.ControleurInterface}} }

Cette methode essaye de convertir un fichier XML dans sa représentation d'objets.
}
\item{
{\bf  Parameters}
  \begin{itemize}
   \item{
\texttt{fichierLivraisons} -- Objet File qui représente le fichier XML}
  \end{itemize}
}%end item
\item{{\bf  Throws}
  \begin{itemize}
   \item{\vskip -.6ex \texttt{java.lang.Exception} -- Lance une exception s'il y a une erreur lors du chargement des objets}
  \end{itemize}
}%end item
\end{itemize}
}%end item
\item{ 
\index{chargerPlan(File)}
\hypertarget{controleur.Controleur.chargerPlan(java.io.File)}{{\bf  chargerPlan}\\}
\begin{lstlisting}[frame=none]
void chargerPlan(java.io.File fichierPlan) throws java.lang.Exception\end{lstlisting} %end signature
\begin{itemize}
\item{
{\bf  Description copied from \hyperlink{controleur.ControleurInterface}{ControleurInterface}{\small \refdefined{controleur.ControleurInterface}} }

Cette methode essaye de convertir un fichier XML dans sa représentation d'objets.
}
\item{
{\bf  Parameters}
  \begin{itemize}
   \item{
\texttt{fichierPlan} -- Objet File qui représente le fichier XML}
  \end{itemize}
}%end item
\item{{\bf  Throws}
  \begin{itemize}
   \item{\vskip -.6ex \texttt{java.lang.Exception} -- Lance une exception s'il y a une erreur lors du chargement des objets}
  \end{itemize}
}%end item
\end{itemize}
}%end item
\item{ 
\index{clicAnnuler()}
\hypertarget{controleur.Controleur.clicAnnuler()}{{\bf  clicAnnuler}\\}
\begin{lstlisting}[frame=none]
void clicAnnuler()\end{lstlisting} %end signature
\begin{itemize}
\item{
{\bf  Description copied from \hyperlink{controleur.ControleurInterface}{ControleurInterface}{\small \refdefined{controleur.ControleurInterface}} }

Appel lors d'un clic sur Annuler
}
\end{itemize}
}%end item
\item{ 
\index{clicCalculTournee()}
\hypertarget{controleur.Controleur.clicCalculTournee()}{{\bf  clicCalculTournee}\\}
\begin{lstlisting}[frame=none]
void clicCalculTournee()\end{lstlisting} %end signature
\begin{itemize}
\item{
{\bf  Description copied from \hyperlink{controleur.ControleurInterface}{ControleurInterface}{\small \refdefined{controleur.ControleurInterface}} }

Appel lors du clic sur le calcul de la tournée
}
\end{itemize}
}%end item
\item{ 
\index{clicDroit()}
\hypertarget{controleur.Controleur.clicDroit()}{{\bf  clicDroit}\\}
\begin{lstlisting}[frame=none]
void clicDroit()\end{lstlisting} %end signature
\begin{itemize}
\item{
{\bf  Description copied from \hyperlink{controleur.ControleurInterface}{ControleurInterface}{\small \refdefined{controleur.ControleurInterface}} }

Appel lors du clic droit
}
\end{itemize}
}%end item
\item{ 
\index{clicOutilAjouter()}
\hypertarget{controleur.Controleur.clicOutilAjouter()}{{\bf  clicOutilAjouter}\\}
\begin{lstlisting}[frame=none]
void clicOutilAjouter()\end{lstlisting} %end signature
\begin{itemize}
\item{
{\bf  Description copied from \hyperlink{controleur.ControleurInterface}{ControleurInterface}{\small \refdefined{controleur.ControleurInterface}} }

Appel lors du clic pour passer dans le mode d'ajout
}
\end{itemize}
}%end item
\item{ 
\index{clicOutilEchanger()}
\hypertarget{controleur.Controleur.clicOutilEchanger()}{{\bf  clicOutilEchanger}\\}
\begin{lstlisting}[frame=none]
void clicOutilEchanger()\end{lstlisting} %end signature
\begin{itemize}
\item{
{\bf  Description copied from \hyperlink{controleur.ControleurInterface}{ControleurInterface}{\small \refdefined{controleur.ControleurInterface}} }

Appel lors du clic pour passer dans le mode d'échange
}
\end{itemize}
}%end item
\item{ 
\index{clicOutilSupprimer()}
\hypertarget{controleur.Controleur.clicOutilSupprimer()}{{\bf  clicOutilSupprimer}\\}
\begin{lstlisting}[frame=none]
void clicOutilSupprimer()\end{lstlisting} %end signature
\begin{itemize}
\item{
{\bf  Description copied from \hyperlink{controleur.ControleurInterface}{ControleurInterface}{\small \refdefined{controleur.ControleurInterface}} }

Appel lors du clic pour passer dans le mode de suppression
}
\end{itemize}
}%end item
\item{ 
\index{clicRetablir()}
\hypertarget{controleur.Controleur.clicRetablir()}{{\bf  clicRetablir}\\}
\begin{lstlisting}[frame=none]
void clicRetablir()\end{lstlisting} %end signature
\begin{itemize}
\item{
{\bf  Description copied from \hyperlink{controleur.ControleurInterface}{ControleurInterface}{\small \refdefined{controleur.ControleurInterface}} }

Appel lors d'un clic sur Rétablir
}
\end{itemize}
}%end item
\item{ 
\index{clicSurLivraison(int)}
\hypertarget{controleur.Controleur.clicSurLivraison(int)}{{\bf  clicSurLivraison}\\}
\begin{lstlisting}[frame=none]
void clicSurLivraison(int livraisonId)\end{lstlisting} %end signature
\begin{itemize}
\item{
{\bf  Description copied from \hyperlink{controleur.ControleurInterface}{ControleurInterface}{\small \refdefined{controleur.ControleurInterface}} }

Appel quand il y a un clic sur une livraison
}
\item{
{\bf  Parameters}
  \begin{itemize}
   \item{
\texttt{livraisonId} -- L'identifiant de la livraison}
  \end{itemize}
}%end item
\end{itemize}
}%end item
\item{ 
\index{clicSurPlan(int)}
\hypertarget{controleur.Controleur.clicSurPlan(int)}{{\bf  clicSurPlan}\\}
\begin{lstlisting}[frame=none]
void clicSurPlan(int intersectionId)\end{lstlisting} %end signature
\begin{itemize}
\item{
{\bf  Description copied from \hyperlink{controleur.ControleurInterface}{ControleurInterface}{\small \refdefined{controleur.ControleurInterface}} }

Appel quand il y a un clic sur plan
}
\item{
{\bf  Parameters}
  \begin{itemize}
   \item{
\texttt{intersectionId} -- L'identifiant de l'intersection cliqué}
  \end{itemize}
}%end item
\end{itemize}
}%end item
\item{ 
\index{genererFeuilleDeRoute(File)}
\hypertarget{controleur.Controleur.genererFeuilleDeRoute(java.io.File)}{{\bf  genererFeuilleDeRoute}\\}
\begin{lstlisting}[frame=none]
void genererFeuilleDeRoute(java.io.File fichier) throws controleur.commande.CommandeException\end{lstlisting} %end signature
\begin{itemize}
\item{
{\bf  Description copied from \hyperlink{controleur.ControleurInterface}{ControleurInterface}{\small \refdefined{controleur.ControleurInterface}} }

Génère la feuille de route
}
\item{
{\bf  Parameters}
  \begin{itemize}
   \item{
\texttt{fichier} -- Le fichier dans lequel on devra écrire la feuille de route}
  \end{itemize}
}%end item
\item{{\bf  Throws}
  \begin{itemize}
   \item{\vskip -.6ex \texttt{controleur.commande.CommandeException} -- Une erreur lors de l'exécution de la commande de génération}
  \end{itemize}
}%end item
\end{itemize}
}%end item
\item{ 
\index{getModele()}
\hypertarget{controleur.Controleur.getModele()}{{\bf  getModele}\\}
\begin{lstlisting}[frame=none]
modele.donneesxml.ModeleLecture getModele()\end{lstlisting} %end signature
\begin{itemize}
\item{
{\bf  Description copied from \hyperlink{controleur.ControleurInterface}{ControleurInterface}{\small \refdefined{controleur.ControleurInterface}} }

Retourn le modèle
}
\item{{\bf  Returns} -- 
Le modèle actuel en lecture 
}%end item
\end{itemize}
}%end item
\item{ 
\index{getPlanDeVille()}
\hypertarget{controleur.Controleur.getPlanDeVille()}{{\bf  getPlanDeVille}\\}
\begin{lstlisting}[frame=none]
modele.donneesxml.PlanDeVille getPlanDeVille()\end{lstlisting} %end signature
\begin{itemize}
\item{
{\bf  Description copied from \hyperlink{controleur.ControleurInterface}{ControleurInterface}{\small \refdefined{controleur.ControleurInterface}} }

Retourne le plande la ville
}
\item{{\bf  Returns} -- 
Le plan de la ville actuellement chargé 
}%end item
\end{itemize}
}%end item
\end{itemize}
}
}
\section{\label{controleur.ControleurDonnees}Class ControleurDonnees}{
\hypertarget{controleur.ControleurDonnees}{}\vskip .1in 
Cette classe contient les données nécessaires pour la gestion des états. On pourrait dire qu'elle représente seulement des données et devrait du coup mieux être située dans le package modèle. Par contre elle est liée à une seule IHM, du coup il y a de bonnes raisons de la laisser ici dans le controleur.\vskip .1in 
\subsection{Declaration}{
\begin{lstlisting}[frame=none]
public class ControleurDonnees
 extends java.lang.Object\end{lstlisting}
\subsection{Constructor summary}{
\begin{verse}
\hyperlink{controleur.ControleurDonnees()}{{\bf ControleurDonnees()}} \\
\end{verse}
}
\subsection{Method summary}{
\begin{verse}
\hyperlink{controleur.ControleurDonnees.ajouterActivationObservateur(controleur.observateur.ActivationFonctionnalitesObservateur)}{{\bf ajouterActivationObservateur(ActivationFonctionnalitesObservateur)}} Ajoute un observateur d'activation\\
\hyperlink{controleur.ControleurDonnees.ajouterAnnulerCommandeObservateur(controleur.observateur.AnnulerCommandeObservateur)}{{\bf ajouterAnnulerCommandeObservateur(AnnulerCommandeObservateur)}} Ajoute un observateur de la commande annuler\\
\hyperlink{controleur.ControleurDonnees.ajouterChargementPlanObservateur(controleur.observateur.ActivationOuvrirPlanObservateur)}{{\bf ajouterChargementPlanObservateur(ActivationOuvrirPlanObservateur)}} Ajoute un observateur du chargement du plan\\
\hyperlink{controleur.ControleurDonnees.ajouterCommande(controleur.commande.Commande)}{{\bf ajouterCommande(Commande)}} Ajoute une commande à l'historique\\
\hyperlink{controleur.ControleurDonnees.ajouterModeleObservateur(controleur.observateur.ModeleObservateur)}{{\bf ajouterModeleObservateur(ModeleObservateur)}} Ajoute un observateur du modèle\\
\hyperlink{controleur.ControleurDonnees.ajouterPlanChargeObservateur(controleur.observateur.PlanChargeObservateur)}{{\bf ajouterPlanChargeObservateur(PlanChargeObservateur)}} \\
\hyperlink{controleur.ControleurDonnees.ajouterPlanObservateur(controleur.observateur.ActivationOuvrirDemandeObservateur)}{{\bf ajouterPlanObservateur(ActivationOuvrirDemandeObservateur)}} \\
\hyperlink{controleur.ControleurDonnees.ajouterRetablirCommandeObservateur(controleur.observateur.RetablirCommandeObservateur)}{{\bf ajouterRetablirCommandeObservateur(RetablirCommandeObservateur)}} Ajoute un observateur de la commande rétablir\\
\hyperlink{controleur.ControleurDonnees.effacerCommandeAAnnuler()}{{\bf effacerCommandeAAnnuler()}} Efface la liste des commandes à annuler et notifie la vue qu'il doir desactiver l'élément du menu correspondant\\
\hyperlink{controleur.ControleurDonnees.effacerCommandesARetablir()}{{\bf effacerCommandesARetablir()}} Efface la liste des commandes à retablir et notifie la vue qu'elle doit désactiver l'élément du menu correspondant\\
\hyperlink{controleur.ControleurDonnees.effacerHistorique()}{{\bf effacerHistorique()}} Efface l'historique (vide les commandes annulable et rétablissable(?))\\
\hyperlink{controleur.ControleurDonnees.getHist()}{{\bf getHist()}} \\
\hyperlink{controleur.ControleurDonnees.getModele()}{{\bf getModele()}} Retourne le modèle associé\\
\hyperlink{controleur.ControleurDonnees.getPlan()}{{\bf getPlan()}} Retourne le plan de la ville\\
\hyperlink{controleur.ControleurDonnees.notifierObservateurOuvrirDemande(boolean)}{{\bf notifierObservateurOuvrirDemande(boolean)}} Notifie les observateurs du plan\\
\hyperlink{controleur.ControleurDonnees.notifierObservateurOuvrirPlan(boolean)}{{\bf notifierObservateurOuvrirPlan(boolean)}} Notifie les observateurs du chargement du plan\\
\hyperlink{controleur.ControleurDonnees.notifierObservateursActivation(boolean)}{{\bf notifierObservateursActivation(boolean)}} Notifie les observateurs de l'activation\\
\hyperlink{controleur.ControleurDonnees.notifierObservateursAnnuler(boolean)}{{\bf notifierObservateursAnnuler(boolean)}} Notifie les observateurs qu'il y a eu une annulation\\
\hyperlink{controleur.ControleurDonnees.notifierObservateursCalculTournee(boolean)}{{\bf notifierObservateursCalculTournee(boolean)}} Notifie les observateurs du calcul de la tournée\\
\hyperlink{controleur.ControleurDonnees.notifierObservateursMessage(java.lang.String)}{{\bf notifierObservateursMessage(String)}} Notifie les observateurs qu'il y a un message\\
\hyperlink{controleur.ControleurDonnees.notifierObservateursModele()}{{\bf notifierObservateursModele()}} Notifie les observateurs du changemetn du modèle\\
\hyperlink{controleur.ControleurDonnees.notifierObservateursRetablir(boolean)}{{\bf notifierObservateursRetablir(boolean)}} Notifie les observateurs qu'il y eu un rétablissement\\
\hyperlink{controleur.ControleurDonnees.notifierPlanChargeObservateur()}{{\bf notifierPlanChargeObservateur()}} \\
\hyperlink{controleur.ControleurDonnees.setHistorique(controleur.commande.Historique)}{{\bf setHistorique(Historique)}} Affecte l'historique\\
\hyperlink{controleur.ControleurDonnees.setModele(modele.donneesxml.Modele)}{{\bf setModele(Modele)}} Affecte le modèle\\
\hyperlink{controleur.ControleurDonnees.setPlan(modele.donneesxml.PlanDeVille)}{{\bf setPlan(PlanDeVille)}} Affecte le plan de la ville\\
\end{verse}
}
\subsection{Constructors}{
\vskip -2em
\begin{itemize}
\item{ 
\index{ControleurDonnees()}
\hypertarget{controleur.ControleurDonnees()}{{\bf  ControleurDonnees}\\}
\begin{lstlisting}[frame=none]
public ControleurDonnees()\end{lstlisting} %end signature
}%end item
\end{itemize}
}
\subsection{Methods}{
\vskip -2em
\begin{itemize}
\item{ 
\index{ajouterActivationObservateur(ActivationFonctionnalitesObservateur)}
\hypertarget{controleur.ControleurDonnees.ajouterActivationObservateur(controleur.observateur.ActivationFonctionnalitesObservateur)}{{\bf  ajouterActivationObservateur}\\}
\begin{lstlisting}[frame=none]
public void ajouterActivationObservateur(observateur.ActivationFonctionnalitesObservateur obs)\end{lstlisting} %end signature
\begin{itemize}
\item{
{\bf  Description}

Ajoute un observateur d'activation
}
\item{
{\bf  Parameters}
  \begin{itemize}
   \item{
\texttt{obs} -- L'objet observateur}
  \end{itemize}
}%end item
\end{itemize}
}%end item
\item{ 
\index{ajouterAnnulerCommandeObservateur(AnnulerCommandeObservateur)}
\hypertarget{controleur.ControleurDonnees.ajouterAnnulerCommandeObservateur(controleur.observateur.AnnulerCommandeObservateur)}{{\bf  ajouterAnnulerCommandeObservateur}\\}
\begin{lstlisting}[frame=none]
public void ajouterAnnulerCommandeObservateur(observateur.AnnulerCommandeObservateur obs)\end{lstlisting} %end signature
\begin{itemize}
\item{
{\bf  Description}

Ajoute un observateur de la commande annuler
}
\item{
{\bf  Parameters}
  \begin{itemize}
   \item{
\texttt{obs} -- L'objet observateur}
  \end{itemize}
}%end item
\end{itemize}
}%end item
\item{ 
\index{ajouterChargementPlanObservateur(ActivationOuvrirPlanObservateur)}
\hypertarget{controleur.ControleurDonnees.ajouterChargementPlanObservateur(controleur.observateur.ActivationOuvrirPlanObservateur)}{{\bf  ajouterChargementPlanObservateur}\\}
\begin{lstlisting}[frame=none]
public void ajouterChargementPlanObservateur(observateur.ActivationOuvrirPlanObservateur chargementPlanObserveur)\end{lstlisting} %end signature
\begin{itemize}
\item{
{\bf  Description}

Ajoute un observateur du chargement du plan
}
\item{
{\bf  Parameters}
  \begin{itemize}
   \item{
\texttt{chargementPlanObserveur} -- L'objet observateur}
  \end{itemize}
}%end item
\end{itemize}
}%end item
\item{ 
\index{ajouterCommande(Commande)}
\hypertarget{controleur.ControleurDonnees.ajouterCommande(controleur.commande.Commande)}{{\bf  ajouterCommande}\\}
\begin{lstlisting}[frame=none]
public void ajouterCommande(commande.Commande commande)\end{lstlisting} %end signature
\begin{itemize}
\item{
{\bf  Description}

Ajoute une commande à l'historique
}
\item{
{\bf  Parameters}
  \begin{itemize}
   \item{
\texttt{commande} -- Une commande exécutée}
  \end{itemize}
}%end item
\end{itemize}
}%end item
\item{ 
\index{ajouterModeleObservateur(ModeleObservateur)}
\hypertarget{controleur.ControleurDonnees.ajouterModeleObservateur(controleur.observateur.ModeleObservateur)}{{\bf  ajouterModeleObservateur}\\}
\begin{lstlisting}[frame=none]
public void ajouterModeleObservateur(observateur.ModeleObservateur obs)\end{lstlisting} %end signature
\begin{itemize}
\item{
{\bf  Description}

Ajoute un observateur du modèle
}
\item{
{\bf  Parameters}
  \begin{itemize}
   \item{
\texttt{obs} -- L'objet observateur}
  \end{itemize}
}%end item
\end{itemize}
}%end item
\item{ 
\index{ajouterPlanChargeObservateur(PlanChargeObservateur)}
\hypertarget{controleur.ControleurDonnees.ajouterPlanChargeObservateur(controleur.observateur.PlanChargeObservateur)}{{\bf  ajouterPlanChargeObservateur}\\}
\begin{lstlisting}[frame=none]
public void ajouterPlanChargeObservateur(observateur.PlanChargeObservateur planChargeObservateur)\end{lstlisting} %end signature
}%end item
\item{ 
\index{ajouterPlanObservateur(ActivationOuvrirDemandeObservateur)}
\hypertarget{controleur.ControleurDonnees.ajouterPlanObservateur(controleur.observateur.ActivationOuvrirDemandeObservateur)}{{\bf  ajouterPlanObservateur}\\}
\begin{lstlisting}[frame=none]
public void ajouterPlanObservateur(observateur.ActivationOuvrirDemandeObservateur planObserveur)\end{lstlisting} %end signature
}%end item
\item{ 
\index{ajouterRetablirCommandeObservateur(RetablirCommandeObservateur)}
\hypertarget{controleur.ControleurDonnees.ajouterRetablirCommandeObservateur(controleur.observateur.RetablirCommandeObservateur)}{{\bf  ajouterRetablirCommandeObservateur}\\}
\begin{lstlisting}[frame=none]
public void ajouterRetablirCommandeObservateur(observateur.RetablirCommandeObservateur obs)\end{lstlisting} %end signature
\begin{itemize}
\item{
{\bf  Description}

Ajoute un observateur de la commande rétablir
}
\item{
{\bf  Parameters}
  \begin{itemize}
   \item{
\texttt{obs} -- L'objet observateur}
  \end{itemize}
}%end item
\end{itemize}
}%end item
\item{ 
\index{effacerCommandeAAnnuler()}
\hypertarget{controleur.ControleurDonnees.effacerCommandeAAnnuler()}{{\bf  effacerCommandeAAnnuler}\\}
\begin{lstlisting}[frame=none]
public void effacerCommandeAAnnuler()\end{lstlisting} %end signature
\begin{itemize}
\item{
{\bf  Description}

Efface la liste des commandes à annuler et notifie la vue qu'il doir desactiver l'élément du menu correspondant
}
\end{itemize}
}%end item
\item{ 
\index{effacerCommandesARetablir()}
\hypertarget{controleur.ControleurDonnees.effacerCommandesARetablir()}{{\bf  effacerCommandesARetablir}\\}
\begin{lstlisting}[frame=none]
public void effacerCommandesARetablir()\end{lstlisting} %end signature
\begin{itemize}
\item{
{\bf  Description}

Efface la liste des commandes à retablir et notifie la vue qu'elle doit désactiver l'élément du menu correspondant
}
\end{itemize}
}%end item
\item{ 
\index{effacerHistorique()}
\hypertarget{controleur.ControleurDonnees.effacerHistorique()}{{\bf  effacerHistorique}\\}
\begin{lstlisting}[frame=none]
public void effacerHistorique()\end{lstlisting} %end signature
\begin{itemize}
\item{
{\bf  Description}

Efface l'historique (vide les commandes annulable et rétablissable(?))
}
\end{itemize}
}%end item
\item{ 
\index{getHist()}
\hypertarget{controleur.ControleurDonnees.getHist()}{{\bf  getHist}\\}
\begin{lstlisting}[frame=none]
public commande.Historique getHist()\end{lstlisting} %end signature
}%end item
\item{ 
\index{getModele()}
\hypertarget{controleur.ControleurDonnees.getModele()}{{\bf  getModele}\\}
\begin{lstlisting}[frame=none]
public modele.donneesxml.Modele getModele()\end{lstlisting} %end signature
\begin{itemize}
\item{
{\bf  Description}

Retourne le modèle associé
}
\item{{\bf  Returns} -- 
Le modèle associé 
}%end item
\end{itemize}
}%end item
\item{ 
\index{getPlan()}
\hypertarget{controleur.ControleurDonnees.getPlan()}{{\bf  getPlan}\\}
\begin{lstlisting}[frame=none]
public modele.donneesxml.PlanDeVille getPlan()\end{lstlisting} %end signature
\begin{itemize}
\item{
{\bf  Description}

Retourne le plan de la ville
}
\item{{\bf  Returns} -- 
Le plan de la ville 
}%end item
\end{itemize}
}%end item
\item{ 
\index{notifierObservateurOuvrirDemande(boolean)}
\hypertarget{controleur.ControleurDonnees.notifierObservateurOuvrirDemande(boolean)}{{\bf  notifierObservateurOuvrirDemande}\\}
\begin{lstlisting}[frame=none]
public void notifierObservateurOuvrirDemande(boolean activer)\end{lstlisting} %end signature
\begin{itemize}
\item{
{\bf  Description}

Notifie les observateurs du plan
}
\item{
{\bf  Parameters}
  \begin{itemize}
   \item{
\texttt{activer} -- Vrai s'il faut envoyer un message d'activation aux observateurs}
  \end{itemize}
}%end item
\end{itemize}
}%end item
\item{ 
\index{notifierObservateurOuvrirPlan(boolean)}
\hypertarget{controleur.ControleurDonnees.notifierObservateurOuvrirPlan(boolean)}{{\bf  notifierObservateurOuvrirPlan}\\}
\begin{lstlisting}[frame=none]
public void notifierObservateurOuvrirPlan(boolean activer)\end{lstlisting} %end signature
\begin{itemize}
\item{
{\bf  Description}

Notifie les observateurs du chargement du plan
}
\item{
{\bf  Parameters}
  \begin{itemize}
   \item{
\texttt{activer} -- Vrai s'il faut envoyer un message d'activation aux observateurs}
  \end{itemize}
}%end item
\end{itemize}
}%end item
\item{ 
\index{notifierObservateursActivation(boolean)}
\hypertarget{controleur.ControleurDonnees.notifierObservateursActivation(boolean)}{{\bf  notifierObservateursActivation}\\}
\begin{lstlisting}[frame=none]
public void notifierObservateursActivation(boolean etat)\end{lstlisting} %end signature
\begin{itemize}
\item{
{\bf  Description}

Notifie les observateurs de l'activation
}
\item{
{\bf  Parameters}
  \begin{itemize}
   \item{
\texttt{etat} -- Vrai s'il faut activer les observateurs}
  \end{itemize}
}%end item
\end{itemize}
}%end item
\item{ 
\index{notifierObservateursAnnuler(boolean)}
\hypertarget{controleur.ControleurDonnees.notifierObservateursAnnuler(boolean)}{{\bf  notifierObservateursAnnuler}\\}
\begin{lstlisting}[frame=none]
public void notifierObservateursAnnuler(boolean activation)\end{lstlisting} %end signature
\begin{itemize}
\item{
{\bf  Description}

Notifie les observateurs qu'il y a eu une annulation
}
\item{
{\bf  Parameters}
  \begin{itemize}
   \item{
\texttt{activation} -- Vrai si les observateurs doivent s'activer dans ce cas d'annulation}
  \end{itemize}
}%end item
\end{itemize}
}%end item
\item{ 
\index{notifierObservateursCalculTournee(boolean)}
\hypertarget{controleur.ControleurDonnees.notifierObservateursCalculTournee(boolean)}{{\bf  notifierObservateursCalculTournee}\\}
\begin{lstlisting}[frame=none]
public void notifierObservateursCalculTournee(boolean activation)\end{lstlisting} %end signature
\begin{itemize}
\item{
{\bf  Description}

Notifie les observateurs du calcul de la tournée
}
\item{
{\bf  Parameters}
  \begin{itemize}
   \item{
\texttt{activation} -- Vrai si les observateurs doivent s'activer}
  \end{itemize}
}%end item
\end{itemize}
}%end item
\item{ 
\index{notifierObservateursMessage(String)}
\hypertarget{controleur.ControleurDonnees.notifierObservateursMessage(java.lang.String)}{{\bf  notifierObservateursMessage}\\}
\begin{lstlisting}[frame=none]
public void notifierObservateursMessage(java.lang.String message)\end{lstlisting} %end signature
\begin{itemize}
\item{
{\bf  Description}

Notifie les observateurs qu'il y a un message
}
\item{
{\bf  Parameters}
  \begin{itemize}
   \item{
\texttt{message} -- Le message envoyé}
  \end{itemize}
}%end item
\end{itemize}
}%end item
\item{ 
\index{notifierObservateursModele()}
\hypertarget{controleur.ControleurDonnees.notifierObservateursModele()}{{\bf  notifierObservateursModele}\\}
\begin{lstlisting}[frame=none]
public void notifierObservateursModele()\end{lstlisting} %end signature
\begin{itemize}
\item{
{\bf  Description}

Notifie les observateurs du changemetn du modèle
}
\end{itemize}
}%end item
\item{ 
\index{notifierObservateursRetablir(boolean)}
\hypertarget{controleur.ControleurDonnees.notifierObservateursRetablir(boolean)}{{\bf  notifierObservateursRetablir}\\}
\begin{lstlisting}[frame=none]
public void notifierObservateursRetablir(boolean activation)\end{lstlisting} %end signature
\begin{itemize}
\item{
{\bf  Description}

Notifie les observateurs qu'il y eu un rétablissement
}
\item{
{\bf  Parameters}
  \begin{itemize}
   \item{
\texttt{activation} -- Vrai si les observateurs doivent s'activer dans ce cas de rétablissement}
  \end{itemize}
}%end item
\end{itemize}
}%end item
\item{ 
\index{notifierPlanChargeObservateur()}
\hypertarget{controleur.ControleurDonnees.notifierPlanChargeObservateur()}{{\bf  notifierPlanChargeObservateur}\\}
\begin{lstlisting}[frame=none]
public void notifierPlanChargeObservateur()\end{lstlisting} %end signature
}%end item
\item{ 
\index{setHistorique(Historique)}
\hypertarget{controleur.ControleurDonnees.setHistorique(controleur.commande.Historique)}{{\bf  setHistorique}\\}
\begin{lstlisting}[frame=none]
public void setHistorique(commande.Historique hist)\end{lstlisting} %end signature
\begin{itemize}
\item{
{\bf  Description}

Affecte l'historique
}
\item{
{\bf  Parameters}
  \begin{itemize}
   \item{
\texttt{hist} -- Le nouvel historique}
  \end{itemize}
}%end item
\end{itemize}
}%end item
\item{ 
\index{setModele(Modele)}
\hypertarget{controleur.ControleurDonnees.setModele(modele.donneesxml.Modele)}{{\bf  setModele}\\}
\begin{lstlisting}[frame=none]
public void setModele(modele.donneesxml.Modele modele)\end{lstlisting} %end signature
\begin{itemize}
\item{
{\bf  Description}

Affecte le modèle
}
\item{
{\bf  Parameters}
  \begin{itemize}
   \item{
\texttt{modele} -- Le nouveau modèle}
  \end{itemize}
}%end item
\end{itemize}
}%end item
\item{ 
\index{setPlan(PlanDeVille)}
\hypertarget{controleur.ControleurDonnees.setPlan(modele.donneesxml.PlanDeVille)}{{\bf  setPlan}\\}
\begin{lstlisting}[frame=none]
public void setPlan(modele.donneesxml.PlanDeVille plan)\end{lstlisting} %end signature
\begin{itemize}
\item{
{\bf  Description}

Affecte le plan de la ville
}
\item{
{\bf  Parameters}
  \begin{itemize}
   \item{
\texttt{plan} -- Le nouveau plan de la ville}
  \end{itemize}
}%end item
\end{itemize}
}%end item
\end{itemize}
}
}
}
\chapter{Package modele}{
\label{modele}\hypertarget{modele}{}
\hskip -.05in
\hbox to \hsize{\textit{ Package Contents\hfil Page}}
\vskip .1in
\vskip .1in
}
\chapter{Package vue}{
\label{vue}\hypertarget{vue}{}
\hskip -.05in
\hbox to \hsize{\textit{ Package Contents\hfil Page}}
\vskip .13in
\hbox{{\bf  Classes}}
\entityintro{BoutonObservateur}{vue.BoutonObservateur}{Bouton particulier qui observe les modifications au niveau du modèle pour savoir s'il doit s'activer ou pas.}
\entityintro{DetailFenetre}{vue.DetailFenetre}{Gére l'affichage sous forme textuelle des details d'une fenêtre de livraison dans la TreeTableView.}
\entityintro{DetailLivraison}{vue.DetailLivraison}{Gére l'affichage sous forme textuelle des details d'une livraison dans la TreeTableView.}
\entityintro{FenetrePrincipale}{vue.FenetrePrincipale}{Cette classe crée la fenetre principale avec ses enfants.}
\entityintro{ObjetVisualisable}{vue.ObjetVisualisable}{Cette classe permet de visualiser une fenêtre (de livraison) ou une livraison sous forme textuelle.}
\entityintro{ObjetVisualisable.CouleurTexte}{vue.ObjetVisualisable.CouleurTexte}{Différentes couleurs possibles pour un élément dans la liste.}
\entityintro{ObserveurMessageChamps}{vue.ObserveurMessageChamps}{Champ texte, écouteur des messages qui peuvent être reçu}
\entityintro{VueGraphiqueAideur}{vue.VueGraphiqueAideur}{Contient les méthodes permettant d'afficher les éléments dans la zone graphique}
\entityintro{VuePrincipale}{vue.VuePrincipale}{Cette classe joue le rôle de binding pour la fenetre principale de l'application.}
\entityintro{VueTextuelle}{vue.VueTextuelle}{Cette classe gère les livraisons et leurs horaires.}
\vskip .1in
\vskip .1in
\section{\label{vue.BoutonObservateur}Class BoutonObservateur}{
\hypertarget{vue.BoutonObservateur}{}\vskip .1in 
Bouton particulier qui observe les modifications au niveau du modèle pour savoir s'il doit s'activer ou pas.\vskip .1in 
\subsection{Declaration}{
\begin{lstlisting}[frame=none]
public class BoutonObservateur
 extends javafx.scene.control.Button implements controleur.observateur.ActivationFonctionnalitesObservateur\end{lstlisting}
\subsection{Constructor summary}{
\begin{verse}
\hyperlink{vue.BoutonObservateur()}{{\bf BoutonObservateur()}} \\
\end{verse}
}
\subsection{Method summary}{
\begin{verse}
\hyperlink{vue.BoutonObservateur.notifierObservateursActivation(boolean)}{{\bf notifierObservateursActivation(boolean)}} \\
\end{verse}
}
\subsection{Constructors}{
\vskip -2em
\begin{itemize}
\item{ 
\index{BoutonObservateur()}
\hypertarget{vue.BoutonObservateur()}{{\bf  BoutonObservateur}\\}
\begin{lstlisting}[frame=none]
public BoutonObservateur()\end{lstlisting} %end signature
}%end item
\end{itemize}
}
\subsection{Methods}{
\vskip -2em
\begin{itemize}
\item{ 
\index{notifierObservateursActivation(boolean)}
\hypertarget{vue.BoutonObservateur.notifierObservateursActivation(boolean)}{{\bf  notifierObservateursActivation}\\}
\begin{lstlisting}[frame=none]
void notifierObservateursActivation(boolean desactiver)\end{lstlisting} %end signature
\begin{itemize}
\item{
{\bf  Description copied from \hyperlink{controleur.observateur.ActivationFonctionnalitesObservateur}{controleur.observateur.ActivationFonctionnalitesObservateur}{\small \refdefined{controleur.observateur.ActivationFonctionnalitesObservateur}} }

Notifie les observateurs qui attendent un message d'activation notamment les boutons de fonctionnalités (ajouter, supprimer, echanger) qu'il faut changer l'état d'activation
}
\item{
{\bf  Parameters}
  \begin{itemize}
   \item{
\texttt{desactiver} -- Vrai si on doit désactiver les observeurs}
  \end{itemize}
}%end item
\end{itemize}
}%end item
\end{itemize}
}
\subsection{Members inherited from class Button }{
\texttt{javafx.scene.control.Button} {\small 
\refdefined{javafx.scene.control.Button}}
{\small 

\vskip -2em
\begin{itemize}
\item{\vskip -1.5ex 
\texttt{public final BooleanProperty {\bf  cancelButtonProperty}()
}%end signature
}%end item
\item{\vskip -1.5ex 
\texttt{protected Skin {\bf  createDefaultSkin}()
}%end signature
}%end item
\item{\vskip -1.5ex 
\texttt{public final BooleanProperty {\bf  defaultButtonProperty}()
}%end signature
}%end item
\item{\vskip -1.5ex 
\texttt{public void {\bf  fire}()
}%end signature
}%end item
\item{\vskip -1.5ex 
\texttt{public final boolean {\bf  isCancelButton}()
}%end signature
}%end item
\item{\vskip -1.5ex 
\texttt{public final boolean {\bf  isDefaultButton}()
}%end signature
}%end item
\item{\vskip -1.5ex 
\texttt{public final void {\bf  setCancelButton}(\texttt{boolean} {\bf  arg0})
}%end signature
}%end item
\item{\vskip -1.5ex 
\texttt{public final void {\bf  setDefaultButton}(\texttt{boolean} {\bf  arg0})
}%end signature
}%end item
\end{itemize}
}
\subsection{Members inherited from class ButtonBase }{
\texttt{javafx.scene.control.ButtonBase} {\small 
\refdefined{javafx.scene.control.ButtonBase}}
{\small 

\vskip -2em
\begin{itemize}
\item{\vskip -1.5ex 
\texttt{public void {\bf  arm}()
}%end signature
}%end item
\item{\vskip -1.5ex 
\texttt{public final ReadOnlyBooleanProperty {\bf  armedProperty}()
}%end signature
}%end item
\item{\vskip -1.5ex 
\texttt{public void {\bf  disarm}()
}%end signature
}%end item
\item{\vskip -1.5ex 
\texttt{public void {\bf  executeAccessibleAction}(\texttt{javafx.scene.AccessibleAction} {\bf  arg0},
\texttt{java.lang.Object\lbrack \rbrack } {\bf  arg1})
}%end signature
}%end item
\item{\vskip -1.5ex 
\texttt{public abstract void {\bf  fire}()
}%end signature
}%end item
\item{\vskip -1.5ex 
\texttt{public final EventHandler {\bf  getOnAction}()
}%end signature
}%end item
\item{\vskip -1.5ex 
\texttt{public final boolean {\bf  isArmed}()
}%end signature
}%end item
\item{\vskip -1.5ex 
\texttt{public final ObjectProperty {\bf  onActionProperty}()
}%end signature
}%end item
\item{\vskip -1.5ex 
\texttt{public final void {\bf  setOnAction}(\texttt{javafx.event.EventHandler} {\bf  arg0})
}%end signature
}%end item
\end{itemize}
}
\subsection{Members inherited from class Labeled }{
\texttt{javafx.scene.control.Labeled} {\small 
\refdefined{javafx.scene.control.Labeled}}
{\small 

\vskip -2em
\begin{itemize}
\item{\vskip -1.5ex 
\texttt{public final ObjectProperty {\bf  alignmentProperty}()
}%end signature
}%end item
\item{\vskip -1.5ex 
\texttt{public final ObjectProperty {\bf  contentDisplayProperty}()
}%end signature
}%end item
\item{\vskip -1.5ex 
\texttt{public final StringProperty {\bf  ellipsisStringProperty}()
}%end signature
}%end item
\item{\vskip -1.5ex 
\texttt{public final ObjectProperty {\bf  fontProperty}()
}%end signature
}%end item
\item{\vskip -1.5ex 
\texttt{public final Pos {\bf  getAlignment}()
}%end signature
}%end item
\item{\vskip -1.5ex 
\texttt{public static List {\bf  getClassCssMetaData}()
}%end signature
}%end item
\item{\vskip -1.5ex 
\texttt{public Orientation {\bf  getContentBias}()
}%end signature
}%end item
\item{\vskip -1.5ex 
\texttt{public final ContentDisplay {\bf  getContentDisplay}()
}%end signature
}%end item
\item{\vskip -1.5ex 
\texttt{public List {\bf  getControlCssMetaData}()
}%end signature
}%end item
\item{\vskip -1.5ex 
\texttt{public final String {\bf  getEllipsisString}()
}%end signature
}%end item
\item{\vskip -1.5ex 
\texttt{public final Font {\bf  getFont}()
}%end signature
}%end item
\item{\vskip -1.5ex 
\texttt{public final Node {\bf  getGraphic}()
}%end signature
}%end item
\item{\vskip -1.5ex 
\texttt{public final double {\bf  getGraphicTextGap}()
}%end signature
}%end item
\item{\vskip -1.5ex 
\texttt{public final Insets {\bf  getLabelPadding}()
}%end signature
}%end item
\item{\vskip -1.5ex 
\texttt{public final double {\bf  getLineSpacing}()
}%end signature
}%end item
\item{\vskip -1.5ex 
\texttt{public final String {\bf  getText}()
}%end signature
}%end item
\item{\vskip -1.5ex 
\texttt{public final TextAlignment {\bf  getTextAlignment}()
}%end signature
}%end item
\item{\vskip -1.5ex 
\texttt{public final Paint {\bf  getTextFill}()
}%end signature
}%end item
\item{\vskip -1.5ex 
\texttt{public final OverrunStyle {\bf  getTextOverrun}()
}%end signature
}%end item
\item{\vskip -1.5ex 
\texttt{public final ObjectProperty {\bf  graphicProperty}()
}%end signature
}%end item
\item{\vskip -1.5ex 
\texttt{public final DoubleProperty {\bf  graphicTextGapProperty}()
}%end signature
}%end item
\item{\vskip -1.5ex 
\texttt{protected Pos {\bf  impl\_cssGetAlignmentInitialValue}()
}%end signature
}%end item
\item{\vskip -1.5ex 
\texttt{public final boolean {\bf  isMnemonicParsing}()
}%end signature
}%end item
\item{\vskip -1.5ex 
\texttt{public final boolean {\bf  isUnderline}()
}%end signature
}%end item
\item{\vskip -1.5ex 
\texttt{public final boolean {\bf  isWrapText}()
}%end signature
}%end item
\item{\vskip -1.5ex 
\texttt{public final ReadOnlyObjectProperty {\bf  labelPaddingProperty}()
}%end signature
}%end item
\item{\vskip -1.5ex 
\texttt{public final DoubleProperty {\bf  lineSpacingProperty}()
}%end signature
}%end item
\item{\vskip -1.5ex 
\texttt{public final BooleanProperty {\bf  mnemonicParsingProperty}()
}%end signature
}%end item
\item{\vskip -1.5ex 
\texttt{public final void {\bf  setAlignment}(\texttt{javafx.geometry.Pos} {\bf  arg0})
}%end signature
}%end item
\item{\vskip -1.5ex 
\texttt{public final void {\bf  setContentDisplay}(\texttt{ContentDisplay} {\bf  arg0})
}%end signature
}%end item
\item{\vskip -1.5ex 
\texttt{public final void {\bf  setEllipsisString}(\texttt{java.lang.String} {\bf  arg0})
}%end signature
}%end item
\item{\vskip -1.5ex 
\texttt{public final void {\bf  setFont}(\texttt{javafx.scene.text.Font} {\bf  arg0})
}%end signature
}%end item
\item{\vskip -1.5ex 
\texttt{public final void {\bf  setGraphic}(\texttt{javafx.scene.Node} {\bf  arg0})
}%end signature
}%end item
\item{\vskip -1.5ex 
\texttt{public final void {\bf  setGraphicTextGap}(\texttt{double} {\bf  arg0})
}%end signature
}%end item
\item{\vskip -1.5ex 
\texttt{public final void {\bf  setLineSpacing}(\texttt{double} {\bf  arg0})
}%end signature
}%end item
\item{\vskip -1.5ex 
\texttt{public final void {\bf  setMnemonicParsing}(\texttt{boolean} {\bf  arg0})
}%end signature
}%end item
\item{\vskip -1.5ex 
\texttt{public final void {\bf  setText}(\texttt{java.lang.String} {\bf  arg0})
}%end signature
}%end item
\item{\vskip -1.5ex 
\texttt{public final void {\bf  setTextAlignment}(\texttt{javafx.scene.text.TextAlignment} {\bf  arg0})
}%end signature
}%end item
\item{\vskip -1.5ex 
\texttt{public final void {\bf  setTextFill}(\texttt{javafx.scene.paint.Paint} {\bf  arg0})
}%end signature
}%end item
\item{\vskip -1.5ex 
\texttt{public final void {\bf  setTextOverrun}(\texttt{OverrunStyle} {\bf  arg0})
}%end signature
}%end item
\item{\vskip -1.5ex 
\texttt{public final void {\bf  setUnderline}(\texttt{boolean} {\bf  arg0})
}%end signature
}%end item
\item{\vskip -1.5ex 
\texttt{public final void {\bf  setWrapText}(\texttt{boolean} {\bf  arg0})
}%end signature
}%end item
\item{\vskip -1.5ex 
\texttt{public final ObjectProperty {\bf  textAlignmentProperty}()
}%end signature
}%end item
\item{\vskip -1.5ex 
\texttt{public final ObjectProperty {\bf  textFillProperty}()
}%end signature
}%end item
\item{\vskip -1.5ex 
\texttt{public final ObjectProperty {\bf  textOverrunProperty}()
}%end signature
}%end item
\item{\vskip -1.5ex 
\texttt{public final StringProperty {\bf  textProperty}()
}%end signature
}%end item
\item{\vskip -1.5ex 
\texttt{public String {\bf  toString}()
}%end signature
}%end item
\item{\vskip -1.5ex 
\texttt{public final BooleanProperty {\bf  underlineProperty}()
}%end signature
}%end item
\item{\vskip -1.5ex 
\texttt{public final BooleanProperty {\bf  wrapTextProperty}()
}%end signature
}%end item
\end{itemize}
}
\subsection{Members inherited from class Control }{
\texttt{javafx.scene.control.Control} {\small 
\refdefined{javafx.scene.control.Control}}
{\small 

\vskip -2em
\begin{itemize}
\item{\vskip -1.5ex 
\texttt{protected double {\bf  computeMaxHeight}(\texttt{double} {\bf  arg0})
}%end signature
}%end item
\item{\vskip -1.5ex 
\texttt{protected double {\bf  computeMaxWidth}(\texttt{double} {\bf  arg0})
}%end signature
}%end item
\item{\vskip -1.5ex 
\texttt{protected double {\bf  computeMinHeight}(\texttt{double} {\bf  arg0})
}%end signature
}%end item
\item{\vskip -1.5ex 
\texttt{protected double {\bf  computeMinWidth}(\texttt{double} {\bf  arg0})
}%end signature
}%end item
\item{\vskip -1.5ex 
\texttt{protected double {\bf  computePrefHeight}(\texttt{double} {\bf  arg0})
}%end signature
}%end item
\item{\vskip -1.5ex 
\texttt{protected double {\bf  computePrefWidth}(\texttt{double} {\bf  arg0})
}%end signature
}%end item
\item{\vskip -1.5ex 
\texttt{public final ObjectProperty {\bf  contextMenuProperty}()
}%end signature
}%end item
\item{\vskip -1.5ex 
\texttt{protected Skin {\bf  createDefaultSkin}()
}%end signature
}%end item
\item{\vskip -1.5ex 
\texttt{public void {\bf  executeAccessibleAction}(\texttt{javafx.scene.AccessibleAction} {\bf  arg0},
\texttt{java.lang.Object\lbrack \rbrack } {\bf  arg1})
}%end signature
}%end item
\item{\vskip -1.5ex 
\texttt{public double {\bf  getBaselineOffset}()
}%end signature
}%end item
\item{\vskip -1.5ex 
\texttt{public static List {\bf  getClassCssMetaData}()
}%end signature
}%end item
\item{\vskip -1.5ex 
\texttt{public final ContextMenu {\bf  getContextMenu}()
}%end signature
}%end item
\item{\vskip -1.5ex 
\texttt{protected List {\bf  getControlCssMetaData}()
}%end signature
}%end item
\item{\vskip -1.5ex 
\texttt{public final List {\bf  getCssMetaData}()
}%end signature
}%end item
\item{\vskip -1.5ex 
\texttt{public final Skin {\bf  getSkin}()
}%end signature
}%end item
\item{\vskip -1.5ex 
\texttt{public final Tooltip {\bf  getTooltip}()
}%end signature
}%end item
\item{\vskip -1.5ex 
\texttt{protected Boolean {\bf  impl\_cssGetFocusTraversableInitialValue}()
}%end signature
}%end item
\item{\vskip -1.5ex 
\texttt{protected void {\bf  impl\_processCSS}(\texttt{javafx.beans.value.WritableValue} {\bf  arg0})
}%end signature
}%end item
\item{\vskip -1.5ex 
\texttt{public boolean {\bf  isResizable}()
}%end signature
}%end item
\item{\vskip -1.5ex 
\texttt{protected void {\bf  layoutChildren}()
}%end signature
}%end item
\item{\vskip -1.5ex 
\texttt{public Object {\bf  queryAccessibleAttribute}(\texttt{javafx.scene.AccessibleAttribute} {\bf  arg0},
\texttt{java.lang.Object\lbrack \rbrack } {\bf  arg1})
}%end signature
}%end item
\item{\vskip -1.5ex 
\texttt{public final void {\bf  setContextMenu}(\texttt{ContextMenu} {\bf  arg0})
}%end signature
}%end item
\item{\vskip -1.5ex 
\texttt{public final void {\bf  setSkin}(\texttt{Skin} {\bf  arg0})
}%end signature
}%end item
\item{\vskip -1.5ex 
\texttt{public final void {\bf  setTooltip}(\texttt{Tooltip} {\bf  arg0})
}%end signature
}%end item
\item{\vskip -1.5ex 
\texttt{protected StringProperty {\bf  skinClassNameProperty}()
}%end signature
}%end item
\item{\vskip -1.5ex 
\texttt{public final ObjectProperty {\bf  skinProperty}()
}%end signature
}%end item
\item{\vskip -1.5ex 
\texttt{public final ObjectProperty {\bf  tooltipProperty}()
}%end signature
}%end item
\end{itemize}
}
\subsection{Members inherited from class Region }{
\texttt{javafx.scene.layout.Region} {\small 
\refdefined{javafx.scene.layout.Region}}
{\small 

\vskip -2em
\begin{itemize}
\item{\vskip -1.5ex 
\texttt{public final ObjectProperty {\bf  backgroundProperty}()
}%end signature
}%end item
\item{\vskip -1.5ex 
\texttt{public final ObjectProperty {\bf  borderProperty}()
}%end signature
}%end item
\item{\vskip -1.5ex 
\texttt{public final BooleanProperty {\bf  cacheShapeProperty}()
}%end signature
}%end item
\item{\vskip -1.5ex 
\texttt{public final BooleanProperty {\bf  centerShapeProperty}()
}%end signature
}%end item
\item{\vskip -1.5ex 
\texttt{protected double {\bf  computeMaxHeight}(\texttt{double} {\bf  arg0})
}%end signature
}%end item
\item{\vskip -1.5ex 
\texttt{protected double {\bf  computeMaxWidth}(\texttt{double} {\bf  arg0})
}%end signature
}%end item
\item{\vskip -1.5ex 
\texttt{protected double {\bf  computeMinHeight}(\texttt{double} {\bf  arg0})
}%end signature
}%end item
\item{\vskip -1.5ex 
\texttt{protected double {\bf  computeMinWidth}(\texttt{double} {\bf  arg0})
}%end signature
}%end item
\item{\vskip -1.5ex 
\texttt{protected double {\bf  computePrefHeight}(\texttt{double} {\bf  arg0})
}%end signature
}%end item
\item{\vskip -1.5ex 
\texttt{protected double {\bf  computePrefWidth}(\texttt{double} {\bf  arg0})
}%end signature
}%end item
\item{\vskip -1.5ex 
\texttt{public final Background {\bf  getBackground}()
}%end signature
}%end item
\item{\vskip -1.5ex 
\texttt{public final Border {\bf  getBorder}()
}%end signature
}%end item
\item{\vskip -1.5ex 
\texttt{public static List {\bf  getClassCssMetaData}()
}%end signature
}%end item
\item{\vskip -1.5ex 
\texttt{public List {\bf  getCssMetaData}()
}%end signature
}%end item
\item{\vskip -1.5ex 
\texttt{public final double {\bf  getHeight}()
}%end signature
}%end item
\item{\vskip -1.5ex 
\texttt{public final Insets {\bf  getInsets}()
}%end signature
}%end item
\item{\vskip -1.5ex 
\texttt{public final double {\bf  getMaxHeight}()
}%end signature
}%end item
\item{\vskip -1.5ex 
\texttt{public final double {\bf  getMaxWidth}()
}%end signature
}%end item
\item{\vskip -1.5ex 
\texttt{public final double {\bf  getMinHeight}()
}%end signature
}%end item
\item{\vskip -1.5ex 
\texttt{public final double {\bf  getMinWidth}()
}%end signature
}%end item
\item{\vskip -1.5ex 
\texttt{public final Insets {\bf  getOpaqueInsets}()
}%end signature
}%end item
\item{\vskip -1.5ex 
\texttt{public final Insets {\bf  getPadding}()
}%end signature
}%end item
\item{\vskip -1.5ex 
\texttt{public final double {\bf  getPrefHeight}()
}%end signature
}%end item
\item{\vskip -1.5ex 
\texttt{public final double {\bf  getPrefWidth}()
}%end signature
}%end item
\item{\vskip -1.5ex 
\texttt{public final Shape {\bf  getShape}()
}%end signature
}%end item
\item{\vskip -1.5ex 
\texttt{public String {\bf  getUserAgentStylesheet}()
}%end signature
}%end item
\item{\vskip -1.5ex 
\texttt{public final double {\bf  getWidth}()
}%end signature
}%end item
\item{\vskip -1.5ex 
\texttt{public final ReadOnlyDoubleProperty {\bf  heightProperty}()
}%end signature
}%end item
\item{\vskip -1.5ex 
\texttt{protected boolean {\bf  impl\_computeContains}(\texttt{double} {\bf  arg0},
\texttt{double} {\bf  arg1})
}%end signature
}%end item
\item{\vskip -1.5ex 
\texttt{public BaseBounds {\bf  impl\_computeGeomBounds}(\texttt{com.sun.javafx.geom.BaseBounds} {\bf  arg0},
\texttt{com.sun.javafx.geom.transform.BaseTransform} {\bf  arg1})
}%end signature
}%end item
\item{\vskip -1.5ex 
\texttt{protected final Bounds {\bf  impl\_computeLayoutBounds}()
}%end signature
}%end item
\item{\vskip -1.5ex 
\texttt{public NGNode {\bf  impl\_createPeer}()
}%end signature
}%end item
\item{\vskip -1.5ex 
\texttt{protected final void {\bf  impl\_notifyLayoutBoundsChanged}()
}%end signature
}%end item
\item{\vskip -1.5ex 
\texttt{protected void {\bf  impl\_pickNodeLocal}(\texttt{com.sun.javafx.geom.PickRay} {\bf  arg0},
\texttt{com.sun.javafx.scene.input.PickResultChooser} {\bf  arg1})
}%end signature
}%end item
\item{\vskip -1.5ex 
\texttt{public void {\bf  impl\_updatePeer}()
}%end signature
}%end item
\item{\vskip -1.5ex 
\texttt{public final ReadOnlyObjectProperty {\bf  insetsProperty}()
}%end signature
}%end item
\item{\vskip -1.5ex 
\texttt{public final boolean {\bf  isCacheShape}()
}%end signature
}%end item
\item{\vskip -1.5ex 
\texttt{public final boolean {\bf  isCenterShape}()
}%end signature
}%end item
\item{\vskip -1.5ex 
\texttt{public boolean {\bf  isResizable}()
}%end signature
}%end item
\item{\vskip -1.5ex 
\texttt{public final boolean {\bf  isScaleShape}()
}%end signature
}%end item
\item{\vskip -1.5ex 
\texttt{public final boolean {\bf  isSnapToPixel}()
}%end signature
}%end item
\item{\vskip -1.5ex 
\texttt{protected void {\bf  layoutInArea}(\texttt{javafx.scene.Node} {\bf  arg0},
\texttt{double} {\bf  arg1},
\texttt{double} {\bf  arg2},
\texttt{double} {\bf  arg3},
\texttt{double} {\bf  arg4},
\texttt{double} {\bf  arg5},
\texttt{javafx.geometry.HPos} {\bf  arg6},
\texttt{javafx.geometry.VPos} {\bf  arg7})
}%end signature
}%end item
\item{\vskip -1.5ex 
\texttt{protected void {\bf  layoutInArea}(\texttt{javafx.scene.Node} {\bf  arg0},
\texttt{double} {\bf  arg1},
\texttt{double} {\bf  arg2},
\texttt{double} {\bf  arg3},
\texttt{double} {\bf  arg4},
\texttt{double} {\bf  arg5},
\texttt{javafx.geometry.Insets} {\bf  arg6},
\texttt{boolean} {\bf  arg7},
\texttt{boolean} {\bf  arg8},
\texttt{javafx.geometry.HPos} {\bf  arg9},
\texttt{javafx.geometry.VPos} {\bf  arg10})
}%end signature
}%end item
\item{\vskip -1.5ex 
\texttt{public static void {\bf  layoutInArea}(\texttt{javafx.scene.Node} {\bf  arg0},
\texttt{double} {\bf  arg1},
\texttt{double} {\bf  arg2},
\texttt{double} {\bf  arg3},
\texttt{double} {\bf  arg4},
\texttt{double} {\bf  arg5},
\texttt{javafx.geometry.Insets} {\bf  arg6},
\texttt{boolean} {\bf  arg7},
\texttt{boolean} {\bf  arg8},
\texttt{javafx.geometry.HPos} {\bf  arg9},
\texttt{javafx.geometry.VPos} {\bf  arg10},
\texttt{boolean} {\bf  arg11})
}%end signature
}%end item
\item{\vskip -1.5ex 
\texttt{protected void {\bf  layoutInArea}(\texttt{javafx.scene.Node} {\bf  arg0},
\texttt{double} {\bf  arg1},
\texttt{double} {\bf  arg2},
\texttt{double} {\bf  arg3},
\texttt{double} {\bf  arg4},
\texttt{double} {\bf  arg5},
\texttt{javafx.geometry.Insets} {\bf  arg6},
\texttt{javafx.geometry.HPos} {\bf  arg7},
\texttt{javafx.geometry.VPos} {\bf  arg8})
}%end signature
}%end item
\item{\vskip -1.5ex 
\texttt{public final double {\bf  maxHeight}(\texttt{double} {\bf  arg0})
}%end signature
}%end item
\item{\vskip -1.5ex 
\texttt{public final DoubleProperty {\bf  maxHeightProperty}()
}%end signature
}%end item
\item{\vskip -1.5ex 
\texttt{public final double {\bf  maxWidth}(\texttt{double} {\bf  arg0})
}%end signature
}%end item
\item{\vskip -1.5ex 
\texttt{public final DoubleProperty {\bf  maxWidthProperty}()
}%end signature
}%end item
\item{\vskip -1.5ex 
\texttt{public final double {\bf  minHeight}(\texttt{double} {\bf  arg0})
}%end signature
}%end item
\item{\vskip -1.5ex 
\texttt{public final DoubleProperty {\bf  minHeightProperty}()
}%end signature
}%end item
\item{\vskip -1.5ex 
\texttt{public final double {\bf  minWidth}(\texttt{double} {\bf  arg0})
}%end signature
}%end item
\item{\vskip -1.5ex 
\texttt{public final DoubleProperty {\bf  minWidthProperty}()
}%end signature
}%end item
\item{\vskip -1.5ex 
\texttt{public final ObjectProperty {\bf  opaqueInsetsProperty}()
}%end signature
}%end item
\item{\vskip -1.5ex 
\texttt{public final ObjectProperty {\bf  paddingProperty}()
}%end signature
}%end item
\item{\vskip -1.5ex 
\texttt{protected void {\bf  positionInArea}(\texttt{javafx.scene.Node} {\bf  arg0},
\texttt{double} {\bf  arg1},
\texttt{double} {\bf  arg2},
\texttt{double} {\bf  arg3},
\texttt{double} {\bf  arg4},
\texttt{double} {\bf  arg5},
\texttt{javafx.geometry.HPos} {\bf  arg6},
\texttt{javafx.geometry.VPos} {\bf  arg7})
}%end signature
}%end item
\item{\vskip -1.5ex 
\texttt{public static void {\bf  positionInArea}(\texttt{javafx.scene.Node} {\bf  arg0},
\texttt{double} {\bf  arg1},
\texttt{double} {\bf  arg2},
\texttt{double} {\bf  arg3},
\texttt{double} {\bf  arg4},
\texttt{double} {\bf  arg5},
\texttt{javafx.geometry.Insets} {\bf  arg6},
\texttt{javafx.geometry.HPos} {\bf  arg7},
\texttt{javafx.geometry.VPos} {\bf  arg8},
\texttt{boolean} {\bf  arg9})
}%end signature
}%end item
\item{\vskip -1.5ex 
\texttt{public final double {\bf  prefHeight}(\texttt{double} {\bf  arg0})
}%end signature
}%end item
\item{\vskip -1.5ex 
\texttt{public final DoubleProperty {\bf  prefHeightProperty}()
}%end signature
}%end item
\item{\vskip -1.5ex 
\texttt{public final double {\bf  prefWidth}(\texttt{double} {\bf  arg0})
}%end signature
}%end item
\item{\vskip -1.5ex 
\texttt{public final DoubleProperty {\bf  prefWidthProperty}()
}%end signature
}%end item
\item{\vskip -1.5ex 
\texttt{public void {\bf  resize}(\texttt{double} {\bf  arg0},
\texttt{double} {\bf  arg1})
}%end signature
}%end item
\item{\vskip -1.5ex 
\texttt{public final BooleanProperty {\bf  scaleShapeProperty}()
}%end signature
}%end item
\item{\vskip -1.5ex 
\texttt{public final void {\bf  setBackground}(\texttt{Background} {\bf  arg0})
}%end signature
}%end item
\item{\vskip -1.5ex 
\texttt{public final void {\bf  setBorder}(\texttt{Border} {\bf  arg0})
}%end signature
}%end item
\item{\vskip -1.5ex 
\texttt{public final void {\bf  setCacheShape}(\texttt{boolean} {\bf  arg0})
}%end signature
}%end item
\item{\vskip -1.5ex 
\texttt{public final void {\bf  setCenterShape}(\texttt{boolean} {\bf  arg0})
}%end signature
}%end item
\item{\vskip -1.5ex 
\texttt{protected void {\bf  setHeight}(\texttt{double} {\bf  arg0})
}%end signature
}%end item
\item{\vskip -1.5ex 
\texttt{public final void {\bf  setMaxHeight}(\texttt{double} {\bf  arg0})
}%end signature
}%end item
\item{\vskip -1.5ex 
\texttt{public void {\bf  setMaxSize}(\texttt{double} {\bf  arg0},
\texttt{double} {\bf  arg1})
}%end signature
}%end item
\item{\vskip -1.5ex 
\texttt{public final void {\bf  setMaxWidth}(\texttt{double} {\bf  arg0})
}%end signature
}%end item
\item{\vskip -1.5ex 
\texttt{public final void {\bf  setMinHeight}(\texttt{double} {\bf  arg0})
}%end signature
}%end item
\item{\vskip -1.5ex 
\texttt{public void {\bf  setMinSize}(\texttt{double} {\bf  arg0},
\texttt{double} {\bf  arg1})
}%end signature
}%end item
\item{\vskip -1.5ex 
\texttt{public final void {\bf  setMinWidth}(\texttt{double} {\bf  arg0})
}%end signature
}%end item
\item{\vskip -1.5ex 
\texttt{public final void {\bf  setOpaqueInsets}(\texttt{javafx.geometry.Insets} {\bf  arg0})
}%end signature
}%end item
\item{\vskip -1.5ex 
\texttt{public final void {\bf  setPadding}(\texttt{javafx.geometry.Insets} {\bf  arg0})
}%end signature
}%end item
\item{\vskip -1.5ex 
\texttt{public final void {\bf  setPrefHeight}(\texttt{double} {\bf  arg0})
}%end signature
}%end item
\item{\vskip -1.5ex 
\texttt{public void {\bf  setPrefSize}(\texttt{double} {\bf  arg0},
\texttt{double} {\bf  arg1})
}%end signature
}%end item
\item{\vskip -1.5ex 
\texttt{public final void {\bf  setPrefWidth}(\texttt{double} {\bf  arg0})
}%end signature
}%end item
\item{\vskip -1.5ex 
\texttt{public final void {\bf  setScaleShape}(\texttt{boolean} {\bf  arg0})
}%end signature
}%end item
\item{\vskip -1.5ex 
\texttt{public final void {\bf  setShape}(\texttt{javafx.scene.shape.Shape} {\bf  arg0})
}%end signature
}%end item
\item{\vskip -1.5ex 
\texttt{public final void {\bf  setSnapToPixel}(\texttt{boolean} {\bf  arg0})
}%end signature
}%end item
\item{\vskip -1.5ex 
\texttt{protected void {\bf  setWidth}(\texttt{double} {\bf  arg0})
}%end signature
}%end item
\item{\vskip -1.5ex 
\texttt{public final ObjectProperty {\bf  shapeProperty}()
}%end signature
}%end item
\item{\vskip -1.5ex 
\texttt{public final double {\bf  snappedBottomInset}()
}%end signature
}%end item
\item{\vskip -1.5ex 
\texttt{public final double {\bf  snappedLeftInset}()
}%end signature
}%end item
\item{\vskip -1.5ex 
\texttt{public final double {\bf  snappedRightInset}()
}%end signature
}%end item
\item{\vskip -1.5ex 
\texttt{public final double {\bf  snappedTopInset}()
}%end signature
}%end item
\item{\vskip -1.5ex 
\texttt{protected double {\bf  snapPosition}(\texttt{double} {\bf  arg0})
}%end signature
}%end item
\item{\vskip -1.5ex 
\texttt{protected double {\bf  snapSize}(\texttt{double} {\bf  arg0})
}%end signature
}%end item
\item{\vskip -1.5ex 
\texttt{protected double {\bf  snapSpace}(\texttt{double} {\bf  arg0})
}%end signature
}%end item
\item{\vskip -1.5ex 
\texttt{public final BooleanProperty {\bf  snapToPixelProperty}()
}%end signature
}%end item
\item{\vskip -1.5ex 
\texttt{public static final {\bf  USE\_COMPUTED\_SIZE}}%end signature
}%end item
\item{\vskip -1.5ex 
\texttt{public static final {\bf  USE\_PREF\_SIZE}}%end signature
}%end item
\item{\vskip -1.5ex 
\texttt{public final ReadOnlyDoubleProperty {\bf  widthProperty}()
}%end signature
}%end item
\end{itemize}
}
\subsection{Members inherited from class Parent }{
\texttt{javafx.scene.Parent} {\small 
\refdefined{javafx.scene.Parent}}
{\small 

\vskip -2em
\begin{itemize}
\item{\vskip -1.5ex 
\texttt{protected double {\bf  computeMinHeight}(\texttt{double} {\bf  arg0})
}%end signature
}%end item
\item{\vskip -1.5ex 
\texttt{protected double {\bf  computeMinWidth}(\texttt{double} {\bf  arg0})
}%end signature
}%end item
\item{\vskip -1.5ex 
\texttt{protected double {\bf  computePrefHeight}(\texttt{double} {\bf  arg0})
}%end signature
}%end item
\item{\vskip -1.5ex 
\texttt{protected double {\bf  computePrefWidth}(\texttt{double} {\bf  arg0})
}%end signature
}%end item
\item{\vskip -1.5ex 
\texttt{public double {\bf  getBaselineOffset}()
}%end signature
}%end item
\item{\vskip -1.5ex 
\texttt{protected ObservableList {\bf  getChildren}()
}%end signature
}%end item
\item{\vskip -1.5ex 
\texttt{public ObservableList {\bf  getChildrenUnmodifiable}()
}%end signature
}%end item
\item{\vskip -1.5ex 
\texttt{public final ParentTraversalEngine {\bf  getImpl\_traversalEngine}()
}%end signature
}%end item
\item{\vskip -1.5ex 
\texttt{protected List {\bf  getManagedChildren}()
}%end signature
}%end item
\item{\vskip -1.5ex 
\texttt{public final ObservableList {\bf  getStylesheets}()
}%end signature
}%end item
\item{\vskip -1.5ex 
\texttt{protected boolean {\bf  impl\_computeContains}(\texttt{double} {\bf  arg0},
\texttt{double} {\bf  arg1})
}%end signature
}%end item
\item{\vskip -1.5ex 
\texttt{public BaseBounds {\bf  impl\_computeGeomBounds}(\texttt{com.sun.javafx.geom.BaseBounds} {\bf  arg0},
\texttt{com.sun.javafx.geom.transform.BaseTransform} {\bf  arg1})
}%end signature
}%end item
\item{\vskip -1.5ex 
\texttt{protected NGNode {\bf  impl\_createPeer}()
}%end signature
}%end item
\item{\vskip -1.5ex 
\texttt{public List {\bf  impl\_getAllParentStylesheets}()
}%end signature
}%end item
\item{\vskip -1.5ex 
\texttt{protected void {\bf  impl\_pickNodeLocal}(\texttt{com.sun.javafx.geom.PickRay} {\bf  arg0},
\texttt{com.sun.javafx.scene.input.PickResultChooser} {\bf  arg1})
}%end signature
}%end item
\item{\vskip -1.5ex 
\texttt{protected void {\bf  impl\_processCSS}(\texttt{javafx.beans.value.WritableValue} {\bf  arg0})
}%end signature
}%end item
\item{\vskip -1.5ex 
\texttt{public Object {\bf  impl\_processMXNode}(\texttt{com.sun.javafx.jmx.MXNodeAlgorithm} {\bf  arg0},
\texttt{com.sun.javafx.jmx.MXNodeAlgorithmContext} {\bf  arg1})
}%end signature
}%end item
\item{\vskip -1.5ex 
\texttt{public final ObjectProperty {\bf  impl\_traversalEngineProperty}()
}%end signature
}%end item
\item{\vskip -1.5ex 
\texttt{public void {\bf  impl\_updatePeer}()
}%end signature
}%end item
\item{\vskip -1.5ex 
\texttt{public final boolean {\bf  isNeedsLayout}()
}%end signature
}%end item
\item{\vskip -1.5ex 
\texttt{public final void {\bf  layout}()
}%end signature
}%end item
\item{\vskip -1.5ex 
\texttt{protected void {\bf  layoutChildren}()
}%end signature
}%end item
\item{\vskip -1.5ex 
\texttt{public Node {\bf  lookup}(\texttt{java.lang.String} {\bf  arg0})
}%end signature
}%end item
\item{\vskip -1.5ex 
\texttt{public double {\bf  minHeight}(\texttt{double} {\bf  arg0})
}%end signature
}%end item
\item{\vskip -1.5ex 
\texttt{public double {\bf  minWidth}(\texttt{double} {\bf  arg0})
}%end signature
}%end item
\item{\vskip -1.5ex 
\texttt{public final ReadOnlyBooleanProperty {\bf  needsLayoutProperty}()
}%end signature
}%end item
\item{\vskip -1.5ex 
\texttt{public double {\bf  prefHeight}(\texttt{double} {\bf  arg0})
}%end signature
}%end item
\item{\vskip -1.5ex 
\texttt{public double {\bf  prefWidth}(\texttt{double} {\bf  arg0})
}%end signature
}%end item
\item{\vskip -1.5ex 
\texttt{public Object {\bf  queryAccessibleAttribute}(\texttt{AccessibleAttribute} {\bf  arg0},
\texttt{java.lang.Object\lbrack \rbrack } {\bf  arg1})
}%end signature
}%end item
\item{\vskip -1.5ex 
\texttt{public void {\bf  requestLayout}()
}%end signature
}%end item
\item{\vskip -1.5ex 
\texttt{protected final void {\bf  requestParentLayout}()
}%end signature
}%end item
\item{\vskip -1.5ex 
\texttt{public final void {\bf  setImpl\_traversalEngine}(\texttt{com.sun.javafx.scene.traversal.ParentTraversalEngine} {\bf  arg0})
}%end signature
}%end item
\item{\vskip -1.5ex 
\texttt{protected final void {\bf  setNeedsLayout}(\texttt{boolean} {\bf  arg0})
}%end signature
}%end item
\item{\vskip -1.5ex 
\texttt{protected void {\bf  updateBounds}()
}%end signature
}%end item
\end{itemize}
}
\subsection{Members inherited from class Node }{
\texttt{javafx.scene.Node} {\small 
\refdefined{javafx.scene.Node}}
{\small 

\vskip -2em
\begin{itemize}
\item{\vskip -1.5ex 
\texttt{public final ObjectProperty {\bf  accessibleHelpProperty}()
}%end signature
}%end item
\item{\vskip -1.5ex 
\texttt{public final ObjectProperty {\bf  accessibleRoleDescriptionProperty}()
}%end signature
}%end item
\item{\vskip -1.5ex 
\texttt{public final ObjectProperty {\bf  accessibleRoleProperty}()
}%end signature
}%end item
\item{\vskip -1.5ex 
\texttt{public final ObjectProperty {\bf  accessibleTextProperty}()
}%end signature
}%end item
\item{\vskip -1.5ex 
\texttt{public final void {\bf  addEventFilter}(\texttt{javafx.event.EventType} {\bf  arg0},
\texttt{javafx.event.EventHandler} {\bf  arg1})
}%end signature
}%end item
\item{\vskip -1.5ex 
\texttt{public final void {\bf  addEventHandler}(\texttt{javafx.event.EventType} {\bf  arg0},
\texttt{javafx.event.EventHandler} {\bf  arg1})
}%end signature
}%end item
\item{\vskip -1.5ex 
\texttt{public final void {\bf  applyCss}()
}%end signature
}%end item
\item{\vskip -1.5ex 
\texttt{public final void {\bf  autosize}()
}%end signature
}%end item
\item{\vskip -1.5ex 
\texttt{public static final {\bf  BASELINE\_OFFSET\_SAME\_AS\_HEIGHT}}%end signature
}%end item
\item{\vskip -1.5ex 
\texttt{public final ObjectProperty {\bf  blendModeProperty}()
}%end signature
}%end item
\item{\vskip -1.5ex 
\texttt{public final ReadOnlyObjectProperty {\bf  boundsInLocalProperty}()
}%end signature
}%end item
\item{\vskip -1.5ex 
\texttt{public final ReadOnlyObjectProperty {\bf  boundsInParentProperty}()
}%end signature
}%end item
\item{\vskip -1.5ex 
\texttt{public EventDispatchChain {\bf  buildEventDispatchChain}(\texttt{javafx.event.EventDispatchChain} {\bf  arg0})
}%end signature
}%end item
\item{\vskip -1.5ex 
\texttt{public final ObjectProperty {\bf  cacheHintProperty}()
}%end signature
}%end item
\item{\vskip -1.5ex 
\texttt{public final BooleanProperty {\bf  cacheProperty}()
}%end signature
}%end item
\item{\vskip -1.5ex 
\texttt{public final ObjectProperty {\bf  clipProperty}()
}%end signature
}%end item
\item{\vskip -1.5ex 
\texttt{public double {\bf  computeAreaInScreen}()
}%end signature
}%end item
\item{\vskip -1.5ex 
\texttt{public boolean {\bf  contains}(\texttt{double} {\bf  arg0},
\texttt{double} {\bf  arg1})
}%end signature
}%end item
\item{\vskip -1.5ex 
\texttt{public boolean {\bf  contains}(\texttt{javafx.geometry.Point2D} {\bf  arg0})
}%end signature
}%end item
\item{\vskip -1.5ex 
\texttt{protected boolean {\bf  containsBounds}(\texttt{double} {\bf  arg0},
\texttt{double} {\bf  arg1})
}%end signature
}%end item
\item{\vskip -1.5ex 
\texttt{public final ObjectProperty {\bf  cursorProperty}()
}%end signature
}%end item
\item{\vskip -1.5ex 
\texttt{public final ObjectProperty {\bf  depthTestProperty}()
}%end signature
}%end item
\item{\vskip -1.5ex 
\texttt{public final ReadOnlyBooleanProperty {\bf  disabledProperty}()
}%end signature
}%end item
\item{\vskip -1.5ex 
\texttt{public final BooleanProperty {\bf  disableProperty}()
}%end signature
}%end item
\item{\vskip -1.5ex 
\texttt{public final ReadOnlyObjectProperty {\bf  effectiveNodeOrientationProperty}()
}%end signature
}%end item
\item{\vskip -1.5ex 
\texttt{public final ObjectProperty {\bf  effectProperty}()
}%end signature
}%end item
\item{\vskip -1.5ex 
\texttt{public final ObjectProperty {\bf  eventDispatcherProperty}()
}%end signature
}%end item
\item{\vskip -1.5ex 
\texttt{public void {\bf  executeAccessibleAction}(\texttt{AccessibleAction} {\bf  arg0},
\texttt{java.lang.Object\lbrack \rbrack } {\bf  arg1})
}%end signature
}%end item
\item{\vskip -1.5ex 
\texttt{public final void {\bf  fireEvent}(\texttt{javafx.event.Event} {\bf  arg0})
}%end signature
}%end item
\item{\vskip -1.5ex 
\texttt{public final ReadOnlyBooleanProperty {\bf  focusedProperty}()
}%end signature
}%end item
\item{\vskip -1.5ex 
\texttt{public final BooleanProperty {\bf  focusTraversableProperty}()
}%end signature
}%end item
\item{\vskip -1.5ex 
\texttt{public final String {\bf  getAccessibleHelp}()
}%end signature
}%end item
\item{\vskip -1.5ex 
\texttt{public final AccessibleRole {\bf  getAccessibleRole}()
}%end signature
}%end item
\item{\vskip -1.5ex 
\texttt{public final String {\bf  getAccessibleRoleDescription}()
}%end signature
}%end item
\item{\vskip -1.5ex 
\texttt{public final String {\bf  getAccessibleText}()
}%end signature
}%end item
\item{\vskip -1.5ex 
\texttt{public double {\bf  getBaselineOffset}()
}%end signature
}%end item
\item{\vskip -1.5ex 
\texttt{public final BlendMode {\bf  getBlendMode}()
}%end signature
}%end item
\item{\vskip -1.5ex 
\texttt{public final Bounds {\bf  getBoundsInLocal}()
}%end signature
}%end item
\item{\vskip -1.5ex 
\texttt{public final Bounds {\bf  getBoundsInParent}()
}%end signature
}%end item
\item{\vskip -1.5ex 
\texttt{public final CacheHint {\bf  getCacheHint}()
}%end signature
}%end item
\item{\vskip -1.5ex 
\texttt{public static List {\bf  getClassCssMetaData}()
}%end signature
}%end item
\item{\vskip -1.5ex 
\texttt{public final Node {\bf  getClip}()
}%end signature
}%end item
\item{\vskip -1.5ex 
\texttt{public Orientation {\bf  getContentBias}()
}%end signature
}%end item
\item{\vskip -1.5ex 
\texttt{public List {\bf  getCssMetaData}()
}%end signature
}%end item
\item{\vskip -1.5ex 
\texttt{public final Cursor {\bf  getCursor}()
}%end signature
}%end item
\item{\vskip -1.5ex 
\texttt{public final DepthTest {\bf  getDepthTest}()
}%end signature
}%end item
\item{\vskip -1.5ex 
\texttt{public final Effect {\bf  getEffect}()
}%end signature
}%end item
\item{\vskip -1.5ex 
\texttt{public final NodeOrientation {\bf  getEffectiveNodeOrientation}()
}%end signature
}%end item
\item{\vskip -1.5ex 
\texttt{public final EventDispatcher {\bf  getEventDispatcher}()
}%end signature
}%end item
\item{\vskip -1.5ex 
\texttt{public final String {\bf  getId}()
}%end signature
}%end item
\item{\vskip -1.5ex 
\texttt{public final InputMethodRequests {\bf  getInputMethodRequests}()
}%end signature
}%end item
\item{\vskip -1.5ex 
\texttt{public final Bounds {\bf  getLayoutBounds}()
}%end signature
}%end item
\item{\vskip -1.5ex 
\texttt{public final double {\bf  getLayoutX}()
}%end signature
}%end item
\item{\vskip -1.5ex 
\texttt{public final double {\bf  getLayoutY}()
}%end signature
}%end item
\item{\vskip -1.5ex 
\texttt{public final Transform {\bf  getLocalToParentTransform}()
}%end signature
}%end item
\item{\vskip -1.5ex 
\texttt{public final Transform {\bf  getLocalToSceneTransform}()
}%end signature
}%end item
\item{\vskip -1.5ex 
\texttt{public final NodeOrientation {\bf  getNodeOrientation}()
}%end signature
}%end item
\item{\vskip -1.5ex 
\texttt{public final EventHandler {\bf  getOnContextMenuRequested}()
}%end signature
}%end item
\item{\vskip -1.5ex 
\texttt{public final EventHandler {\bf  getOnDragDetected}()
}%end signature
}%end item
\item{\vskip -1.5ex 
\texttt{public final EventHandler {\bf  getOnDragDone}()
}%end signature
}%end item
\item{\vskip -1.5ex 
\texttt{public final EventHandler {\bf  getOnDragDropped}()
}%end signature
}%end item
\item{\vskip -1.5ex 
\texttt{public final EventHandler {\bf  getOnDragEntered}()
}%end signature
}%end item
\item{\vskip -1.5ex 
\texttt{public final EventHandler {\bf  getOnDragExited}()
}%end signature
}%end item
\item{\vskip -1.5ex 
\texttt{public final EventHandler {\bf  getOnDragOver}()
}%end signature
}%end item
\item{\vskip -1.5ex 
\texttt{public final EventHandler {\bf  getOnInputMethodTextChanged}()
}%end signature
}%end item
\item{\vskip -1.5ex 
\texttt{public final EventHandler {\bf  getOnKeyPressed}()
}%end signature
}%end item
\item{\vskip -1.5ex 
\texttt{public final EventHandler {\bf  getOnKeyReleased}()
}%end signature
}%end item
\item{\vskip -1.5ex 
\texttt{public final EventHandler {\bf  getOnKeyTyped}()
}%end signature
}%end item
\item{\vskip -1.5ex 
\texttt{public final EventHandler {\bf  getOnMouseClicked}()
}%end signature
}%end item
\item{\vskip -1.5ex 
\texttt{public final EventHandler {\bf  getOnMouseDragEntered}()
}%end signature
}%end item
\item{\vskip -1.5ex 
\texttt{public final EventHandler {\bf  getOnMouseDragExited}()
}%end signature
}%end item
\item{\vskip -1.5ex 
\texttt{public final EventHandler {\bf  getOnMouseDragged}()
}%end signature
}%end item
\item{\vskip -1.5ex 
\texttt{public final EventHandler {\bf  getOnMouseDragOver}()
}%end signature
}%end item
\item{\vskip -1.5ex 
\texttt{public final EventHandler {\bf  getOnMouseDragReleased}()
}%end signature
}%end item
\item{\vskip -1.5ex 
\texttt{public final EventHandler {\bf  getOnMouseEntered}()
}%end signature
}%end item
\item{\vskip -1.5ex 
\texttt{public final EventHandler {\bf  getOnMouseExited}()
}%end signature
}%end item
\item{\vskip -1.5ex 
\texttt{public final EventHandler {\bf  getOnMouseMoved}()
}%end signature
}%end item
\item{\vskip -1.5ex 
\texttt{public final EventHandler {\bf  getOnMousePressed}()
}%end signature
}%end item
\item{\vskip -1.5ex 
\texttt{public final EventHandler {\bf  getOnMouseReleased}()
}%end signature
}%end item
\item{\vskip -1.5ex 
\texttt{public final EventHandler {\bf  getOnRotate}()
}%end signature
}%end item
\item{\vskip -1.5ex 
\texttt{public final EventHandler {\bf  getOnRotationFinished}()
}%end signature
}%end item
\item{\vskip -1.5ex 
\texttt{public final EventHandler {\bf  getOnRotationStarted}()
}%end signature
}%end item
\item{\vskip -1.5ex 
\texttt{public final EventHandler {\bf  getOnScroll}()
}%end signature
}%end item
\item{\vskip -1.5ex 
\texttt{public final EventHandler {\bf  getOnScrollFinished}()
}%end signature
}%end item
\item{\vskip -1.5ex 
\texttt{public final EventHandler {\bf  getOnScrollStarted}()
}%end signature
}%end item
\item{\vskip -1.5ex 
\texttt{public final EventHandler {\bf  getOnSwipeDown}()
}%end signature
}%end item
\item{\vskip -1.5ex 
\texttt{public final EventHandler {\bf  getOnSwipeLeft}()
}%end signature
}%end item
\item{\vskip -1.5ex 
\texttt{public final EventHandler {\bf  getOnSwipeRight}()
}%end signature
}%end item
\item{\vskip -1.5ex 
\texttt{public final EventHandler {\bf  getOnSwipeUp}()
}%end signature
}%end item
\item{\vskip -1.5ex 
\texttt{public final EventHandler {\bf  getOnTouchMoved}()
}%end signature
}%end item
\item{\vskip -1.5ex 
\texttt{public final EventHandler {\bf  getOnTouchPressed}()
}%end signature
}%end item
\item{\vskip -1.5ex 
\texttt{public final EventHandler {\bf  getOnTouchReleased}()
}%end signature
}%end item
\item{\vskip -1.5ex 
\texttt{public final EventHandler {\bf  getOnTouchStationary}()
}%end signature
}%end item
\item{\vskip -1.5ex 
\texttt{public final EventHandler {\bf  getOnZoom}()
}%end signature
}%end item
\item{\vskip -1.5ex 
\texttt{public final EventHandler {\bf  getOnZoomFinished}()
}%end signature
}%end item
\item{\vskip -1.5ex 
\texttt{public final EventHandler {\bf  getOnZoomStarted}()
}%end signature
}%end item
\item{\vskip -1.5ex 
\texttt{public final double {\bf  getOpacity}()
}%end signature
}%end item
\item{\vskip -1.5ex 
\texttt{public final Parent {\bf  getParent}()
}%end signature
}%end item
\item{\vskip -1.5ex 
\texttt{public final ObservableMap {\bf  getProperties}()
}%end signature
}%end item
\item{\vskip -1.5ex 
\texttt{public final ObservableSet {\bf  getPseudoClassStates}()
}%end signature
}%end item
\item{\vskip -1.5ex 
\texttt{public final double {\bf  getRotate}()
}%end signature
}%end item
\item{\vskip -1.5ex 
\texttt{public final Point3D {\bf  getRotationAxis}()
}%end signature
}%end item
\item{\vskip -1.5ex 
\texttt{public final double {\bf  getScaleX}()
}%end signature
}%end item
\item{\vskip -1.5ex 
\texttt{public final double {\bf  getScaleY}()
}%end signature
}%end item
\item{\vskip -1.5ex 
\texttt{public final double {\bf  getScaleZ}()
}%end signature
}%end item
\item{\vskip -1.5ex 
\texttt{public final Scene {\bf  getScene}()
}%end signature
}%end item
\item{\vskip -1.5ex 
\texttt{public final String {\bf  getStyle}()
}%end signature
}%end item
\item{\vskip -1.5ex 
\texttt{public Styleable {\bf  getStyleableParent}()
}%end signature
}%end item
\item{\vskip -1.5ex 
\texttt{public final ObservableList {\bf  getStyleClass}()
}%end signature
}%end item
\item{\vskip -1.5ex 
\texttt{public final ObservableList {\bf  getTransforms}()
}%end signature
}%end item
\item{\vskip -1.5ex 
\texttt{public final double {\bf  getTranslateX}()
}%end signature
}%end item
\item{\vskip -1.5ex 
\texttt{public final double {\bf  getTranslateY}()
}%end signature
}%end item
\item{\vskip -1.5ex 
\texttt{public final double {\bf  getTranslateZ}()
}%end signature
}%end item
\item{\vskip -1.5ex 
\texttt{public String {\bf  getTypeSelector}()
}%end signature
}%end item
\item{\vskip -1.5ex 
\texttt{public Object {\bf  getUserData}()
}%end signature
}%end item
\item{\vskip -1.5ex 
\texttt{public boolean {\bf  hasProperties}()
}%end signature
}%end item
\item{\vskip -1.5ex 
\texttt{public final ReadOnlyBooleanProperty {\bf  hoverProperty}()
}%end signature
}%end item
\item{\vskip -1.5ex 
\texttt{public final StringProperty {\bf  idProperty}()
}%end signature
}%end item
\item{\vskip -1.5ex 
\texttt{protected final void {\bf  impl\_clearDirty}(\texttt{com.sun.javafx.scene.DirtyBits} {\bf  arg0})
}%end signature
}%end item
\item{\vskip -1.5ex 
\texttt{protected abstract boolean {\bf  impl\_computeContains}(\texttt{double} {\bf  arg0},
\texttt{double} {\bf  arg1})
}%end signature
}%end item
\item{\vskip -1.5ex 
\texttt{public abstract BaseBounds {\bf  impl\_computeGeomBounds}(\texttt{com.sun.javafx.geom.BaseBounds} {\bf  arg0},
\texttt{com.sun.javafx.geom.transform.BaseTransform} {\bf  arg1})
}%end signature
}%end item
\item{\vskip -1.5ex 
\texttt{protected boolean {\bf  impl\_computeIntersects}(\texttt{com.sun.javafx.geom.PickRay} {\bf  arg0},
\texttt{com.sun.javafx.scene.input.PickResultChooser} {\bf  arg1})
}%end signature
}%end item
\item{\vskip -1.5ex 
\texttt{protected Bounds {\bf  impl\_computeLayoutBounds}()
}%end signature
}%end item
\item{\vskip -1.5ex 
\texttt{protected abstract NGNode {\bf  impl\_createPeer}()
}%end signature
}%end item
\item{\vskip -1.5ex 
\texttt{protected Cursor {\bf  impl\_cssGetCursorInitialValue}()
}%end signature
}%end item
\item{\vskip -1.5ex 
\texttt{protected Boolean {\bf  impl\_cssGetFocusTraversableInitialValue}()
}%end signature
}%end item
\item{\vskip -1.5ex 
\texttt{public Map {\bf  impl\_findStyles}(\texttt{java.util.Map} {\bf  arg0})
}%end signature
}%end item
\item{\vskip -1.5ex 
\texttt{protected void {\bf  impl\_geomChanged}()
}%end signature
}%end item
\item{\vskip -1.5ex 
\texttt{public final BaseTransform {\bf  impl\_getLeafTransform}()
}%end signature
}%end item
\item{\vskip -1.5ex 
\texttt{public static List {\bf  impl\_getMatchingStyles}(\texttt{javafx.css.CssMetaData} {\bf  arg0},
\texttt{javafx.css.Styleable} {\bf  arg1})
}%end signature
}%end item
\item{\vskip -1.5ex 
\texttt{public NGNode {\bf  impl\_getPeer}()
}%end signature
}%end item
\item{\vskip -1.5ex 
\texttt{public final double {\bf  impl\_getPivotX}()
}%end signature
}%end item
\item{\vskip -1.5ex 
\texttt{public final double {\bf  impl\_getPivotY}()
}%end signature
}%end item
\item{\vskip -1.5ex 
\texttt{public final double {\bf  impl\_getPivotZ}()
}%end signature
}%end item
\item{\vskip -1.5ex 
\texttt{public final ObservableMap {\bf  impl\_getStyleMap}()
}%end signature
}%end item
\item{\vskip -1.5ex 
\texttt{public boolean {\bf  impl\_hasTransforms}()
}%end signature
}%end item
\item{\vskip -1.5ex 
\texttt{protected final boolean {\bf  impl\_intersects}(\texttt{com.sun.javafx.geom.PickRay} {\bf  arg0},
\texttt{com.sun.javafx.scene.input.PickResultChooser} {\bf  arg1})
}%end signature
}%end item
\item{\vskip -1.5ex 
\texttt{protected final double {\bf  impl\_intersectsBounds}(\texttt{com.sun.javafx.geom.PickRay} {\bf  arg0})
}%end signature
}%end item
\item{\vskip -1.5ex 
\texttt{protected final boolean {\bf  impl\_isDirty}(\texttt{com.sun.javafx.scene.DirtyBits} {\bf  arg0})
}%end signature
}%end item
\item{\vskip -1.5ex 
\texttt{protected final boolean {\bf  impl\_isDirtyEmpty}()
}%end signature
}%end item
\item{\vskip -1.5ex 
\texttt{public final boolean {\bf  impl\_isShowMnemonics}()
}%end signature
}%end item
\item{\vskip -1.5ex 
\texttt{public final boolean {\bf  impl\_isTreeVisible}()
}%end signature
}%end item
\item{\vskip -1.5ex 
\texttt{protected final void {\bf  impl\_layoutBoundsChanged}()
}%end signature
}%end item
\item{\vskip -1.5ex 
\texttt{protected void {\bf  impl\_markDirty}(\texttt{com.sun.javafx.scene.DirtyBits} {\bf  arg0})
}%end signature
}%end item
\item{\vskip -1.5ex 
\texttt{protected void {\bf  impl\_notifyLayoutBoundsChanged}()
}%end signature
}%end item
\item{\vskip -1.5ex 
\texttt{public final void {\bf  impl\_pickNode}(\texttt{com.sun.javafx.geom.PickRay} {\bf  arg0},
\texttt{com.sun.javafx.scene.input.PickResultChooser} {\bf  arg1})
}%end signature
}%end item
\item{\vskip -1.5ex 
\texttt{protected void {\bf  impl\_pickNodeLocal}(\texttt{com.sun.javafx.geom.PickRay} {\bf  arg0},
\texttt{com.sun.javafx.scene.input.PickResultChooser} {\bf  arg1})
}%end signature
}%end item
\item{\vskip -1.5ex 
\texttt{public final void {\bf  impl\_processCSS}(\texttt{boolean} {\bf  arg0})
}%end signature
}%end item
\item{\vskip -1.5ex 
\texttt{protected void {\bf  impl\_processCSS}(\texttt{javafx.beans.value.WritableValue} {\bf  arg0})
}%end signature
}%end item
\item{\vskip -1.5ex 
\texttt{public abstract Object {\bf  impl\_processMXNode}(\texttt{com.sun.javafx.jmx.MXNodeAlgorithm} {\bf  arg0},
\texttt{com.sun.javafx.jmx.MXNodeAlgorithmContext} {\bf  arg1})
}%end signature
}%end item
\item{\vskip -1.5ex 
\texttt{public final void {\bf  impl\_reapplyCSS}()
}%end signature
}%end item
\item{\vskip -1.5ex 
\texttt{public final void {\bf  impl\_setShowMnemonics}(\texttt{boolean} {\bf  arg0})
}%end signature
}%end item
\item{\vskip -1.5ex 
\texttt{public final void {\bf  impl\_setStyleMap}(\texttt{javafx.collections.ObservableMap} {\bf  arg0})
}%end signature
}%end item
\item{\vskip -1.5ex 
\texttt{public final BooleanProperty {\bf  impl\_showMnemonicsProperty}()
}%end signature
}%end item
\item{\vskip -1.5ex 
\texttt{public final void {\bf  impl\_syncPeer}()
}%end signature
}%end item
\item{\vskip -1.5ex 
\texttt{public void {\bf  impl\_transformsChanged}()
}%end signature
}%end item
\item{\vskip -1.5ex 
\texttt{public final boolean {\bf  impl\_traverse}(\texttt{com.sun.javafx.scene.traversal.Direction} {\bf  arg0})
}%end signature
}%end item
\item{\vskip -1.5ex 
\texttt{protected final BooleanExpression {\bf  impl\_treeVisibleProperty}()
}%end signature
}%end item
\item{\vskip -1.5ex 
\texttt{public void {\bf  impl\_updatePeer}()
}%end signature
}%end item
\item{\vskip -1.5ex 
\texttt{public final ObjectProperty {\bf  inputMethodRequestsProperty}()
}%end signature
}%end item
\item{\vskip -1.5ex 
\texttt{public boolean {\bf  intersects}(\texttt{javafx.geometry.Bounds} {\bf  arg0})
}%end signature
}%end item
\item{\vskip -1.5ex 
\texttt{public boolean {\bf  intersects}(\texttt{double} {\bf  arg0},
\texttt{double} {\bf  arg1},
\texttt{double} {\bf  arg2},
\texttt{double} {\bf  arg3})
}%end signature
}%end item
\item{\vskip -1.5ex 
\texttt{public final boolean {\bf  isCache}()
}%end signature
}%end item
\item{\vskip -1.5ex 
\texttt{public final boolean {\bf  isDisable}()
}%end signature
}%end item
\item{\vskip -1.5ex 
\texttt{public final boolean {\bf  isDisabled}()
}%end signature
}%end item
\item{\vskip -1.5ex 
\texttt{public final boolean {\bf  isFocused}()
}%end signature
}%end item
\item{\vskip -1.5ex 
\texttt{public final boolean {\bf  isFocusTraversable}()
}%end signature
}%end item
\item{\vskip -1.5ex 
\texttt{public final boolean {\bf  isHover}()
}%end signature
}%end item
\item{\vskip -1.5ex 
\texttt{public final boolean {\bf  isManaged}()
}%end signature
}%end item
\item{\vskip -1.5ex 
\texttt{public final boolean {\bf  isMouseTransparent}()
}%end signature
}%end item
\item{\vskip -1.5ex 
\texttt{public final boolean {\bf  isPickOnBounds}()
}%end signature
}%end item
\item{\vskip -1.5ex 
\texttt{public final boolean {\bf  isPressed}()
}%end signature
}%end item
\item{\vskip -1.5ex 
\texttt{public boolean {\bf  isResizable}()
}%end signature
}%end item
\item{\vskip -1.5ex 
\texttt{public final boolean {\bf  isVisible}()
}%end signature
}%end item
\item{\vskip -1.5ex 
\texttt{public final ReadOnlyObjectProperty {\bf  layoutBoundsProperty}()
}%end signature
}%end item
\item{\vskip -1.5ex 
\texttt{public final DoubleProperty {\bf  layoutXProperty}()
}%end signature
}%end item
\item{\vskip -1.5ex 
\texttt{public final DoubleProperty {\bf  layoutYProperty}()
}%end signature
}%end item
\item{\vskip -1.5ex 
\texttt{public Bounds {\bf  localToParent}(\texttt{javafx.geometry.Bounds} {\bf  arg0})
}%end signature
}%end item
\item{\vskip -1.5ex 
\texttt{public Point2D {\bf  localToParent}(\texttt{double} {\bf  arg0},
\texttt{double} {\bf  arg1})
}%end signature
}%end item
\item{\vskip -1.5ex 
\texttt{public Point3D {\bf  localToParent}(\texttt{double} {\bf  arg0},
\texttt{double} {\bf  arg1},
\texttt{double} {\bf  arg2})
}%end signature
}%end item
\item{\vskip -1.5ex 
\texttt{public Point2D {\bf  localToParent}(\texttt{javafx.geometry.Point2D} {\bf  arg0})
}%end signature
}%end item
\item{\vskip -1.5ex 
\texttt{public Point3D {\bf  localToParent}(\texttt{javafx.geometry.Point3D} {\bf  arg0})
}%end signature
}%end item
\item{\vskip -1.5ex 
\texttt{public final ReadOnlyObjectProperty {\bf  localToParentTransformProperty}()
}%end signature
}%end item
\item{\vskip -1.5ex 
\texttt{public Bounds {\bf  localToScene}(\texttt{javafx.geometry.Bounds} {\bf  arg0})
}%end signature
}%end item
\item{\vskip -1.5ex 
\texttt{public Bounds {\bf  localToScene}(\texttt{javafx.geometry.Bounds} {\bf  arg0},
\texttt{boolean} {\bf  arg1})
}%end signature
}%end item
\item{\vskip -1.5ex 
\texttt{public Point2D {\bf  localToScene}(\texttt{double} {\bf  arg0},
\texttt{double} {\bf  arg1})
}%end signature
}%end item
\item{\vskip -1.5ex 
\texttt{public Point2D {\bf  localToScene}(\texttt{double} {\bf  arg0},
\texttt{double} {\bf  arg1},
\texttt{boolean} {\bf  arg2})
}%end signature
}%end item
\item{\vskip -1.5ex 
\texttt{public Point3D {\bf  localToScene}(\texttt{double} {\bf  arg0},
\texttt{double} {\bf  arg1},
\texttt{double} {\bf  arg2})
}%end signature
}%end item
\item{\vskip -1.5ex 
\texttt{public Point3D {\bf  localToScene}(\texttt{double} {\bf  arg0},
\texttt{double} {\bf  arg1},
\texttt{double} {\bf  arg2},
\texttt{boolean} {\bf  arg3})
}%end signature
}%end item
\item{\vskip -1.5ex 
\texttt{public Point2D {\bf  localToScene}(\texttt{javafx.geometry.Point2D} {\bf  arg0})
}%end signature
}%end item
\item{\vskip -1.5ex 
\texttt{public Point2D {\bf  localToScene}(\texttt{javafx.geometry.Point2D} {\bf  arg0},
\texttt{boolean} {\bf  arg1})
}%end signature
}%end item
\item{\vskip -1.5ex 
\texttt{public Point3D {\bf  localToScene}(\texttt{javafx.geometry.Point3D} {\bf  arg0})
}%end signature
}%end item
\item{\vskip -1.5ex 
\texttt{public Point3D {\bf  localToScene}(\texttt{javafx.geometry.Point3D} {\bf  arg0},
\texttt{boolean} {\bf  arg1})
}%end signature
}%end item
\item{\vskip -1.5ex 
\texttt{public final ReadOnlyObjectProperty {\bf  localToSceneTransformProperty}()
}%end signature
}%end item
\item{\vskip -1.5ex 
\texttt{public Bounds {\bf  localToScreen}(\texttt{javafx.geometry.Bounds} {\bf  arg0})
}%end signature
}%end item
\item{\vskip -1.5ex 
\texttt{public Point2D {\bf  localToScreen}(\texttt{double} {\bf  arg0},
\texttt{double} {\bf  arg1})
}%end signature
}%end item
\item{\vskip -1.5ex 
\texttt{public Point2D {\bf  localToScreen}(\texttt{double} {\bf  arg0},
\texttt{double} {\bf  arg1},
\texttt{double} {\bf  arg2})
}%end signature
}%end item
\item{\vskip -1.5ex 
\texttt{public Point2D {\bf  localToScreen}(\texttt{javafx.geometry.Point2D} {\bf  arg0})
}%end signature
}%end item
\item{\vskip -1.5ex 
\texttt{public Point2D {\bf  localToScreen}(\texttt{javafx.geometry.Point3D} {\bf  arg0})
}%end signature
}%end item
\item{\vskip -1.5ex 
\texttt{public Node {\bf  lookup}(\texttt{java.lang.String} {\bf  arg0})
}%end signature
}%end item
\item{\vskip -1.5ex 
\texttt{public Set {\bf  lookupAll}(\texttt{java.lang.String} {\bf  arg0})
}%end signature
}%end item
\item{\vskip -1.5ex 
\texttt{public final BooleanProperty {\bf  managedProperty}()
}%end signature
}%end item
\item{\vskip -1.5ex 
\texttt{public double {\bf  maxHeight}(\texttt{double} {\bf  arg0})
}%end signature
}%end item
\item{\vskip -1.5ex 
\texttt{public double {\bf  maxWidth}(\texttt{double} {\bf  arg0})
}%end signature
}%end item
\item{\vskip -1.5ex 
\texttt{public double {\bf  minHeight}(\texttt{double} {\bf  arg0})
}%end signature
}%end item
\item{\vskip -1.5ex 
\texttt{public double {\bf  minWidth}(\texttt{double} {\bf  arg0})
}%end signature
}%end item
\item{\vskip -1.5ex 
\texttt{public final BooleanProperty {\bf  mouseTransparentProperty}()
}%end signature
}%end item
\item{\vskip -1.5ex 
\texttt{public final ObjectProperty {\bf  nodeOrientationProperty}()
}%end signature
}%end item
\item{\vskip -1.5ex 
\texttt{public final void {\bf  notifyAccessibleAttributeChanged}(\texttt{AccessibleAttribute} {\bf  arg0})
}%end signature
}%end item
\item{\vskip -1.5ex 
\texttt{public final ObjectProperty {\bf  onContextMenuRequestedProperty}()
}%end signature
}%end item
\item{\vskip -1.5ex 
\texttt{public final ObjectProperty {\bf  onDragDetectedProperty}()
}%end signature
}%end item
\item{\vskip -1.5ex 
\texttt{public final ObjectProperty {\bf  onDragDoneProperty}()
}%end signature
}%end item
\item{\vskip -1.5ex 
\texttt{public final ObjectProperty {\bf  onDragDroppedProperty}()
}%end signature
}%end item
\item{\vskip -1.5ex 
\texttt{public final ObjectProperty {\bf  onDragEnteredProperty}()
}%end signature
}%end item
\item{\vskip -1.5ex 
\texttt{public final ObjectProperty {\bf  onDragExitedProperty}()
}%end signature
}%end item
\item{\vskip -1.5ex 
\texttt{public final ObjectProperty {\bf  onDragOverProperty}()
}%end signature
}%end item
\item{\vskip -1.5ex 
\texttt{public final ObjectProperty {\bf  onInputMethodTextChangedProperty}()
}%end signature
}%end item
\item{\vskip -1.5ex 
\texttt{public final ObjectProperty {\bf  onKeyPressedProperty}()
}%end signature
}%end item
\item{\vskip -1.5ex 
\texttt{public final ObjectProperty {\bf  onKeyReleasedProperty}()
}%end signature
}%end item
\item{\vskip -1.5ex 
\texttt{public final ObjectProperty {\bf  onKeyTypedProperty}()
}%end signature
}%end item
\item{\vskip -1.5ex 
\texttt{public final ObjectProperty {\bf  onMouseClickedProperty}()
}%end signature
}%end item
\item{\vskip -1.5ex 
\texttt{public final ObjectProperty {\bf  onMouseDragEnteredProperty}()
}%end signature
}%end item
\item{\vskip -1.5ex 
\texttt{public final ObjectProperty {\bf  onMouseDragExitedProperty}()
}%end signature
}%end item
\item{\vskip -1.5ex 
\texttt{public final ObjectProperty {\bf  onMouseDraggedProperty}()
}%end signature
}%end item
\item{\vskip -1.5ex 
\texttt{public final ObjectProperty {\bf  onMouseDragOverProperty}()
}%end signature
}%end item
\item{\vskip -1.5ex 
\texttt{public final ObjectProperty {\bf  onMouseDragReleasedProperty}()
}%end signature
}%end item
\item{\vskip -1.5ex 
\texttt{public final ObjectProperty {\bf  onMouseEnteredProperty}()
}%end signature
}%end item
\item{\vskip -1.5ex 
\texttt{public final ObjectProperty {\bf  onMouseExitedProperty}()
}%end signature
}%end item
\item{\vskip -1.5ex 
\texttt{public final ObjectProperty {\bf  onMouseMovedProperty}()
}%end signature
}%end item
\item{\vskip -1.5ex 
\texttt{public final ObjectProperty {\bf  onMousePressedProperty}()
}%end signature
}%end item
\item{\vskip -1.5ex 
\texttt{public final ObjectProperty {\bf  onMouseReleasedProperty}()
}%end signature
}%end item
\item{\vskip -1.5ex 
\texttt{public final ObjectProperty {\bf  onRotateProperty}()
}%end signature
}%end item
\item{\vskip -1.5ex 
\texttt{public final ObjectProperty {\bf  onRotationFinishedProperty}()
}%end signature
}%end item
\item{\vskip -1.5ex 
\texttt{public final ObjectProperty {\bf  onRotationStartedProperty}()
}%end signature
}%end item
\item{\vskip -1.5ex 
\texttt{public final ObjectProperty {\bf  onScrollFinishedProperty}()
}%end signature
}%end item
\item{\vskip -1.5ex 
\texttt{public final ObjectProperty {\bf  onScrollProperty}()
}%end signature
}%end item
\item{\vskip -1.5ex 
\texttt{public final ObjectProperty {\bf  onScrollStartedProperty}()
}%end signature
}%end item
\item{\vskip -1.5ex 
\texttt{public final ObjectProperty {\bf  onSwipeDownProperty}()
}%end signature
}%end item
\item{\vskip -1.5ex 
\texttt{public final ObjectProperty {\bf  onSwipeLeftProperty}()
}%end signature
}%end item
\item{\vskip -1.5ex 
\texttt{public final ObjectProperty {\bf  onSwipeRightProperty}()
}%end signature
}%end item
\item{\vskip -1.5ex 
\texttt{public final ObjectProperty {\bf  onSwipeUpProperty}()
}%end signature
}%end item
\item{\vskip -1.5ex 
\texttt{public final ObjectProperty {\bf  onTouchMovedProperty}()
}%end signature
}%end item
\item{\vskip -1.5ex 
\texttt{public final ObjectProperty {\bf  onTouchPressedProperty}()
}%end signature
}%end item
\item{\vskip -1.5ex 
\texttt{public final ObjectProperty {\bf  onTouchReleasedProperty}()
}%end signature
}%end item
\item{\vskip -1.5ex 
\texttt{public final ObjectProperty {\bf  onTouchStationaryProperty}()
}%end signature
}%end item
\item{\vskip -1.5ex 
\texttt{public final ObjectProperty {\bf  onZoomFinishedProperty}()
}%end signature
}%end item
\item{\vskip -1.5ex 
\texttt{public final ObjectProperty {\bf  onZoomProperty}()
}%end signature
}%end item
\item{\vskip -1.5ex 
\texttt{public final ObjectProperty {\bf  onZoomStartedProperty}()
}%end signature
}%end item
\item{\vskip -1.5ex 
\texttt{public final DoubleProperty {\bf  opacityProperty}()
}%end signature
}%end item
\item{\vskip -1.5ex 
\texttt{public final ReadOnlyObjectProperty {\bf  parentProperty}()
}%end signature
}%end item
\item{\vskip -1.5ex 
\texttt{public Bounds {\bf  parentToLocal}(\texttt{javafx.geometry.Bounds} {\bf  arg0})
}%end signature
}%end item
\item{\vskip -1.5ex 
\texttt{public Point2D {\bf  parentToLocal}(\texttt{double} {\bf  arg0},
\texttt{double} {\bf  arg1})
}%end signature
}%end item
\item{\vskip -1.5ex 
\texttt{public Point3D {\bf  parentToLocal}(\texttt{double} {\bf  arg0},
\texttt{double} {\bf  arg1},
\texttt{double} {\bf  arg2})
}%end signature
}%end item
\item{\vskip -1.5ex 
\texttt{public Point2D {\bf  parentToLocal}(\texttt{javafx.geometry.Point2D} {\bf  arg0})
}%end signature
}%end item
\item{\vskip -1.5ex 
\texttt{public Point3D {\bf  parentToLocal}(\texttt{javafx.geometry.Point3D} {\bf  arg0})
}%end signature
}%end item
\item{\vskip -1.5ex 
\texttt{public final BooleanProperty {\bf  pickOnBoundsProperty}()
}%end signature
}%end item
\item{\vskip -1.5ex 
\texttt{public double {\bf  prefHeight}(\texttt{double} {\bf  arg0})
}%end signature
}%end item
\item{\vskip -1.5ex 
\texttt{public double {\bf  prefWidth}(\texttt{double} {\bf  arg0})
}%end signature
}%end item
\item{\vskip -1.5ex 
\texttt{public final ReadOnlyBooleanProperty {\bf  pressedProperty}()
}%end signature
}%end item
\item{\vskip -1.5ex 
\texttt{public final void {\bf  pseudoClassStateChanged}(\texttt{javafx.css.PseudoClass} {\bf  arg0},
\texttt{boolean} {\bf  arg1})
}%end signature
}%end item
\item{\vskip -1.5ex 
\texttt{public Object {\bf  queryAccessibleAttribute}(\texttt{AccessibleAttribute} {\bf  arg0},
\texttt{java.lang.Object\lbrack \rbrack } {\bf  arg1})
}%end signature
}%end item
\item{\vskip -1.5ex 
\texttt{public void {\bf  relocate}(\texttt{double} {\bf  arg0},
\texttt{double} {\bf  arg1})
}%end signature
}%end item
\item{\vskip -1.5ex 
\texttt{public final void {\bf  removeEventFilter}(\texttt{javafx.event.EventType} {\bf  arg0},
\texttt{javafx.event.EventHandler} {\bf  arg1})
}%end signature
}%end item
\item{\vskip -1.5ex 
\texttt{public final void {\bf  removeEventHandler}(\texttt{javafx.event.EventType} {\bf  arg0},
\texttt{javafx.event.EventHandler} {\bf  arg1})
}%end signature
}%end item
\item{\vskip -1.5ex 
\texttt{public void {\bf  requestFocus}()
}%end signature
}%end item
\item{\vskip -1.5ex 
\texttt{public void {\bf  resize}(\texttt{double} {\bf  arg0},
\texttt{double} {\bf  arg1})
}%end signature
}%end item
\item{\vskip -1.5ex 
\texttt{public void {\bf  resizeRelocate}(\texttt{double} {\bf  arg0},
\texttt{double} {\bf  arg1},
\texttt{double} {\bf  arg2},
\texttt{double} {\bf  arg3})
}%end signature
}%end item
\item{\vskip -1.5ex 
\texttt{public final DoubleProperty {\bf  rotateProperty}()
}%end signature
}%end item
\item{\vskip -1.5ex 
\texttt{public final ObjectProperty {\bf  rotationAxisProperty}()
}%end signature
}%end item
\item{\vskip -1.5ex 
\texttt{public final DoubleProperty {\bf  scaleXProperty}()
}%end signature
}%end item
\item{\vskip -1.5ex 
\texttt{public final DoubleProperty {\bf  scaleYProperty}()
}%end signature
}%end item
\item{\vskip -1.5ex 
\texttt{public final DoubleProperty {\bf  scaleZProperty}()
}%end signature
}%end item
\item{\vskip -1.5ex 
\texttt{public final ReadOnlyObjectProperty {\bf  sceneProperty}()
}%end signature
}%end item
\item{\vskip -1.5ex 
\texttt{public Bounds {\bf  sceneToLocal}(\texttt{javafx.geometry.Bounds} {\bf  arg0})
}%end signature
}%end item
\item{\vskip -1.5ex 
\texttt{public Bounds {\bf  sceneToLocal}(\texttt{javafx.geometry.Bounds} {\bf  arg0},
\texttt{boolean} {\bf  arg1})
}%end signature
}%end item
\item{\vskip -1.5ex 
\texttt{public Point2D {\bf  sceneToLocal}(\texttt{double} {\bf  arg0},
\texttt{double} {\bf  arg1})
}%end signature
}%end item
\item{\vskip -1.5ex 
\texttt{public Point2D {\bf  sceneToLocal}(\texttt{double} {\bf  arg0},
\texttt{double} {\bf  arg1},
\texttt{boolean} {\bf  arg2})
}%end signature
}%end item
\item{\vskip -1.5ex 
\texttt{public Point3D {\bf  sceneToLocal}(\texttt{double} {\bf  arg0},
\texttt{double} {\bf  arg1},
\texttt{double} {\bf  arg2})
}%end signature
}%end item
\item{\vskip -1.5ex 
\texttt{public Point2D {\bf  sceneToLocal}(\texttt{javafx.geometry.Point2D} {\bf  arg0})
}%end signature
}%end item
\item{\vskip -1.5ex 
\texttt{public Point2D {\bf  sceneToLocal}(\texttt{javafx.geometry.Point2D} {\bf  arg0},
\texttt{boolean} {\bf  arg1})
}%end signature
}%end item
\item{\vskip -1.5ex 
\texttt{public Point3D {\bf  sceneToLocal}(\texttt{javafx.geometry.Point3D} {\bf  arg0})
}%end signature
}%end item
\item{\vskip -1.5ex 
\texttt{public Bounds {\bf  screenToLocal}(\texttt{javafx.geometry.Bounds} {\bf  arg0})
}%end signature
}%end item
\item{\vskip -1.5ex 
\texttt{public Point2D {\bf  screenToLocal}(\texttt{double} {\bf  arg0},
\texttt{double} {\bf  arg1})
}%end signature
}%end item
\item{\vskip -1.5ex 
\texttt{public Point2D {\bf  screenToLocal}(\texttt{javafx.geometry.Point2D} {\bf  arg0})
}%end signature
}%end item
\item{\vskip -1.5ex 
\texttt{public final void {\bf  setAccessibleHelp}(\texttt{java.lang.String} {\bf  arg0})
}%end signature
}%end item
\item{\vskip -1.5ex 
\texttt{public final void {\bf  setAccessibleRole}(\texttt{AccessibleRole} {\bf  arg0})
}%end signature
}%end item
\item{\vskip -1.5ex 
\texttt{public final void {\bf  setAccessibleRoleDescription}(\texttt{java.lang.String} {\bf  arg0})
}%end signature
}%end item
\item{\vskip -1.5ex 
\texttt{public final void {\bf  setAccessibleText}(\texttt{java.lang.String} {\bf  arg0})
}%end signature
}%end item
\item{\vskip -1.5ex 
\texttt{public final void {\bf  setBlendMode}(\texttt{effect.BlendMode} {\bf  arg0})
}%end signature
}%end item
\item{\vskip -1.5ex 
\texttt{public final void {\bf  setCache}(\texttt{boolean} {\bf  arg0})
}%end signature
}%end item
\item{\vskip -1.5ex 
\texttt{public final void {\bf  setCacheHint}(\texttt{CacheHint} {\bf  arg0})
}%end signature
}%end item
\item{\vskip -1.5ex 
\texttt{public final void {\bf  setClip}(\texttt{Node} {\bf  arg0})
}%end signature
}%end item
\item{\vskip -1.5ex 
\texttt{public final void {\bf  setCursor}(\texttt{Cursor} {\bf  arg0})
}%end signature
}%end item
\item{\vskip -1.5ex 
\texttt{public final void {\bf  setDepthTest}(\texttt{DepthTest} {\bf  arg0})
}%end signature
}%end item
\item{\vskip -1.5ex 
\texttt{public final void {\bf  setDisable}(\texttt{boolean} {\bf  arg0})
}%end signature
}%end item
\item{\vskip -1.5ex 
\texttt{protected final void {\bf  setDisabled}(\texttt{boolean} {\bf  arg0})
}%end signature
}%end item
\item{\vskip -1.5ex 
\texttt{public final void {\bf  setEffect}(\texttt{effect.Effect} {\bf  arg0})
}%end signature
}%end item
\item{\vskip -1.5ex 
\texttt{public final void {\bf  setEventDispatcher}(\texttt{javafx.event.EventDispatcher} {\bf  arg0})
}%end signature
}%end item
\item{\vskip -1.5ex 
\texttt{protected final void {\bf  setEventHandler}(\texttt{javafx.event.EventType} {\bf  arg0},
\texttt{javafx.event.EventHandler} {\bf  arg1})
}%end signature
}%end item
\item{\vskip -1.5ex 
\texttt{protected final void {\bf  setFocused}(\texttt{boolean} {\bf  arg0})
}%end signature
}%end item
\item{\vskip -1.5ex 
\texttt{public final void {\bf  setFocusTraversable}(\texttt{boolean} {\bf  arg0})
}%end signature
}%end item
\item{\vskip -1.5ex 
\texttt{protected final void {\bf  setHover}(\texttt{boolean} {\bf  arg0})
}%end signature
}%end item
\item{\vskip -1.5ex 
\texttt{public final void {\bf  setId}(\texttt{java.lang.String} {\bf  arg0})
}%end signature
}%end item
\item{\vskip -1.5ex 
\texttt{public final void {\bf  setInputMethodRequests}(\texttt{input.InputMethodRequests} {\bf  arg0})
}%end signature
}%end item
\item{\vskip -1.5ex 
\texttt{public final void {\bf  setLayoutX}(\texttt{double} {\bf  arg0})
}%end signature
}%end item
\item{\vskip -1.5ex 
\texttt{public final void {\bf  setLayoutY}(\texttt{double} {\bf  arg0})
}%end signature
}%end item
\item{\vskip -1.5ex 
\texttt{public final void {\bf  setManaged}(\texttt{boolean} {\bf  arg0})
}%end signature
}%end item
\item{\vskip -1.5ex 
\texttt{public final void {\bf  setMouseTransparent}(\texttt{boolean} {\bf  arg0})
}%end signature
}%end item
\item{\vskip -1.5ex 
\texttt{public final void {\bf  setNodeOrientation}(\texttt{javafx.geometry.NodeOrientation} {\bf  arg0})
}%end signature
}%end item
\item{\vskip -1.5ex 
\texttt{public final void {\bf  setOnContextMenuRequested}(\texttt{javafx.event.EventHandler} {\bf  arg0})
}%end signature
}%end item
\item{\vskip -1.5ex 
\texttt{public final void {\bf  setOnDragDetected}(\texttt{javafx.event.EventHandler} {\bf  arg0})
}%end signature
}%end item
\item{\vskip -1.5ex 
\texttt{public final void {\bf  setOnDragDone}(\texttt{javafx.event.EventHandler} {\bf  arg0})
}%end signature
}%end item
\item{\vskip -1.5ex 
\texttt{public final void {\bf  setOnDragDropped}(\texttt{javafx.event.EventHandler} {\bf  arg0})
}%end signature
}%end item
\item{\vskip -1.5ex 
\texttt{public final void {\bf  setOnDragEntered}(\texttt{javafx.event.EventHandler} {\bf  arg0})
}%end signature
}%end item
\item{\vskip -1.5ex 
\texttt{public final void {\bf  setOnDragExited}(\texttt{javafx.event.EventHandler} {\bf  arg0})
}%end signature
}%end item
\item{\vskip -1.5ex 
\texttt{public final void {\bf  setOnDragOver}(\texttt{javafx.event.EventHandler} {\bf  arg0})
}%end signature
}%end item
\item{\vskip -1.5ex 
\texttt{public final void {\bf  setOnInputMethodTextChanged}(\texttt{javafx.event.EventHandler} {\bf  arg0})
}%end signature
}%end item
\item{\vskip -1.5ex 
\texttt{public final void {\bf  setOnKeyPressed}(\texttt{javafx.event.EventHandler} {\bf  arg0})
}%end signature
}%end item
\item{\vskip -1.5ex 
\texttt{public final void {\bf  setOnKeyReleased}(\texttt{javafx.event.EventHandler} {\bf  arg0})
}%end signature
}%end item
\item{\vskip -1.5ex 
\texttt{public final void {\bf  setOnKeyTyped}(\texttt{javafx.event.EventHandler} {\bf  arg0})
}%end signature
}%end item
\item{\vskip -1.5ex 
\texttt{public final void {\bf  setOnMouseClicked}(\texttt{javafx.event.EventHandler} {\bf  arg0})
}%end signature
}%end item
\item{\vskip -1.5ex 
\texttt{public final void {\bf  setOnMouseDragEntered}(\texttt{javafx.event.EventHandler} {\bf  arg0})
}%end signature
}%end item
\item{\vskip -1.5ex 
\texttt{public final void {\bf  setOnMouseDragExited}(\texttt{javafx.event.EventHandler} {\bf  arg0})
}%end signature
}%end item
\item{\vskip -1.5ex 
\texttt{public final void {\bf  setOnMouseDragged}(\texttt{javafx.event.EventHandler} {\bf  arg0})
}%end signature
}%end item
\item{\vskip -1.5ex 
\texttt{public final void {\bf  setOnMouseDragOver}(\texttt{javafx.event.EventHandler} {\bf  arg0})
}%end signature
}%end item
\item{\vskip -1.5ex 
\texttt{public final void {\bf  setOnMouseDragReleased}(\texttt{javafx.event.EventHandler} {\bf  arg0})
}%end signature
}%end item
\item{\vskip -1.5ex 
\texttt{public final void {\bf  setOnMouseEntered}(\texttt{javafx.event.EventHandler} {\bf  arg0})
}%end signature
}%end item
\item{\vskip -1.5ex 
\texttt{public final void {\bf  setOnMouseExited}(\texttt{javafx.event.EventHandler} {\bf  arg0})
}%end signature
}%end item
\item{\vskip -1.5ex 
\texttt{public final void {\bf  setOnMouseMoved}(\texttt{javafx.event.EventHandler} {\bf  arg0})
}%end signature
}%end item
\item{\vskip -1.5ex 
\texttt{public final void {\bf  setOnMousePressed}(\texttt{javafx.event.EventHandler} {\bf  arg0})
}%end signature
}%end item
\item{\vskip -1.5ex 
\texttt{public final void {\bf  setOnMouseReleased}(\texttt{javafx.event.EventHandler} {\bf  arg0})
}%end signature
}%end item
\item{\vskip -1.5ex 
\texttt{public final void {\bf  setOnRotate}(\texttt{javafx.event.EventHandler} {\bf  arg0})
}%end signature
}%end item
\item{\vskip -1.5ex 
\texttt{public final void {\bf  setOnRotationFinished}(\texttt{javafx.event.EventHandler} {\bf  arg0})
}%end signature
}%end item
\item{\vskip -1.5ex 
\texttt{public final void {\bf  setOnRotationStarted}(\texttt{javafx.event.EventHandler} {\bf  arg0})
}%end signature
}%end item
\item{\vskip -1.5ex 
\texttt{public final void {\bf  setOnScroll}(\texttt{javafx.event.EventHandler} {\bf  arg0})
}%end signature
}%end item
\item{\vskip -1.5ex 
\texttt{public final void {\bf  setOnScrollFinished}(\texttt{javafx.event.EventHandler} {\bf  arg0})
}%end signature
}%end item
\item{\vskip -1.5ex 
\texttt{public final void {\bf  setOnScrollStarted}(\texttt{javafx.event.EventHandler} {\bf  arg0})
}%end signature
}%end item
\item{\vskip -1.5ex 
\texttt{public final void {\bf  setOnSwipeDown}(\texttt{javafx.event.EventHandler} {\bf  arg0})
}%end signature
}%end item
\item{\vskip -1.5ex 
\texttt{public final void {\bf  setOnSwipeLeft}(\texttt{javafx.event.EventHandler} {\bf  arg0})
}%end signature
}%end item
\item{\vskip -1.5ex 
\texttt{public final void {\bf  setOnSwipeRight}(\texttt{javafx.event.EventHandler} {\bf  arg0})
}%end signature
}%end item
\item{\vskip -1.5ex 
\texttt{public final void {\bf  setOnSwipeUp}(\texttt{javafx.event.EventHandler} {\bf  arg0})
}%end signature
}%end item
\item{\vskip -1.5ex 
\texttt{public final void {\bf  setOnTouchMoved}(\texttt{javafx.event.EventHandler} {\bf  arg0})
}%end signature
}%end item
\item{\vskip -1.5ex 
\texttt{public final void {\bf  setOnTouchPressed}(\texttt{javafx.event.EventHandler} {\bf  arg0})
}%end signature
}%end item
\item{\vskip -1.5ex 
\texttt{public final void {\bf  setOnTouchReleased}(\texttt{javafx.event.EventHandler} {\bf  arg0})
}%end signature
}%end item
\item{\vskip -1.5ex 
\texttt{public final void {\bf  setOnTouchStationary}(\texttt{javafx.event.EventHandler} {\bf  arg0})
}%end signature
}%end item
\item{\vskip -1.5ex 
\texttt{public final void {\bf  setOnZoom}(\texttt{javafx.event.EventHandler} {\bf  arg0})
}%end signature
}%end item
\item{\vskip -1.5ex 
\texttt{public final void {\bf  setOnZoomFinished}(\texttt{javafx.event.EventHandler} {\bf  arg0})
}%end signature
}%end item
\item{\vskip -1.5ex 
\texttt{public final void {\bf  setOnZoomStarted}(\texttt{javafx.event.EventHandler} {\bf  arg0})
}%end signature
}%end item
\item{\vskip -1.5ex 
\texttt{public final void {\bf  setOpacity}(\texttt{double} {\bf  arg0})
}%end signature
}%end item
\item{\vskip -1.5ex 
\texttt{public final void {\bf  setPickOnBounds}(\texttt{boolean} {\bf  arg0})
}%end signature
}%end item
\item{\vskip -1.5ex 
\texttt{protected final void {\bf  setPressed}(\texttt{boolean} {\bf  arg0})
}%end signature
}%end item
\item{\vskip -1.5ex 
\texttt{public final void {\bf  setRotate}(\texttt{double} {\bf  arg0})
}%end signature
}%end item
\item{\vskip -1.5ex 
\texttt{public final void {\bf  setRotationAxis}(\texttt{javafx.geometry.Point3D} {\bf  arg0})
}%end signature
}%end item
\item{\vskip -1.5ex 
\texttt{public final void {\bf  setScaleX}(\texttt{double} {\bf  arg0})
}%end signature
}%end item
\item{\vskip -1.5ex 
\texttt{public final void {\bf  setScaleY}(\texttt{double} {\bf  arg0})
}%end signature
}%end item
\item{\vskip -1.5ex 
\texttt{public final void {\bf  setScaleZ}(\texttt{double} {\bf  arg0})
}%end signature
}%end item
\item{\vskip -1.5ex 
\texttt{public final void {\bf  setStyle}(\texttt{java.lang.String} {\bf  arg0})
}%end signature
}%end item
\item{\vskip -1.5ex 
\texttt{public final void {\bf  setTranslateX}(\texttt{double} {\bf  arg0})
}%end signature
}%end item
\item{\vskip -1.5ex 
\texttt{public final void {\bf  setTranslateY}(\texttt{double} {\bf  arg0})
}%end signature
}%end item
\item{\vskip -1.5ex 
\texttt{public final void {\bf  setTranslateZ}(\texttt{double} {\bf  arg0})
}%end signature
}%end item
\item{\vskip -1.5ex 
\texttt{public void {\bf  setUserData}(\texttt{java.lang.Object} {\bf  arg0})
}%end signature
}%end item
\item{\vskip -1.5ex 
\texttt{public final void {\bf  setVisible}(\texttt{boolean} {\bf  arg0})
}%end signature
}%end item
\item{\vskip -1.5ex 
\texttt{public void {\bf  snapshot}(\texttt{javafx.util.Callback} {\bf  arg0},
\texttt{SnapshotParameters} {\bf  arg1},
\texttt{image.WritableImage} {\bf  arg2})
}%end signature
}%end item
\item{\vskip -1.5ex 
\texttt{public WritableImage {\bf  snapshot}(\texttt{SnapshotParameters} {\bf  arg0},
\texttt{image.WritableImage} {\bf  arg1})
}%end signature
}%end item
\item{\vskip -1.5ex 
\texttt{public Dragboard {\bf  startDragAndDrop}(\texttt{input.TransferMode\lbrack \rbrack } {\bf  arg0})
}%end signature
}%end item
\item{\vskip -1.5ex 
\texttt{public void {\bf  startFullDrag}()
}%end signature
}%end item
\item{\vskip -1.5ex 
\texttt{public final StringProperty {\bf  styleProperty}()
}%end signature
}%end item
\item{\vskip -1.5ex 
\texttt{public void {\bf  toBack}()
}%end signature
}%end item
\item{\vskip -1.5ex 
\texttt{public void {\bf  toFront}()
}%end signature
}%end item
\item{\vskip -1.5ex 
\texttt{public String {\bf  toString}()
}%end signature
}%end item
\item{\vskip -1.5ex 
\texttt{public final DoubleProperty {\bf  translateXProperty}()
}%end signature
}%end item
\item{\vskip -1.5ex 
\texttt{public final DoubleProperty {\bf  translateYProperty}()
}%end signature
}%end item
\item{\vskip -1.5ex 
\texttt{public final DoubleProperty {\bf  translateZProperty}()
}%end signature
}%end item
\item{\vskip -1.5ex 
\texttt{public boolean {\bf  usesMirroring}()
}%end signature
}%end item
\item{\vskip -1.5ex 
\texttt{public final BooleanProperty {\bf  visibleProperty}()
}%end signature
}%end item
\end{itemize}
}
}
\section{\label{vue.DetailFenetre}Class DetailFenetre}{
\hypertarget{vue.DetailFenetre}{}\vskip .1in 
Gére l'affichage sous forme textuelle des details d'une fenêtre de livraison dans la TreeTableView.\vskip .1in 
\subsection{Declaration}{
\begin{lstlisting}[frame=none]
public class DetailFenetre
 extends vue.ObjetVisualisable\end{lstlisting}
\subsection{Constructor summary}{
\begin{verse}
\hyperlink{vue.DetailFenetre(modele.donneesxml.Fenetre)}{{\bf DetailFenetre(Fenetre)}} Constructeur du détail fenêtre\\
\end{verse}
}
\subsection{Method summary}{
\begin{verse}
\hyperlink{vue.DetailFenetre.afficherCaracteriquesGlobales()}{{\bf afficherCaracteriquesGlobales()}} \\
\hyperlink{vue.DetailFenetre.afficherCaracteriquesSpeciales()}{{\bf afficherCaracteriquesSpeciales()}} \\
\hyperlink{vue.DetailFenetre.getFenetre()}{{\bf getFenetre()}} \\
\end{verse}
}
\subsection{Constructors}{
\vskip -2em
\begin{itemize}
\item{ 
\index{DetailFenetre(Fenetre)}
\hypertarget{vue.DetailFenetre(modele.donneesxml.Fenetre)}{{\bf  DetailFenetre}\\}
\begin{lstlisting}[frame=none]
public DetailFenetre(modele.donneesxml.Fenetre fenetre)\end{lstlisting} %end signature
\begin{itemize}
\item{
{\bf  Description}

Constructeur du détail fenêtre
}
\item{
{\bf  Parameters}
  \begin{itemize}
   \item{
\texttt{fenetre} -- La fenêtre associée}
  \end{itemize}
}%end item
\end{itemize}
}%end item
\end{itemize}
}
\subsection{Methods}{
\vskip -2em
\begin{itemize}
\item{ 
\index{afficherCaracteriquesGlobales()}
\hypertarget{vue.DetailFenetre.afficherCaracteriquesGlobales()}{{\bf  afficherCaracteriquesGlobales}\\}
\begin{lstlisting}[frame=none]
public abstract java.lang.String afficherCaracteriquesGlobales()\end{lstlisting} %end signature
\begin{itemize}
\item{{\bf  Returns} -- 
Les caractéristiques globales de l'élément affiché 
}%end item
\end{itemize}
}%end item
\item{ 
\index{afficherCaracteriquesSpeciales()}
\hypertarget{vue.DetailFenetre.afficherCaracteriquesSpeciales()}{{\bf  afficherCaracteriquesSpeciales}\\}
\begin{lstlisting}[frame=none]
public abstract java.lang.String afficherCaracteriquesSpeciales()\end{lstlisting} %end signature
\begin{itemize}
\item{{\bf  Returns} -- 
Les caractéristiques spéciales de l'élément affiché 
}%end item
\end{itemize}
}%end item
\item{ 
\index{getFenetre()}
\hypertarget{vue.DetailFenetre.getFenetre()}{{\bf  getFenetre}\\}
\begin{lstlisting}[frame=none]
public modele.donneesxml.Fenetre getFenetre()\end{lstlisting} %end signature
\begin{itemize}
\item{{\bf  Returns} -- 
La fenêtre associée à ce détail 
}%end item
\end{itemize}
}%end item
\end{itemize}
}
\subsection{Members inherited from class ObjetVisualisable }{
\texttt{vue.ObjetVisualisable} {\small 
\refdefined{vue.ObjetVisualisable}}
{\small 

\vskip -2em
\begin{itemize}
\item{\vskip -1.5ex 
\texttt{public abstract String {\bf  afficherCaracteriquesGlobales}()
}%end signature
}%end item
\item{\vskip -1.5ex 
\texttt{public abstract String {\bf  afficherCaracteriquesSpeciales}()
}%end signature
}%end item
\item{\vskip -1.5ex 
\texttt{protected static String {\bf  convertirEnHeureLisible}(\texttt{int} {\bf  tempsEnSeconde})
}%end signature
}%end item
\item{\vskip -1.5ex 
\texttt{public ObjetVisualisable.CouleurTexte {\bf  getCouleurDefaut}()
}%end signature
}%end item
\item{\vskip -1.5ex 
\texttt{public void {\bf  setCouleurDefaut}(\texttt{ObjetVisualisable.CouleurTexte} {\bf  couleur})
}%end signature
}%end item
\end{itemize}
}
}
\section{\label{vue.DetailLivraison}Class DetailLivraison}{
\hypertarget{vue.DetailLivraison}{}\vskip .1in 
Gére l'affichage sous forme textuelle des details d'une livraison dans la TreeTableView.\vskip .1in 
\subsection{Declaration}{
\begin{lstlisting}[frame=none]
public class DetailLivraison
 extends vue.ObjetVisualisable\end{lstlisting}
\subsection{Constructor summary}{
\begin{verse}
\hyperlink{vue.DetailLivraison(modele.donneesxml.Livraison)}{{\bf DetailLivraison(Livraison)}} Constructeur du détail de livraison\\
\end{verse}
}
\subsection{Method summary}{
\begin{verse}
\hyperlink{vue.DetailLivraison.afficherCaracteriquesGlobales()}{{\bf afficherCaracteriquesGlobales()}} \\
\hyperlink{vue.DetailLivraison.afficherCaracteriquesSpeciales()}{{\bf afficherCaracteriquesSpeciales()}} \\
\hyperlink{vue.DetailLivraison.getLivraison()}{{\bf getLivraison()}} \\
\end{verse}
}
\subsection{Constructors}{
\vskip -2em
\begin{itemize}
\item{ 
\index{DetailLivraison(Livraison)}
\hypertarget{vue.DetailLivraison(modele.donneesxml.Livraison)}{{\bf  DetailLivraison}\\}
\begin{lstlisting}[frame=none]
public DetailLivraison(modele.donneesxml.Livraison livraison)\end{lstlisting} %end signature
\begin{itemize}
\item{
{\bf  Description}

Constructeur du détail de livraison
}
\item{
{\bf  Parameters}
  \begin{itemize}
   \item{
\texttt{livraison} -- La livraison associée}
  \end{itemize}
}%end item
\end{itemize}
}%end item
\end{itemize}
}
\subsection{Methods}{
\vskip -2em
\begin{itemize}
\item{ 
\index{afficherCaracteriquesGlobales()}
\hypertarget{vue.DetailLivraison.afficherCaracteriquesGlobales()}{{\bf  afficherCaracteriquesGlobales}\\}
\begin{lstlisting}[frame=none]
public abstract java.lang.String afficherCaracteriquesGlobales()\end{lstlisting} %end signature
\begin{itemize}
\item{{\bf  Returns} -- 
Les caractéristiques globales de l'élément affiché 
}%end item
\end{itemize}
}%end item
\item{ 
\index{afficherCaracteriquesSpeciales()}
\hypertarget{vue.DetailLivraison.afficherCaracteriquesSpeciales()}{{\bf  afficherCaracteriquesSpeciales}\\}
\begin{lstlisting}[frame=none]
public abstract java.lang.String afficherCaracteriquesSpeciales()\end{lstlisting} %end signature
\begin{itemize}
\item{{\bf  Returns} -- 
Les caractéristiques spéciales de l'élément affiché 
}%end item
\end{itemize}
}%end item
\item{ 
\index{getLivraison()}
\hypertarget{vue.DetailLivraison.getLivraison()}{{\bf  getLivraison}\\}
\begin{lstlisting}[frame=none]
public modele.donneesxml.Livraison getLivraison()\end{lstlisting} %end signature
\begin{itemize}
\item{{\bf  Returns} -- 
La livraison associée 
}%end item
\end{itemize}
}%end item
\end{itemize}
}
\subsection{Members inherited from class ObjetVisualisable }{
\texttt{vue.ObjetVisualisable} {\small 
\refdefined{vue.ObjetVisualisable}}
{\small 

\vskip -2em
\begin{itemize}
\item{\vskip -1.5ex 
\texttt{public abstract String {\bf  afficherCaracteriquesGlobales}()
}%end signature
}%end item
\item{\vskip -1.5ex 
\texttt{public abstract String {\bf  afficherCaracteriquesSpeciales}()
}%end signature
}%end item
\item{\vskip -1.5ex 
\texttt{protected static String {\bf  convertirEnHeureLisible}(\texttt{int} {\bf  tempsEnSeconde})
}%end signature
}%end item
\item{\vskip -1.5ex 
\texttt{public ObjetVisualisable.CouleurTexte {\bf  getCouleurDefaut}()
}%end signature
}%end item
\item{\vskip -1.5ex 
\texttt{public void {\bf  setCouleurDefaut}(\texttt{ObjetVisualisable.CouleurTexte} {\bf  couleur})
}%end signature
}%end item
\end{itemize}
}
}
\section{\label{vue.FenetrePrincipale}Class FenetrePrincipale}{
\hypertarget{vue.FenetrePrincipale}{}\vskip .1in 
Cette classe crée la fenetre principale avec ses enfants. Elle se charge aussi de créer le controleur.\vskip .1in 
\subsection{Declaration}{
\begin{lstlisting}[frame=none]
public class FenetrePrincipale
 extends javafx.application.Application\end{lstlisting}
\subsection{Constructor summary}{
\begin{verse}
\hyperlink{vue.FenetrePrincipale()}{{\bf FenetrePrincipale()}} \\
\end{verse}
}
\subsection{Method summary}{
\begin{verse}
\hyperlink{vue.FenetrePrincipale.start(javafx.stage.Stage)}{{\bf start(Stage)}} Lance l'application\\
\end{verse}
}
\subsection{Constructors}{
\vskip -2em
\begin{itemize}
\item{ 
\index{FenetrePrincipale()}
\hypertarget{vue.FenetrePrincipale()}{{\bf  FenetrePrincipale}\\}
\begin{lstlisting}[frame=none]
public FenetrePrincipale()\end{lstlisting} %end signature
}%end item
\end{itemize}
}
\subsection{Methods}{
\vskip -2em
\begin{itemize}
\item{ 
\index{start(Stage)}
\hypertarget{vue.FenetrePrincipale.start(javafx.stage.Stage)}{{\bf  start}\\}
\begin{lstlisting}[frame=none]
public void start(javafx.stage.Stage primaryStage) throws java.lang.Exception\end{lstlisting} %end signature
\begin{itemize}
\item{
{\bf  Description}

Lance l'application
}
\item{
{\bf  Parameters}
  \begin{itemize}
   \item{
\texttt{primaryStage} -- objet père qui contient la fenetre principale}
  \end{itemize}
}%end item
\end{itemize}
}%end item
\end{itemize}
}
\subsection{Members inherited from class Application }{
\texttt{javafx.application.Application} {\small 
\refdefined{javafx.application.Application}}
{\small 

\vskip -2em
\begin{itemize}
\item{\vskip -1.5ex 
\texttt{public final HostServices {\bf  getHostServices}()
}%end signature
}%end item
\item{\vskip -1.5ex 
\texttt{public final Application.Parameters {\bf  getParameters}()
}%end signature
}%end item
\item{\vskip -1.5ex 
\texttt{public static String {\bf  getUserAgentStylesheet}()
}%end signature
}%end item
\item{\vskip -1.5ex 
\texttt{public void {\bf  init}() throws java.lang.Exception
}%end signature
}%end item
\item{\vskip -1.5ex 
\texttt{public static void {\bf  launch}(\texttt{java.lang.Class} {\bf  arg0},
\texttt{java.lang.String\lbrack \rbrack } {\bf  arg1})
}%end signature
}%end item
\item{\vskip -1.5ex 
\texttt{public static void {\bf  launch}(\texttt{java.lang.String\lbrack \rbrack } {\bf  arg0})
}%end signature
}%end item
\item{\vskip -1.5ex 
\texttt{public final void {\bf  notifyPreloader}(\texttt{Preloader.PreloaderNotification} {\bf  arg0})
}%end signature
}%end item
\item{\vskip -1.5ex 
\texttt{public static void {\bf  setUserAgentStylesheet}(\texttt{java.lang.String} {\bf  arg0})
}%end signature
}%end item
\item{\vskip -1.5ex 
\texttt{public abstract void {\bf  start}(\texttt{javafx.stage.Stage} {\bf  arg0}) throws java.lang.Exception
}%end signature
}%end item
\item{\vskip -1.5ex 
\texttt{public void {\bf  stop}() throws java.lang.Exception
}%end signature
}%end item
\item{\vskip -1.5ex 
\texttt{public static final {\bf  STYLESHEET\_CASPIAN}}%end signature
}%end item
\item{\vskip -1.5ex 
\texttt{public static final {\bf  STYLESHEET\_MODENA}}%end signature
}%end item
\end{itemize}
}
}
\section{\label{vue.ObjetVisualisable}Class ObjetVisualisable}{
\hypertarget{vue.ObjetVisualisable}{}\vskip .1in 
Cette classe permet de visualiser une fenêtre (de livraison) ou une livraison sous forme textuelle. Cette visualisation peut se faire de deux façons façons différentes en fonction du type concret de l'objet.\vskip .1in 
\subsection{Declaration}{
\begin{lstlisting}[frame=none]
public abstract class ObjetVisualisable
 extends java.lang.Object\end{lstlisting}
\subsection{All known subclasses}{DetailFenetre\small{\refdefined{vue.DetailFenetre}}, DetailLivraison\small{\refdefined{vue.DetailLivraison}}}
\subsection{Constructor summary}{
\begin{verse}
\hyperlink{vue.ObjetVisualisable()}{{\bf ObjetVisualisable()}} \\
\end{verse}
}
\subsection{Method summary}{
\begin{verse}
\hyperlink{vue.ObjetVisualisable.afficherCaracteriquesGlobales()}{{\bf afficherCaracteriquesGlobales()}} \\
\hyperlink{vue.ObjetVisualisable.afficherCaracteriquesSpeciales()}{{\bf afficherCaracteriquesSpeciales()}} \\
\hyperlink{vue.ObjetVisualisable.convertirEnHeureLisible(int)}{{\bf convertirEnHeureLisible(int)}} Convertit un temps en seconde en chaine de caractère sous la forme HH:mm:ss\\
\hyperlink{vue.ObjetVisualisable.getCouleurDefaut()}{{\bf getCouleurDefaut()}} \\
\hyperlink{vue.ObjetVisualisable.setCouleurDefaut(vue.ObjetVisualisable.CouleurTexte)}{{\bf setCouleurDefaut(ObjetVisualisable.CouleurTexte)}} \\
\end{verse}
}
\subsection{Constructors}{
\vskip -2em
\begin{itemize}
\item{ 
\index{ObjetVisualisable()}
\hypertarget{vue.ObjetVisualisable()}{{\bf  ObjetVisualisable}\\}
\begin{lstlisting}[frame=none]
public ObjetVisualisable()\end{lstlisting} %end signature
}%end item
\end{itemize}
}
\subsection{Methods}{
\vskip -2em
\begin{itemize}
\item{ 
\index{afficherCaracteriquesGlobales()}
\hypertarget{vue.ObjetVisualisable.afficherCaracteriquesGlobales()}{{\bf  afficherCaracteriquesGlobales}\\}
\begin{lstlisting}[frame=none]
public abstract java.lang.String afficherCaracteriquesGlobales()\end{lstlisting} %end signature
\begin{itemize}
\item{{\bf  Returns} -- 
Les caractéristiques globales de l'élément affiché 
}%end item
\end{itemize}
}%end item
\item{ 
\index{afficherCaracteriquesSpeciales()}
\hypertarget{vue.ObjetVisualisable.afficherCaracteriquesSpeciales()}{{\bf  afficherCaracteriquesSpeciales}\\}
\begin{lstlisting}[frame=none]
public abstract java.lang.String afficherCaracteriquesSpeciales()\end{lstlisting} %end signature
\begin{itemize}
\item{{\bf  Returns} -- 
Les caractéristiques spéciales de l'élément affiché 
}%end item
\end{itemize}
}%end item
\item{ 
\index{convertirEnHeureLisible(int)}
\hypertarget{vue.ObjetVisualisable.convertirEnHeureLisible(int)}{{\bf  convertirEnHeureLisible}\\}
\begin{lstlisting}[frame=none]
protected static java.lang.String convertirEnHeureLisible(int tempsEnSeconde)\end{lstlisting} %end signature
\begin{itemize}
\item{
{\bf  Description}

Convertit un temps en seconde en chaine de caractère sous la forme HH:mm:ss
}
\item{
{\bf  Parameters}
  \begin{itemize}
   \item{
\texttt{tempsEnSeconde} -- temps à convertir}
  \end{itemize}
}%end item
\end{itemize}
}%end item
\item{ 
\index{getCouleurDefaut()}
\hypertarget{vue.ObjetVisualisable.getCouleurDefaut()}{{\bf  getCouleurDefaut}\\}
\begin{lstlisting}[frame=none]
public ObjetVisualisable.CouleurTexte getCouleurDefaut()\end{lstlisting} %end signature
\begin{itemize}
\item{{\bf  Returns} -- 
Retourne la couleur actuelle 
}%end item
\end{itemize}
}%end item
\item{ 
\index{setCouleurDefaut(ObjetVisualisable.CouleurTexte)}
\hypertarget{vue.ObjetVisualisable.setCouleurDefaut(vue.ObjetVisualisable.CouleurTexte)}{{\bf  setCouleurDefaut}\\}
\begin{lstlisting}[frame=none]
public void setCouleurDefaut(ObjetVisualisable.CouleurTexte couleur)\end{lstlisting} %end signature
\begin{itemize}
\item{
{\bf  Parameters}
  \begin{itemize}
   \item{
\texttt{couleur} -- La nouvelle couleur}
  \end{itemize}
}%end item
\end{itemize}
}%end item
\end{itemize}
}
}
\section{\label{vue.ObjetVisualisable.CouleurTexte}Class ObjetVisualisable.CouleurTexte}{
\hypertarget{vue.ObjetVisualisable.CouleurTexte}{}\vskip .1in 
Différentes couleurs possibles pour un élément dans la liste.\vskip .1in 
\subsection{Declaration}{
\begin{lstlisting}[frame=none]
public static final class ObjetVisualisable.CouleurTexte
 extends java.lang.Enum\end{lstlisting}
\subsection{Field summary}{
\begin{verse}
\hyperlink{vue.ObjetVisualisable.CouleurTexte.NON_SURBRILLANCE}{{\bf NON\_SURBRILLANCE}} \\
\hyperlink{vue.ObjetVisualisable.CouleurTexte.RETARD}{{\bf RETARD}} \\
\hyperlink{vue.ObjetVisualisable.CouleurTexte.SURBRILLANCE}{{\bf SURBRILLANCE}} \\
\end{verse}
}
\subsection{Method summary}{
\begin{verse}
\hyperlink{vue.ObjetVisualisable.CouleurTexte.valueOf(java.lang.String)}{{\bf valueOf(String)}} \\
\hyperlink{vue.ObjetVisualisable.CouleurTexte.values()}{{\bf values()}} \\
\end{verse}
}
\subsection{Fields}{
\begin{itemize}
\item{
\index{SURBRILLANCE}
\label{vue.ObjetVisualisable.CouleurTexte.SURBRILLANCE}\hypertarget{vue.ObjetVisualisable.CouleurTexte.SURBRILLANCE}{\texttt{public static final ObjetVisualisable.CouleurTexte\ {\bf  SURBRILLANCE}}
}
}
\item{
\index{NON\_SURBRILLANCE}
\label{vue.ObjetVisualisable.CouleurTexte.NON_SURBRILLANCE}\hypertarget{vue.ObjetVisualisable.CouleurTexte.NON_SURBRILLANCE}{\texttt{public static final ObjetVisualisable.CouleurTexte\ {\bf  NON\_SURBRILLANCE}}
}
}
\item{
\index{RETARD}
\label{vue.ObjetVisualisable.CouleurTexte.RETARD}\hypertarget{vue.ObjetVisualisable.CouleurTexte.RETARD}{\texttt{public static final ObjetVisualisable.CouleurTexte\ {\bf  RETARD}}
}
}
\end{itemize}
}
\subsection{Methods}{
\vskip -2em
\begin{itemize}
\item{ 
\index{valueOf(String)}
\hypertarget{vue.ObjetVisualisable.CouleurTexte.valueOf(java.lang.String)}{{\bf  valueOf}\\}
\begin{lstlisting}[frame=none]
public static ObjetVisualisable.CouleurTexte valueOf(java.lang.String name)\end{lstlisting} %end signature
}%end item
\item{ 
\index{values()}
\hypertarget{vue.ObjetVisualisable.CouleurTexte.values()}{{\bf  values}\\}
\begin{lstlisting}[frame=none]
public static ObjetVisualisable.CouleurTexte[] values()\end{lstlisting} %end signature
}%end item
\end{itemize}
}
\subsection{Members inherited from class Enum }{
\texttt{java.lang.Enum} {\small 
\refdefined{java.lang.Enum}}
{\small 

\vskip -2em
\begin{itemize}
\item{\vskip -1.5ex 
\texttt{protected final Object {\bf  clone}() throws CloneNotSupportedException
}%end signature
}%end item
\item{\vskip -1.5ex 
\texttt{public final int {\bf  compareTo}(\texttt{Enum} {\bf  arg0})
}%end signature
}%end item
\item{\vskip -1.5ex 
\texttt{public final boolean {\bf  equals}(\texttt{Object} {\bf  arg0})
}%end signature
}%end item
\item{\vskip -1.5ex 
\texttt{protected final void {\bf  finalize}()
}%end signature
}%end item
\item{\vskip -1.5ex 
\texttt{public final Class {\bf  getDeclaringClass}()
}%end signature
}%end item
\item{\vskip -1.5ex 
\texttt{public final int {\bf  hashCode}()
}%end signature
}%end item
\item{\vskip -1.5ex 
\texttt{public final String {\bf  name}()
}%end signature
}%end item
\item{\vskip -1.5ex 
\texttt{public final int {\bf  ordinal}()
}%end signature
}%end item
\item{\vskip -1.5ex 
\texttt{public String {\bf  toString}()
}%end signature
}%end item
\item{\vskip -1.5ex 
\texttt{public static Enum {\bf  valueOf}(\texttt{Class} {\bf  arg0},
\texttt{String} {\bf  arg1})
}%end signature
}%end item
\end{itemize}
}
}
\section{\label{vue.ObserveurMessageChamps}Class ObserveurMessageChamps}{
\hypertarget{vue.ObserveurMessageChamps}{}\vskip .1in 
Champ texte, écouteur des messages qui peuvent être reçu\vskip .1in 
\subsection{Declaration}{
\begin{lstlisting}[frame=none]
public class ObserveurMessageChamps
 extends javafx.scene.text.Text implements controleur.observateur.MessageObservateur\end{lstlisting}
\subsection{Constructor summary}{
\begin{verse}
\hyperlink{vue.ObserveurMessageChamps()}{{\bf ObserveurMessageChamps()}} \\
\end{verse}
}
\subsection{Method summary}{
\begin{verse}
\hyperlink{vue.ObserveurMessageChamps.notifierObservateursMessage(java.lang.String)}{{\bf notifierObservateursMessage(String)}} \\
\end{verse}
}
\subsection{Constructors}{
\vskip -2em
\begin{itemize}
\item{ 
\index{ObserveurMessageChamps()}
\hypertarget{vue.ObserveurMessageChamps()}{{\bf  ObserveurMessageChamps}\\}
\begin{lstlisting}[frame=none]
public ObserveurMessageChamps()\end{lstlisting} %end signature
}%end item
\end{itemize}
}
\subsection{Methods}{
\vskip -2em
\begin{itemize}
\item{ 
\index{notifierObservateursMessage(String)}
\hypertarget{vue.ObserveurMessageChamps.notifierObservateursMessage(java.lang.String)}{{\bf  notifierObservateursMessage}\\}
\begin{lstlisting}[frame=none]
void notifierObservateursMessage(java.lang.String message)\end{lstlisting} %end signature
\begin{itemize}
\item{
{\bf  Description copied from \hyperlink{controleur.observateur.MessageObservateur}{controleur.observateur.MessageObservateur}{\small \refdefined{controleur.observateur.MessageObservateur}} }

Notifie les observateurs qu'il doit afficher un nouveau message
}
\item{
{\bf  Parameters}
  \begin{itemize}
   \item{
\texttt{message} -- Le message envoyé}
  \end{itemize}
}%end item
\end{itemize}
}%end item
\end{itemize}
}
\subsection{Members inherited from class Text }{
\texttt{javafx.scene.text.Text} {\small 
\refdefined{javafx.scene.text.Text}}
{\small 

\vskip -2em
\begin{itemize}
\item{\vskip -1.5ex 
\texttt{public final ReadOnlyDoubleProperty {\bf  baselineOffsetProperty}()
}%end signature
}%end item
\item{\vskip -1.5ex 
\texttt{public final ObjectProperty {\bf  boundsTypeProperty}()
}%end signature
}%end item
\item{\vskip -1.5ex 
\texttt{public final ObjectProperty {\bf  fontProperty}()
}%end signature
}%end item
\item{\vskip -1.5ex 
\texttt{public final ObjectProperty {\bf  fontSmoothingTypeProperty}()
}%end signature
}%end item
\item{\vskip -1.5ex 
\texttt{public final double {\bf  getBaselineOffset}()
}%end signature
}%end item
\item{\vskip -1.5ex 
\texttt{public final TextBoundsType {\bf  getBoundsType}()
}%end signature
}%end item
\item{\vskip -1.5ex 
\texttt{public static List {\bf  getClassCssMetaData}()
}%end signature
}%end item
\item{\vskip -1.5ex 
\texttt{public List {\bf  getCssMetaData}()
}%end signature
}%end item
\item{\vskip -1.5ex 
\texttt{public final Font {\bf  getFont}()
}%end signature
}%end item
\item{\vskip -1.5ex 
\texttt{public final FontSmoothingType {\bf  getFontSmoothingType}()
}%end signature
}%end item
\item{\vskip -1.5ex 
\texttt{public final int {\bf  getImpl\_caretPosition}()
}%end signature
}%end item
\item{\vskip -1.5ex 
\texttt{public final PathElement {\bf  getImpl\_caretShape}()
}%end signature
}%end item
\item{\vskip -1.5ex 
\texttt{public final int {\bf  getImpl\_selectionEnd}()
}%end signature
}%end item
\item{\vskip -1.5ex 
\texttt{public final PathElement {\bf  getImpl\_selectionShape}()
}%end signature
}%end item
\item{\vskip -1.5ex 
\texttt{public final int {\bf  getImpl\_selectionStart}()
}%end signature
}%end item
\item{\vskip -1.5ex 
\texttt{public final double {\bf  getLineSpacing}()
}%end signature
}%end item
\item{\vskip -1.5ex 
\texttt{public final String {\bf  getText}()
}%end signature
}%end item
\item{\vskip -1.5ex 
\texttt{public final TextAlignment {\bf  getTextAlignment}()
}%end signature
}%end item
\item{\vskip -1.5ex 
\texttt{public final VPos {\bf  getTextOrigin}()
}%end signature
}%end item
\item{\vskip -1.5ex 
\texttt{public final double {\bf  getWrappingWidth}()
}%end signature
}%end item
\item{\vskip -1.5ex 
\texttt{public final double {\bf  getX}()
}%end signature
}%end item
\item{\vskip -1.5ex 
\texttt{public final double {\bf  getY}()
}%end signature
}%end item
\item{\vskip -1.5ex 
\texttt{public final BooleanProperty {\bf  impl\_caretBiasProperty}()
}%end signature
}%end item
\item{\vskip -1.5ex 
\texttt{public final IntegerProperty {\bf  impl\_caretPositionProperty}()
}%end signature
}%end item
\item{\vskip -1.5ex 
\texttt{public final ReadOnlyObjectProperty {\bf  impl\_caretShapeProperty}()
}%end signature
}%end item
\item{\vskip -1.5ex 
\texttt{protected final boolean {\bf  impl\_computeContains}(\texttt{double} {\bf  arg0},
\texttt{double} {\bf  arg1})
}%end signature
}%end item
\item{\vskip -1.5ex 
\texttt{public final BaseBounds {\bf  impl\_computeGeomBounds}(\texttt{com.sun.javafx.geom.BaseBounds} {\bf  arg0},
\texttt{com.sun.javafx.geom.transform.BaseTransform} {\bf  arg1})
}%end signature
}%end item
\item{\vskip -1.5ex 
\texttt{protected final Bounds {\bf  impl\_computeLayoutBounds}()
}%end signature
}%end item
\item{\vskip -1.5ex 
\texttt{public final Shape {\bf  impl\_configShape}()
}%end signature
}%end item
\item{\vskip -1.5ex 
\texttt{protected final NGNode {\bf  impl\_createPeer}()
}%end signature
}%end item
\item{\vskip -1.5ex 
\texttt{public final void {\bf  impl\_displaySoftwareKeyboard}(\texttt{boolean} {\bf  arg0})
}%end signature
}%end item
\item{\vskip -1.5ex 
\texttt{protected final void {\bf  impl\_geomChanged}()
}%end signature
}%end item
\item{\vskip -1.5ex 
\texttt{public final PathElement {\bf  impl\_getRangeShape}(\texttt{int} {\bf  arg0},
\texttt{int} {\bf  arg1})
}%end signature
}%end item
\item{\vskip -1.5ex 
\texttt{public final PathElement {\bf  impl\_getUnderlineShape}(\texttt{int} {\bf  arg0},
\texttt{int} {\bf  arg1})
}%end signature
}%end item
\item{\vskip -1.5ex 
\texttt{public final HitInfo {\bf  impl\_hitTestChar}(\texttt{javafx.geometry.Point2D} {\bf  arg0})
}%end signature
}%end item
\item{\vskip -1.5ex 
\texttt{public final IntegerProperty {\bf  impl\_selectionEndProperty}()
}%end signature
}%end item
\item{\vskip -1.5ex 
\texttt{public final ObjectProperty {\bf  impl\_selectionFillProperty}()
}%end signature
}%end item
\item{\vskip -1.5ex 
\texttt{public final ReadOnlyObjectProperty {\bf  impl\_selectionShapeProperty}()
}%end signature
}%end item
\item{\vskip -1.5ex 
\texttt{public final IntegerProperty {\bf  impl\_selectionStartProperty}()
}%end signature
}%end item
\item{\vskip -1.5ex 
\texttt{public final void {\bf  impl\_updatePeer}()
}%end signature
}%end item
\item{\vskip -1.5ex 
\texttt{public final boolean {\bf  isImpl\_caretBias}()
}%end signature
}%end item
\item{\vskip -1.5ex 
\texttt{public final boolean {\bf  isStrikethrough}()
}%end signature
}%end item
\item{\vskip -1.5ex 
\texttt{public final boolean {\bf  isUnderline}()
}%end signature
}%end item
\item{\vskip -1.5ex 
\texttt{public final DoubleProperty {\bf  lineSpacingProperty}()
}%end signature
}%end item
\item{\vskip -1.5ex 
\texttt{public Object {\bf  queryAccessibleAttribute}(\texttt{javafx.scene.AccessibleAttribute} {\bf  arg0},
\texttt{java.lang.Object\lbrack \rbrack } {\bf  arg1})
}%end signature
}%end item
\item{\vskip -1.5ex 
\texttt{public final void {\bf  setBoundsType}(\texttt{TextBoundsType} {\bf  arg0})
}%end signature
}%end item
\item{\vskip -1.5ex 
\texttt{public final void {\bf  setFont}(\texttt{Font} {\bf  arg0})
}%end signature
}%end item
\item{\vskip -1.5ex 
\texttt{public final void {\bf  setFontSmoothingType}(\texttt{FontSmoothingType} {\bf  arg0})
}%end signature
}%end item
\item{\vskip -1.5ex 
\texttt{public final void {\bf  setImpl\_caretBias}(\texttt{boolean} {\bf  arg0})
}%end signature
}%end item
\item{\vskip -1.5ex 
\texttt{public final void {\bf  setImpl\_caretPosition}(\texttt{int} {\bf  arg0})
}%end signature
}%end item
\item{\vskip -1.5ex 
\texttt{public final void {\bf  setImpl\_selectionEnd}(\texttt{int} {\bf  arg0})
}%end signature
}%end item
\item{\vskip -1.5ex 
\texttt{public final void {\bf  setImpl\_selectionStart}(\texttt{int} {\bf  arg0})
}%end signature
}%end item
\item{\vskip -1.5ex 
\texttt{public final void {\bf  setLineSpacing}(\texttt{double} {\bf  arg0})
}%end signature
}%end item
\item{\vskip -1.5ex 
\texttt{public final void {\bf  setStrikethrough}(\texttt{boolean} {\bf  arg0})
}%end signature
}%end item
\item{\vskip -1.5ex 
\texttt{public final void {\bf  setText}(\texttt{java.lang.String} {\bf  arg0})
}%end signature
}%end item
\item{\vskip -1.5ex 
\texttt{public final void {\bf  setTextAlignment}(\texttt{TextAlignment} {\bf  arg0})
}%end signature
}%end item
\item{\vskip -1.5ex 
\texttt{public final void {\bf  setTextOrigin}(\texttt{javafx.geometry.VPos} {\bf  arg0})
}%end signature
}%end item
\item{\vskip -1.5ex 
\texttt{public final void {\bf  setUnderline}(\texttt{boolean} {\bf  arg0})
}%end signature
}%end item
\item{\vskip -1.5ex 
\texttt{public final void {\bf  setWrappingWidth}(\texttt{double} {\bf  arg0})
}%end signature
}%end item
\item{\vskip -1.5ex 
\texttt{public final void {\bf  setX}(\texttt{double} {\bf  arg0})
}%end signature
}%end item
\item{\vskip -1.5ex 
\texttt{public final void {\bf  setY}(\texttt{double} {\bf  arg0})
}%end signature
}%end item
\item{\vskip -1.5ex 
\texttt{public final BooleanProperty {\bf  strikethroughProperty}()
}%end signature
}%end item
\item{\vskip -1.5ex 
\texttt{public final ObjectProperty {\bf  textAlignmentProperty}()
}%end signature
}%end item
\item{\vskip -1.5ex 
\texttt{public final ObjectProperty {\bf  textOriginProperty}()
}%end signature
}%end item
\item{\vskip -1.5ex 
\texttt{public final StringProperty {\bf  textProperty}()
}%end signature
}%end item
\item{\vskip -1.5ex 
\texttt{public String {\bf  toString}()
}%end signature
}%end item
\item{\vskip -1.5ex 
\texttt{public final BooleanProperty {\bf  underlineProperty}()
}%end signature
}%end item
\item{\vskip -1.5ex 
\texttt{public boolean {\bf  usesMirroring}()
}%end signature
}%end item
\item{\vskip -1.5ex 
\texttt{public final DoubleProperty {\bf  wrappingWidthProperty}()
}%end signature
}%end item
\item{\vskip -1.5ex 
\texttt{public final DoubleProperty {\bf  xProperty}()
}%end signature
}%end item
\item{\vskip -1.5ex 
\texttt{public final DoubleProperty {\bf  yProperty}()
}%end signature
}%end item
\end{itemize}
}
\subsection{Members inherited from class Shape }{
\texttt{javafx.scene.shape.Shape} {\small 
\refdefined{javafx.scene.shape.Shape}}
{\small 

\vskip -2em
\begin{itemize}
\item{\vskip -1.5ex 
\texttt{public final ObjectProperty {\bf  fillProperty}()
}%end signature
}%end item
\item{\vskip -1.5ex 
\texttt{public static List {\bf  getClassCssMetaData}()
}%end signature
}%end item
\item{\vskip -1.5ex 
\texttt{public List {\bf  getCssMetaData}()
}%end signature
}%end item
\item{\vskip -1.5ex 
\texttt{public final Paint {\bf  getFill}()
}%end signature
}%end item
\item{\vskip -1.5ex 
\texttt{public final Paint {\bf  getStroke}()
}%end signature
}%end item
\item{\vskip -1.5ex 
\texttt{public final ObservableList {\bf  getStrokeDashArray}()
}%end signature
}%end item
\item{\vskip -1.5ex 
\texttt{public final double {\bf  getStrokeDashOffset}()
}%end signature
}%end item
\item{\vskip -1.5ex 
\texttt{public final StrokeLineCap {\bf  getStrokeLineCap}()
}%end signature
}%end item
\item{\vskip -1.5ex 
\texttt{public final StrokeLineJoin {\bf  getStrokeLineJoin}()
}%end signature
}%end item
\item{\vskip -1.5ex 
\texttt{public final double {\bf  getStrokeMiterLimit}()
}%end signature
}%end item
\item{\vskip -1.5ex 
\texttt{public final StrokeType {\bf  getStrokeType}()
}%end signature
}%end item
\item{\vskip -1.5ex 
\texttt{public final double {\bf  getStrokeWidth}()
}%end signature
}%end item
\item{\vskip -1.5ex 
\texttt{protected boolean {\bf  impl\_computeContains}(\texttt{double} {\bf  arg0},
\texttt{double} {\bf  arg1})
}%end signature
}%end item
\item{\vskip -1.5ex 
\texttt{public BaseBounds {\bf  impl\_computeGeomBounds}(\texttt{com.sun.javafx.geom.BaseBounds} {\bf  arg0},
\texttt{com.sun.javafx.geom.transform.BaseTransform} {\bf  arg1})
}%end signature
}%end item
\item{\vskip -1.5ex 
\texttt{public abstract Shape {\bf  impl\_configShape}()
}%end signature
}%end item
\item{\vskip -1.5ex 
\texttt{protected NGNode {\bf  impl\_createPeer}()
}%end signature
}%end item
\item{\vskip -1.5ex 
\texttt{protected Paint {\bf  impl\_cssGetFillInitialValue}()
}%end signature
}%end item
\item{\vskip -1.5ex 
\texttt{protected Paint {\bf  impl\_cssGetStrokeInitialValue}()
}%end signature
}%end item
\item{\vskip -1.5ex 
\texttt{protected void {\bf  impl\_markDirty}(\texttt{com.sun.javafx.scene.DirtyBits} {\bf  arg0})
}%end signature
}%end item
\item{\vskip -1.5ex 
\texttt{protected {\bf  impl\_mode}}%end signature
}%end item
\item{\vskip -1.5ex 
\texttt{public Object {\bf  impl\_processMXNode}(\texttt{com.sun.javafx.jmx.MXNodeAlgorithm} {\bf  arg0},
\texttt{com.sun.javafx.jmx.MXNodeAlgorithmContext} {\bf  arg1})
}%end signature
}%end item
\item{\vskip -1.5ex 
\texttt{public void {\bf  impl\_setShapeChangeListener}(\texttt{java.lang.Runnable} {\bf  arg0})
}%end signature
}%end item
\item{\vskip -1.5ex 
\texttt{public void {\bf  impl\_updatePeer}()
}%end signature
}%end item
\item{\vskip -1.5ex 
\texttt{public static Shape {\bf  intersect}(\texttt{Shape} {\bf  arg0},
\texttt{Shape} {\bf  arg1})
}%end signature
}%end item
\item{\vskip -1.5ex 
\texttt{public final boolean {\bf  isSmooth}()
}%end signature
}%end item
\item{\vskip -1.5ex 
\texttt{public final void {\bf  setFill}(\texttt{javafx.scene.paint.Paint} {\bf  arg0})
}%end signature
}%end item
\item{\vskip -1.5ex 
\texttt{public final void {\bf  setSmooth}(\texttt{boolean} {\bf  arg0})
}%end signature
}%end item
\item{\vskip -1.5ex 
\texttt{public final void {\bf  setStroke}(\texttt{javafx.scene.paint.Paint} {\bf  arg0})
}%end signature
}%end item
\item{\vskip -1.5ex 
\texttt{public final void {\bf  setStrokeDashOffset}(\texttt{double} {\bf  arg0})
}%end signature
}%end item
\item{\vskip -1.5ex 
\texttt{public final void {\bf  setStrokeLineCap}(\texttt{StrokeLineCap} {\bf  arg0})
}%end signature
}%end item
\item{\vskip -1.5ex 
\texttt{public final void {\bf  setStrokeLineJoin}(\texttt{StrokeLineJoin} {\bf  arg0})
}%end signature
}%end item
\item{\vskip -1.5ex 
\texttt{public final void {\bf  setStrokeMiterLimit}(\texttt{double} {\bf  arg0})
}%end signature
}%end item
\item{\vskip -1.5ex 
\texttt{public final void {\bf  setStrokeType}(\texttt{StrokeType} {\bf  arg0})
}%end signature
}%end item
\item{\vskip -1.5ex 
\texttt{public final void {\bf  setStrokeWidth}(\texttt{double} {\bf  arg0})
}%end signature
}%end item
\item{\vskip -1.5ex 
\texttt{public final BooleanProperty {\bf  smoothProperty}()
}%end signature
}%end item
\item{\vskip -1.5ex 
\texttt{public final DoubleProperty {\bf  strokeDashOffsetProperty}()
}%end signature
}%end item
\item{\vskip -1.5ex 
\texttt{public final ObjectProperty {\bf  strokeLineCapProperty}()
}%end signature
}%end item
\item{\vskip -1.5ex 
\texttt{public final ObjectProperty {\bf  strokeLineJoinProperty}()
}%end signature
}%end item
\item{\vskip -1.5ex 
\texttt{public final DoubleProperty {\bf  strokeMiterLimitProperty}()
}%end signature
}%end item
\item{\vskip -1.5ex 
\texttt{public final ObjectProperty {\bf  strokeProperty}()
}%end signature
}%end item
\item{\vskip -1.5ex 
\texttt{public final ObjectProperty {\bf  strokeTypeProperty}()
}%end signature
}%end item
\item{\vskip -1.5ex 
\texttt{public final DoubleProperty {\bf  strokeWidthProperty}()
}%end signature
}%end item
\item{\vskip -1.5ex 
\texttt{public static Shape {\bf  subtract}(\texttt{Shape} {\bf  arg0},
\texttt{Shape} {\bf  arg1})
}%end signature
}%end item
\item{\vskip -1.5ex 
\texttt{public static Shape {\bf  union}(\texttt{Shape} {\bf  arg0},
\texttt{Shape} {\bf  arg1})
}%end signature
}%end item
\end{itemize}
}
\subsection{Members inherited from class Node }{
\texttt{javafx.scene.Node} {\small 
\refdefined{javafx.scene.Node}}
{\small 

\vskip -2em
\begin{itemize}
\item{\vskip -1.5ex 
\texttt{public final ObjectProperty {\bf  accessibleHelpProperty}()
}%end signature
}%end item
\item{\vskip -1.5ex 
\texttt{public final ObjectProperty {\bf  accessibleRoleDescriptionProperty}()
}%end signature
}%end item
\item{\vskip -1.5ex 
\texttt{public final ObjectProperty {\bf  accessibleRoleProperty}()
}%end signature
}%end item
\item{\vskip -1.5ex 
\texttt{public final ObjectProperty {\bf  accessibleTextProperty}()
}%end signature
}%end item
\item{\vskip -1.5ex 
\texttt{public final void {\bf  addEventFilter}(\texttt{javafx.event.EventType} {\bf  arg0},
\texttt{javafx.event.EventHandler} {\bf  arg1})
}%end signature
}%end item
\item{\vskip -1.5ex 
\texttt{public final void {\bf  addEventHandler}(\texttt{javafx.event.EventType} {\bf  arg0},
\texttt{javafx.event.EventHandler} {\bf  arg1})
}%end signature
}%end item
\item{\vskip -1.5ex 
\texttt{public final void {\bf  applyCss}()
}%end signature
}%end item
\item{\vskip -1.5ex 
\texttt{public final void {\bf  autosize}()
}%end signature
}%end item
\item{\vskip -1.5ex 
\texttt{public static final {\bf  BASELINE\_OFFSET\_SAME\_AS\_HEIGHT}}%end signature
}%end item
\item{\vskip -1.5ex 
\texttt{public final ObjectProperty {\bf  blendModeProperty}()
}%end signature
}%end item
\item{\vskip -1.5ex 
\texttt{public final ReadOnlyObjectProperty {\bf  boundsInLocalProperty}()
}%end signature
}%end item
\item{\vskip -1.5ex 
\texttt{public final ReadOnlyObjectProperty {\bf  boundsInParentProperty}()
}%end signature
}%end item
\item{\vskip -1.5ex 
\texttt{public EventDispatchChain {\bf  buildEventDispatchChain}(\texttt{javafx.event.EventDispatchChain} {\bf  arg0})
}%end signature
}%end item
\item{\vskip -1.5ex 
\texttt{public final ObjectProperty {\bf  cacheHintProperty}()
}%end signature
}%end item
\item{\vskip -1.5ex 
\texttt{public final BooleanProperty {\bf  cacheProperty}()
}%end signature
}%end item
\item{\vskip -1.5ex 
\texttt{public final ObjectProperty {\bf  clipProperty}()
}%end signature
}%end item
\item{\vskip -1.5ex 
\texttt{public double {\bf  computeAreaInScreen}()
}%end signature
}%end item
\item{\vskip -1.5ex 
\texttt{public boolean {\bf  contains}(\texttt{double} {\bf  arg0},
\texttt{double} {\bf  arg1})
}%end signature
}%end item
\item{\vskip -1.5ex 
\texttt{public boolean {\bf  contains}(\texttt{javafx.geometry.Point2D} {\bf  arg0})
}%end signature
}%end item
\item{\vskip -1.5ex 
\texttt{protected boolean {\bf  containsBounds}(\texttt{double} {\bf  arg0},
\texttt{double} {\bf  arg1})
}%end signature
}%end item
\item{\vskip -1.5ex 
\texttt{public final ObjectProperty {\bf  cursorProperty}()
}%end signature
}%end item
\item{\vskip -1.5ex 
\texttt{public final ObjectProperty {\bf  depthTestProperty}()
}%end signature
}%end item
\item{\vskip -1.5ex 
\texttt{public final ReadOnlyBooleanProperty {\bf  disabledProperty}()
}%end signature
}%end item
\item{\vskip -1.5ex 
\texttt{public final BooleanProperty {\bf  disableProperty}()
}%end signature
}%end item
\item{\vskip -1.5ex 
\texttt{public final ReadOnlyObjectProperty {\bf  effectiveNodeOrientationProperty}()
}%end signature
}%end item
\item{\vskip -1.5ex 
\texttt{public final ObjectProperty {\bf  effectProperty}()
}%end signature
}%end item
\item{\vskip -1.5ex 
\texttt{public final ObjectProperty {\bf  eventDispatcherProperty}()
}%end signature
}%end item
\item{\vskip -1.5ex 
\texttt{public void {\bf  executeAccessibleAction}(\texttt{AccessibleAction} {\bf  arg0},
\texttt{java.lang.Object\lbrack \rbrack } {\bf  arg1})
}%end signature
}%end item
\item{\vskip -1.5ex 
\texttt{public final void {\bf  fireEvent}(\texttt{javafx.event.Event} {\bf  arg0})
}%end signature
}%end item
\item{\vskip -1.5ex 
\texttt{public final ReadOnlyBooleanProperty {\bf  focusedProperty}()
}%end signature
}%end item
\item{\vskip -1.5ex 
\texttt{public final BooleanProperty {\bf  focusTraversableProperty}()
}%end signature
}%end item
\item{\vskip -1.5ex 
\texttt{public final String {\bf  getAccessibleHelp}()
}%end signature
}%end item
\item{\vskip -1.5ex 
\texttt{public final AccessibleRole {\bf  getAccessibleRole}()
}%end signature
}%end item
\item{\vskip -1.5ex 
\texttt{public final String {\bf  getAccessibleRoleDescription}()
}%end signature
}%end item
\item{\vskip -1.5ex 
\texttt{public final String {\bf  getAccessibleText}()
}%end signature
}%end item
\item{\vskip -1.5ex 
\texttt{public double {\bf  getBaselineOffset}()
}%end signature
}%end item
\item{\vskip -1.5ex 
\texttt{public final BlendMode {\bf  getBlendMode}()
}%end signature
}%end item
\item{\vskip -1.5ex 
\texttt{public final Bounds {\bf  getBoundsInLocal}()
}%end signature
}%end item
\item{\vskip -1.5ex 
\texttt{public final Bounds {\bf  getBoundsInParent}()
}%end signature
}%end item
\item{\vskip -1.5ex 
\texttt{public final CacheHint {\bf  getCacheHint}()
}%end signature
}%end item
\item{\vskip -1.5ex 
\texttt{public static List {\bf  getClassCssMetaData}()
}%end signature
}%end item
\item{\vskip -1.5ex 
\texttt{public final Node {\bf  getClip}()
}%end signature
}%end item
\item{\vskip -1.5ex 
\texttt{public Orientation {\bf  getContentBias}()
}%end signature
}%end item
\item{\vskip -1.5ex 
\texttt{public List {\bf  getCssMetaData}()
}%end signature
}%end item
\item{\vskip -1.5ex 
\texttt{public final Cursor {\bf  getCursor}()
}%end signature
}%end item
\item{\vskip -1.5ex 
\texttt{public final DepthTest {\bf  getDepthTest}()
}%end signature
}%end item
\item{\vskip -1.5ex 
\texttt{public final Effect {\bf  getEffect}()
}%end signature
}%end item
\item{\vskip -1.5ex 
\texttt{public final NodeOrientation {\bf  getEffectiveNodeOrientation}()
}%end signature
}%end item
\item{\vskip -1.5ex 
\texttt{public final EventDispatcher {\bf  getEventDispatcher}()
}%end signature
}%end item
\item{\vskip -1.5ex 
\texttt{public final String {\bf  getId}()
}%end signature
}%end item
\item{\vskip -1.5ex 
\texttt{public final InputMethodRequests {\bf  getInputMethodRequests}()
}%end signature
}%end item
\item{\vskip -1.5ex 
\texttt{public final Bounds {\bf  getLayoutBounds}()
}%end signature
}%end item
\item{\vskip -1.5ex 
\texttt{public final double {\bf  getLayoutX}()
}%end signature
}%end item
\item{\vskip -1.5ex 
\texttt{public final double {\bf  getLayoutY}()
}%end signature
}%end item
\item{\vskip -1.5ex 
\texttt{public final Transform {\bf  getLocalToParentTransform}()
}%end signature
}%end item
\item{\vskip -1.5ex 
\texttt{public final Transform {\bf  getLocalToSceneTransform}()
}%end signature
}%end item
\item{\vskip -1.5ex 
\texttt{public final NodeOrientation {\bf  getNodeOrientation}()
}%end signature
}%end item
\item{\vskip -1.5ex 
\texttt{public final EventHandler {\bf  getOnContextMenuRequested}()
}%end signature
}%end item
\item{\vskip -1.5ex 
\texttt{public final EventHandler {\bf  getOnDragDetected}()
}%end signature
}%end item
\item{\vskip -1.5ex 
\texttt{public final EventHandler {\bf  getOnDragDone}()
}%end signature
}%end item
\item{\vskip -1.5ex 
\texttt{public final EventHandler {\bf  getOnDragDropped}()
}%end signature
}%end item
\item{\vskip -1.5ex 
\texttt{public final EventHandler {\bf  getOnDragEntered}()
}%end signature
}%end item
\item{\vskip -1.5ex 
\texttt{public final EventHandler {\bf  getOnDragExited}()
}%end signature
}%end item
\item{\vskip -1.5ex 
\texttt{public final EventHandler {\bf  getOnDragOver}()
}%end signature
}%end item
\item{\vskip -1.5ex 
\texttt{public final EventHandler {\bf  getOnInputMethodTextChanged}()
}%end signature
}%end item
\item{\vskip -1.5ex 
\texttt{public final EventHandler {\bf  getOnKeyPressed}()
}%end signature
}%end item
\item{\vskip -1.5ex 
\texttt{public final EventHandler {\bf  getOnKeyReleased}()
}%end signature
}%end item
\item{\vskip -1.5ex 
\texttt{public final EventHandler {\bf  getOnKeyTyped}()
}%end signature
}%end item
\item{\vskip -1.5ex 
\texttt{public final EventHandler {\bf  getOnMouseClicked}()
}%end signature
}%end item
\item{\vskip -1.5ex 
\texttt{public final EventHandler {\bf  getOnMouseDragEntered}()
}%end signature
}%end item
\item{\vskip -1.5ex 
\texttt{public final EventHandler {\bf  getOnMouseDragExited}()
}%end signature
}%end item
\item{\vskip -1.5ex 
\texttt{public final EventHandler {\bf  getOnMouseDragged}()
}%end signature
}%end item
\item{\vskip -1.5ex 
\texttt{public final EventHandler {\bf  getOnMouseDragOver}()
}%end signature
}%end item
\item{\vskip -1.5ex 
\texttt{public final EventHandler {\bf  getOnMouseDragReleased}()
}%end signature
}%end item
\item{\vskip -1.5ex 
\texttt{public final EventHandler {\bf  getOnMouseEntered}()
}%end signature
}%end item
\item{\vskip -1.5ex 
\texttt{public final EventHandler {\bf  getOnMouseExited}()
}%end signature
}%end item
\item{\vskip -1.5ex 
\texttt{public final EventHandler {\bf  getOnMouseMoved}()
}%end signature
}%end item
\item{\vskip -1.5ex 
\texttt{public final EventHandler {\bf  getOnMousePressed}()
}%end signature
}%end item
\item{\vskip -1.5ex 
\texttt{public final EventHandler {\bf  getOnMouseReleased}()
}%end signature
}%end item
\item{\vskip -1.5ex 
\texttt{public final EventHandler {\bf  getOnRotate}()
}%end signature
}%end item
\item{\vskip -1.5ex 
\texttt{public final EventHandler {\bf  getOnRotationFinished}()
}%end signature
}%end item
\item{\vskip -1.5ex 
\texttt{public final EventHandler {\bf  getOnRotationStarted}()
}%end signature
}%end item
\item{\vskip -1.5ex 
\texttt{public final EventHandler {\bf  getOnScroll}()
}%end signature
}%end item
\item{\vskip -1.5ex 
\texttt{public final EventHandler {\bf  getOnScrollFinished}()
}%end signature
}%end item
\item{\vskip -1.5ex 
\texttt{public final EventHandler {\bf  getOnScrollStarted}()
}%end signature
}%end item
\item{\vskip -1.5ex 
\texttt{public final EventHandler {\bf  getOnSwipeDown}()
}%end signature
}%end item
\item{\vskip -1.5ex 
\texttt{public final EventHandler {\bf  getOnSwipeLeft}()
}%end signature
}%end item
\item{\vskip -1.5ex 
\texttt{public final EventHandler {\bf  getOnSwipeRight}()
}%end signature
}%end item
\item{\vskip -1.5ex 
\texttt{public final EventHandler {\bf  getOnSwipeUp}()
}%end signature
}%end item
\item{\vskip -1.5ex 
\texttt{public final EventHandler {\bf  getOnTouchMoved}()
}%end signature
}%end item
\item{\vskip -1.5ex 
\texttt{public final EventHandler {\bf  getOnTouchPressed}()
}%end signature
}%end item
\item{\vskip -1.5ex 
\texttt{public final EventHandler {\bf  getOnTouchReleased}()
}%end signature
}%end item
\item{\vskip -1.5ex 
\texttt{public final EventHandler {\bf  getOnTouchStationary}()
}%end signature
}%end item
\item{\vskip -1.5ex 
\texttt{public final EventHandler {\bf  getOnZoom}()
}%end signature
}%end item
\item{\vskip -1.5ex 
\texttt{public final EventHandler {\bf  getOnZoomFinished}()
}%end signature
}%end item
\item{\vskip -1.5ex 
\texttt{public final EventHandler {\bf  getOnZoomStarted}()
}%end signature
}%end item
\item{\vskip -1.5ex 
\texttt{public final double {\bf  getOpacity}()
}%end signature
}%end item
\item{\vskip -1.5ex 
\texttt{public final Parent {\bf  getParent}()
}%end signature
}%end item
\item{\vskip -1.5ex 
\texttt{public final ObservableMap {\bf  getProperties}()
}%end signature
}%end item
\item{\vskip -1.5ex 
\texttt{public final ObservableSet {\bf  getPseudoClassStates}()
}%end signature
}%end item
\item{\vskip -1.5ex 
\texttt{public final double {\bf  getRotate}()
}%end signature
}%end item
\item{\vskip -1.5ex 
\texttt{public final Point3D {\bf  getRotationAxis}()
}%end signature
}%end item
\item{\vskip -1.5ex 
\texttt{public final double {\bf  getScaleX}()
}%end signature
}%end item
\item{\vskip -1.5ex 
\texttt{public final double {\bf  getScaleY}()
}%end signature
}%end item
\item{\vskip -1.5ex 
\texttt{public final double {\bf  getScaleZ}()
}%end signature
}%end item
\item{\vskip -1.5ex 
\texttt{public final Scene {\bf  getScene}()
}%end signature
}%end item
\item{\vskip -1.5ex 
\texttt{public final String {\bf  getStyle}()
}%end signature
}%end item
\item{\vskip -1.5ex 
\texttt{public Styleable {\bf  getStyleableParent}()
}%end signature
}%end item
\item{\vskip -1.5ex 
\texttt{public final ObservableList {\bf  getStyleClass}()
}%end signature
}%end item
\item{\vskip -1.5ex 
\texttt{public final ObservableList {\bf  getTransforms}()
}%end signature
}%end item
\item{\vskip -1.5ex 
\texttt{public final double {\bf  getTranslateX}()
}%end signature
}%end item
\item{\vskip -1.5ex 
\texttt{public final double {\bf  getTranslateY}()
}%end signature
}%end item
\item{\vskip -1.5ex 
\texttt{public final double {\bf  getTranslateZ}()
}%end signature
}%end item
\item{\vskip -1.5ex 
\texttt{public String {\bf  getTypeSelector}()
}%end signature
}%end item
\item{\vskip -1.5ex 
\texttt{public Object {\bf  getUserData}()
}%end signature
}%end item
\item{\vskip -1.5ex 
\texttt{public boolean {\bf  hasProperties}()
}%end signature
}%end item
\item{\vskip -1.5ex 
\texttt{public final ReadOnlyBooleanProperty {\bf  hoverProperty}()
}%end signature
}%end item
\item{\vskip -1.5ex 
\texttt{public final StringProperty {\bf  idProperty}()
}%end signature
}%end item
\item{\vskip -1.5ex 
\texttt{protected final void {\bf  impl\_clearDirty}(\texttt{com.sun.javafx.scene.DirtyBits} {\bf  arg0})
}%end signature
}%end item
\item{\vskip -1.5ex 
\texttt{protected abstract boolean {\bf  impl\_computeContains}(\texttt{double} {\bf  arg0},
\texttt{double} {\bf  arg1})
}%end signature
}%end item
\item{\vskip -1.5ex 
\texttt{public abstract BaseBounds {\bf  impl\_computeGeomBounds}(\texttt{com.sun.javafx.geom.BaseBounds} {\bf  arg0},
\texttt{com.sun.javafx.geom.transform.BaseTransform} {\bf  arg1})
}%end signature
}%end item
\item{\vskip -1.5ex 
\texttt{protected boolean {\bf  impl\_computeIntersects}(\texttt{com.sun.javafx.geom.PickRay} {\bf  arg0},
\texttt{com.sun.javafx.scene.input.PickResultChooser} {\bf  arg1})
}%end signature
}%end item
\item{\vskip -1.5ex 
\texttt{protected Bounds {\bf  impl\_computeLayoutBounds}()
}%end signature
}%end item
\item{\vskip -1.5ex 
\texttt{protected abstract NGNode {\bf  impl\_createPeer}()
}%end signature
}%end item
\item{\vskip -1.5ex 
\texttt{protected Cursor {\bf  impl\_cssGetCursorInitialValue}()
}%end signature
}%end item
\item{\vskip -1.5ex 
\texttt{protected Boolean {\bf  impl\_cssGetFocusTraversableInitialValue}()
}%end signature
}%end item
\item{\vskip -1.5ex 
\texttt{public Map {\bf  impl\_findStyles}(\texttt{java.util.Map} {\bf  arg0})
}%end signature
}%end item
\item{\vskip -1.5ex 
\texttt{protected void {\bf  impl\_geomChanged}()
}%end signature
}%end item
\item{\vskip -1.5ex 
\texttt{public final BaseTransform {\bf  impl\_getLeafTransform}()
}%end signature
}%end item
\item{\vskip -1.5ex 
\texttt{public static List {\bf  impl\_getMatchingStyles}(\texttt{javafx.css.CssMetaData} {\bf  arg0},
\texttt{javafx.css.Styleable} {\bf  arg1})
}%end signature
}%end item
\item{\vskip -1.5ex 
\texttt{public NGNode {\bf  impl\_getPeer}()
}%end signature
}%end item
\item{\vskip -1.5ex 
\texttt{public final double {\bf  impl\_getPivotX}()
}%end signature
}%end item
\item{\vskip -1.5ex 
\texttt{public final double {\bf  impl\_getPivotY}()
}%end signature
}%end item
\item{\vskip -1.5ex 
\texttt{public final double {\bf  impl\_getPivotZ}()
}%end signature
}%end item
\item{\vskip -1.5ex 
\texttt{public final ObservableMap {\bf  impl\_getStyleMap}()
}%end signature
}%end item
\item{\vskip -1.5ex 
\texttt{public boolean {\bf  impl\_hasTransforms}()
}%end signature
}%end item
\item{\vskip -1.5ex 
\texttt{protected final boolean {\bf  impl\_intersects}(\texttt{com.sun.javafx.geom.PickRay} {\bf  arg0},
\texttt{com.sun.javafx.scene.input.PickResultChooser} {\bf  arg1})
}%end signature
}%end item
\item{\vskip -1.5ex 
\texttt{protected final double {\bf  impl\_intersectsBounds}(\texttt{com.sun.javafx.geom.PickRay} {\bf  arg0})
}%end signature
}%end item
\item{\vskip -1.5ex 
\texttt{protected final boolean {\bf  impl\_isDirty}(\texttt{com.sun.javafx.scene.DirtyBits} {\bf  arg0})
}%end signature
}%end item
\item{\vskip -1.5ex 
\texttt{protected final boolean {\bf  impl\_isDirtyEmpty}()
}%end signature
}%end item
\item{\vskip -1.5ex 
\texttt{public final boolean {\bf  impl\_isShowMnemonics}()
}%end signature
}%end item
\item{\vskip -1.5ex 
\texttt{public final boolean {\bf  impl\_isTreeVisible}()
}%end signature
}%end item
\item{\vskip -1.5ex 
\texttt{protected final void {\bf  impl\_layoutBoundsChanged}()
}%end signature
}%end item
\item{\vskip -1.5ex 
\texttt{protected void {\bf  impl\_markDirty}(\texttt{com.sun.javafx.scene.DirtyBits} {\bf  arg0})
}%end signature
}%end item
\item{\vskip -1.5ex 
\texttt{protected void {\bf  impl\_notifyLayoutBoundsChanged}()
}%end signature
}%end item
\item{\vskip -1.5ex 
\texttt{public final void {\bf  impl\_pickNode}(\texttt{com.sun.javafx.geom.PickRay} {\bf  arg0},
\texttt{com.sun.javafx.scene.input.PickResultChooser} {\bf  arg1})
}%end signature
}%end item
\item{\vskip -1.5ex 
\texttt{protected void {\bf  impl\_pickNodeLocal}(\texttt{com.sun.javafx.geom.PickRay} {\bf  arg0},
\texttt{com.sun.javafx.scene.input.PickResultChooser} {\bf  arg1})
}%end signature
}%end item
\item{\vskip -1.5ex 
\texttt{public final void {\bf  impl\_processCSS}(\texttt{boolean} {\bf  arg0})
}%end signature
}%end item
\item{\vskip -1.5ex 
\texttt{protected void {\bf  impl\_processCSS}(\texttt{javafx.beans.value.WritableValue} {\bf  arg0})
}%end signature
}%end item
\item{\vskip -1.5ex 
\texttt{public abstract Object {\bf  impl\_processMXNode}(\texttt{com.sun.javafx.jmx.MXNodeAlgorithm} {\bf  arg0},
\texttt{com.sun.javafx.jmx.MXNodeAlgorithmContext} {\bf  arg1})
}%end signature
}%end item
\item{\vskip -1.5ex 
\texttt{public final void {\bf  impl\_reapplyCSS}()
}%end signature
}%end item
\item{\vskip -1.5ex 
\texttt{public final void {\bf  impl\_setShowMnemonics}(\texttt{boolean} {\bf  arg0})
}%end signature
}%end item
\item{\vskip -1.5ex 
\texttt{public final void {\bf  impl\_setStyleMap}(\texttt{javafx.collections.ObservableMap} {\bf  arg0})
}%end signature
}%end item
\item{\vskip -1.5ex 
\texttt{public final BooleanProperty {\bf  impl\_showMnemonicsProperty}()
}%end signature
}%end item
\item{\vskip -1.5ex 
\texttt{public final void {\bf  impl\_syncPeer}()
}%end signature
}%end item
\item{\vskip -1.5ex 
\texttt{public void {\bf  impl\_transformsChanged}()
}%end signature
}%end item
\item{\vskip -1.5ex 
\texttt{public final boolean {\bf  impl\_traverse}(\texttt{com.sun.javafx.scene.traversal.Direction} {\bf  arg0})
}%end signature
}%end item
\item{\vskip -1.5ex 
\texttt{protected final BooleanExpression {\bf  impl\_treeVisibleProperty}()
}%end signature
}%end item
\item{\vskip -1.5ex 
\texttt{public void {\bf  impl\_updatePeer}()
}%end signature
}%end item
\item{\vskip -1.5ex 
\texttt{public final ObjectProperty {\bf  inputMethodRequestsProperty}()
}%end signature
}%end item
\item{\vskip -1.5ex 
\texttt{public boolean {\bf  intersects}(\texttt{javafx.geometry.Bounds} {\bf  arg0})
}%end signature
}%end item
\item{\vskip -1.5ex 
\texttt{public boolean {\bf  intersects}(\texttt{double} {\bf  arg0},
\texttt{double} {\bf  arg1},
\texttt{double} {\bf  arg2},
\texttt{double} {\bf  arg3})
}%end signature
}%end item
\item{\vskip -1.5ex 
\texttt{public final boolean {\bf  isCache}()
}%end signature
}%end item
\item{\vskip -1.5ex 
\texttt{public final boolean {\bf  isDisable}()
}%end signature
}%end item
\item{\vskip -1.5ex 
\texttt{public final boolean {\bf  isDisabled}()
}%end signature
}%end item
\item{\vskip -1.5ex 
\texttt{public final boolean {\bf  isFocused}()
}%end signature
}%end item
\item{\vskip -1.5ex 
\texttt{public final boolean {\bf  isFocusTraversable}()
}%end signature
}%end item
\item{\vskip -1.5ex 
\texttt{public final boolean {\bf  isHover}()
}%end signature
}%end item
\item{\vskip -1.5ex 
\texttt{public final boolean {\bf  isManaged}()
}%end signature
}%end item
\item{\vskip -1.5ex 
\texttt{public final boolean {\bf  isMouseTransparent}()
}%end signature
}%end item
\item{\vskip -1.5ex 
\texttt{public final boolean {\bf  isPickOnBounds}()
}%end signature
}%end item
\item{\vskip -1.5ex 
\texttt{public final boolean {\bf  isPressed}()
}%end signature
}%end item
\item{\vskip -1.5ex 
\texttt{public boolean {\bf  isResizable}()
}%end signature
}%end item
\item{\vskip -1.5ex 
\texttt{public final boolean {\bf  isVisible}()
}%end signature
}%end item
\item{\vskip -1.5ex 
\texttt{public final ReadOnlyObjectProperty {\bf  layoutBoundsProperty}()
}%end signature
}%end item
\item{\vskip -1.5ex 
\texttt{public final DoubleProperty {\bf  layoutXProperty}()
}%end signature
}%end item
\item{\vskip -1.5ex 
\texttt{public final DoubleProperty {\bf  layoutYProperty}()
}%end signature
}%end item
\item{\vskip -1.5ex 
\texttt{public Bounds {\bf  localToParent}(\texttt{javafx.geometry.Bounds} {\bf  arg0})
}%end signature
}%end item
\item{\vskip -1.5ex 
\texttt{public Point2D {\bf  localToParent}(\texttt{double} {\bf  arg0},
\texttt{double} {\bf  arg1})
}%end signature
}%end item
\item{\vskip -1.5ex 
\texttt{public Point3D {\bf  localToParent}(\texttt{double} {\bf  arg0},
\texttt{double} {\bf  arg1},
\texttt{double} {\bf  arg2})
}%end signature
}%end item
\item{\vskip -1.5ex 
\texttt{public Point2D {\bf  localToParent}(\texttt{javafx.geometry.Point2D} {\bf  arg0})
}%end signature
}%end item
\item{\vskip -1.5ex 
\texttt{public Point3D {\bf  localToParent}(\texttt{javafx.geometry.Point3D} {\bf  arg0})
}%end signature
}%end item
\item{\vskip -1.5ex 
\texttt{public final ReadOnlyObjectProperty {\bf  localToParentTransformProperty}()
}%end signature
}%end item
\item{\vskip -1.5ex 
\texttt{public Bounds {\bf  localToScene}(\texttt{javafx.geometry.Bounds} {\bf  arg0})
}%end signature
}%end item
\item{\vskip -1.5ex 
\texttt{public Bounds {\bf  localToScene}(\texttt{javafx.geometry.Bounds} {\bf  arg0},
\texttt{boolean} {\bf  arg1})
}%end signature
}%end item
\item{\vskip -1.5ex 
\texttt{public Point2D {\bf  localToScene}(\texttt{double} {\bf  arg0},
\texttt{double} {\bf  arg1})
}%end signature
}%end item
\item{\vskip -1.5ex 
\texttt{public Point2D {\bf  localToScene}(\texttt{double} {\bf  arg0},
\texttt{double} {\bf  arg1},
\texttt{boolean} {\bf  arg2})
}%end signature
}%end item
\item{\vskip -1.5ex 
\texttt{public Point3D {\bf  localToScene}(\texttt{double} {\bf  arg0},
\texttt{double} {\bf  arg1},
\texttt{double} {\bf  arg2})
}%end signature
}%end item
\item{\vskip -1.5ex 
\texttt{public Point3D {\bf  localToScene}(\texttt{double} {\bf  arg0},
\texttt{double} {\bf  arg1},
\texttt{double} {\bf  arg2},
\texttt{boolean} {\bf  arg3})
}%end signature
}%end item
\item{\vskip -1.5ex 
\texttt{public Point2D {\bf  localToScene}(\texttt{javafx.geometry.Point2D} {\bf  arg0})
}%end signature
}%end item
\item{\vskip -1.5ex 
\texttt{public Point2D {\bf  localToScene}(\texttt{javafx.geometry.Point2D} {\bf  arg0},
\texttt{boolean} {\bf  arg1})
}%end signature
}%end item
\item{\vskip -1.5ex 
\texttt{public Point3D {\bf  localToScene}(\texttt{javafx.geometry.Point3D} {\bf  arg0})
}%end signature
}%end item
\item{\vskip -1.5ex 
\texttt{public Point3D {\bf  localToScene}(\texttt{javafx.geometry.Point3D} {\bf  arg0},
\texttt{boolean} {\bf  arg1})
}%end signature
}%end item
\item{\vskip -1.5ex 
\texttt{public final ReadOnlyObjectProperty {\bf  localToSceneTransformProperty}()
}%end signature
}%end item
\item{\vskip -1.5ex 
\texttt{public Bounds {\bf  localToScreen}(\texttt{javafx.geometry.Bounds} {\bf  arg0})
}%end signature
}%end item
\item{\vskip -1.5ex 
\texttt{public Point2D {\bf  localToScreen}(\texttt{double} {\bf  arg0},
\texttt{double} {\bf  arg1})
}%end signature
}%end item
\item{\vskip -1.5ex 
\texttt{public Point2D {\bf  localToScreen}(\texttt{double} {\bf  arg0},
\texttt{double} {\bf  arg1},
\texttt{double} {\bf  arg2})
}%end signature
}%end item
\item{\vskip -1.5ex 
\texttt{public Point2D {\bf  localToScreen}(\texttt{javafx.geometry.Point2D} {\bf  arg0})
}%end signature
}%end item
\item{\vskip -1.5ex 
\texttt{public Point2D {\bf  localToScreen}(\texttt{javafx.geometry.Point3D} {\bf  arg0})
}%end signature
}%end item
\item{\vskip -1.5ex 
\texttt{public Node {\bf  lookup}(\texttt{java.lang.String} {\bf  arg0})
}%end signature
}%end item
\item{\vskip -1.5ex 
\texttt{public Set {\bf  lookupAll}(\texttt{java.lang.String} {\bf  arg0})
}%end signature
}%end item
\item{\vskip -1.5ex 
\texttt{public final BooleanProperty {\bf  managedProperty}()
}%end signature
}%end item
\item{\vskip -1.5ex 
\texttt{public double {\bf  maxHeight}(\texttt{double} {\bf  arg0})
}%end signature
}%end item
\item{\vskip -1.5ex 
\texttt{public double {\bf  maxWidth}(\texttt{double} {\bf  arg0})
}%end signature
}%end item
\item{\vskip -1.5ex 
\texttt{public double {\bf  minHeight}(\texttt{double} {\bf  arg0})
}%end signature
}%end item
\item{\vskip -1.5ex 
\texttt{public double {\bf  minWidth}(\texttt{double} {\bf  arg0})
}%end signature
}%end item
\item{\vskip -1.5ex 
\texttt{public final BooleanProperty {\bf  mouseTransparentProperty}()
}%end signature
}%end item
\item{\vskip -1.5ex 
\texttt{public final ObjectProperty {\bf  nodeOrientationProperty}()
}%end signature
}%end item
\item{\vskip -1.5ex 
\texttt{public final void {\bf  notifyAccessibleAttributeChanged}(\texttt{AccessibleAttribute} {\bf  arg0})
}%end signature
}%end item
\item{\vskip -1.5ex 
\texttt{public final ObjectProperty {\bf  onContextMenuRequestedProperty}()
}%end signature
}%end item
\item{\vskip -1.5ex 
\texttt{public final ObjectProperty {\bf  onDragDetectedProperty}()
}%end signature
}%end item
\item{\vskip -1.5ex 
\texttt{public final ObjectProperty {\bf  onDragDoneProperty}()
}%end signature
}%end item
\item{\vskip -1.5ex 
\texttt{public final ObjectProperty {\bf  onDragDroppedProperty}()
}%end signature
}%end item
\item{\vskip -1.5ex 
\texttt{public final ObjectProperty {\bf  onDragEnteredProperty}()
}%end signature
}%end item
\item{\vskip -1.5ex 
\texttt{public final ObjectProperty {\bf  onDragExitedProperty}()
}%end signature
}%end item
\item{\vskip -1.5ex 
\texttt{public final ObjectProperty {\bf  onDragOverProperty}()
}%end signature
}%end item
\item{\vskip -1.5ex 
\texttt{public final ObjectProperty {\bf  onInputMethodTextChangedProperty}()
}%end signature
}%end item
\item{\vskip -1.5ex 
\texttt{public final ObjectProperty {\bf  onKeyPressedProperty}()
}%end signature
}%end item
\item{\vskip -1.5ex 
\texttt{public final ObjectProperty {\bf  onKeyReleasedProperty}()
}%end signature
}%end item
\item{\vskip -1.5ex 
\texttt{public final ObjectProperty {\bf  onKeyTypedProperty}()
}%end signature
}%end item
\item{\vskip -1.5ex 
\texttt{public final ObjectProperty {\bf  onMouseClickedProperty}()
}%end signature
}%end item
\item{\vskip -1.5ex 
\texttt{public final ObjectProperty {\bf  onMouseDragEnteredProperty}()
}%end signature
}%end item
\item{\vskip -1.5ex 
\texttt{public final ObjectProperty {\bf  onMouseDragExitedProperty}()
}%end signature
}%end item
\item{\vskip -1.5ex 
\texttt{public final ObjectProperty {\bf  onMouseDraggedProperty}()
}%end signature
}%end item
\item{\vskip -1.5ex 
\texttt{public final ObjectProperty {\bf  onMouseDragOverProperty}()
}%end signature
}%end item
\item{\vskip -1.5ex 
\texttt{public final ObjectProperty {\bf  onMouseDragReleasedProperty}()
}%end signature
}%end item
\item{\vskip -1.5ex 
\texttt{public final ObjectProperty {\bf  onMouseEnteredProperty}()
}%end signature
}%end item
\item{\vskip -1.5ex 
\texttt{public final ObjectProperty {\bf  onMouseExitedProperty}()
}%end signature
}%end item
\item{\vskip -1.5ex 
\texttt{public final ObjectProperty {\bf  onMouseMovedProperty}()
}%end signature
}%end item
\item{\vskip -1.5ex 
\texttt{public final ObjectProperty {\bf  onMousePressedProperty}()
}%end signature
}%end item
\item{\vskip -1.5ex 
\texttt{public final ObjectProperty {\bf  onMouseReleasedProperty}()
}%end signature
}%end item
\item{\vskip -1.5ex 
\texttt{public final ObjectProperty {\bf  onRotateProperty}()
}%end signature
}%end item
\item{\vskip -1.5ex 
\texttt{public final ObjectProperty {\bf  onRotationFinishedProperty}()
}%end signature
}%end item
\item{\vskip -1.5ex 
\texttt{public final ObjectProperty {\bf  onRotationStartedProperty}()
}%end signature
}%end item
\item{\vskip -1.5ex 
\texttt{public final ObjectProperty {\bf  onScrollFinishedProperty}()
}%end signature
}%end item
\item{\vskip -1.5ex 
\texttt{public final ObjectProperty {\bf  onScrollProperty}()
}%end signature
}%end item
\item{\vskip -1.5ex 
\texttt{public final ObjectProperty {\bf  onScrollStartedProperty}()
}%end signature
}%end item
\item{\vskip -1.5ex 
\texttt{public final ObjectProperty {\bf  onSwipeDownProperty}()
}%end signature
}%end item
\item{\vskip -1.5ex 
\texttt{public final ObjectProperty {\bf  onSwipeLeftProperty}()
}%end signature
}%end item
\item{\vskip -1.5ex 
\texttt{public final ObjectProperty {\bf  onSwipeRightProperty}()
}%end signature
}%end item
\item{\vskip -1.5ex 
\texttt{public final ObjectProperty {\bf  onSwipeUpProperty}()
}%end signature
}%end item
\item{\vskip -1.5ex 
\texttt{public final ObjectProperty {\bf  onTouchMovedProperty}()
}%end signature
}%end item
\item{\vskip -1.5ex 
\texttt{public final ObjectProperty {\bf  onTouchPressedProperty}()
}%end signature
}%end item
\item{\vskip -1.5ex 
\texttt{public final ObjectProperty {\bf  onTouchReleasedProperty}()
}%end signature
}%end item
\item{\vskip -1.5ex 
\texttt{public final ObjectProperty {\bf  onTouchStationaryProperty}()
}%end signature
}%end item
\item{\vskip -1.5ex 
\texttt{public final ObjectProperty {\bf  onZoomFinishedProperty}()
}%end signature
}%end item
\item{\vskip -1.5ex 
\texttt{public final ObjectProperty {\bf  onZoomProperty}()
}%end signature
}%end item
\item{\vskip -1.5ex 
\texttt{public final ObjectProperty {\bf  onZoomStartedProperty}()
}%end signature
}%end item
\item{\vskip -1.5ex 
\texttt{public final DoubleProperty {\bf  opacityProperty}()
}%end signature
}%end item
\item{\vskip -1.5ex 
\texttt{public final ReadOnlyObjectProperty {\bf  parentProperty}()
}%end signature
}%end item
\item{\vskip -1.5ex 
\texttt{public Bounds {\bf  parentToLocal}(\texttt{javafx.geometry.Bounds} {\bf  arg0})
}%end signature
}%end item
\item{\vskip -1.5ex 
\texttt{public Point2D {\bf  parentToLocal}(\texttt{double} {\bf  arg0},
\texttt{double} {\bf  arg1})
}%end signature
}%end item
\item{\vskip -1.5ex 
\texttt{public Point3D {\bf  parentToLocal}(\texttt{double} {\bf  arg0},
\texttt{double} {\bf  arg1},
\texttt{double} {\bf  arg2})
}%end signature
}%end item
\item{\vskip -1.5ex 
\texttt{public Point2D {\bf  parentToLocal}(\texttt{javafx.geometry.Point2D} {\bf  arg0})
}%end signature
}%end item
\item{\vskip -1.5ex 
\texttt{public Point3D {\bf  parentToLocal}(\texttt{javafx.geometry.Point3D} {\bf  arg0})
}%end signature
}%end item
\item{\vskip -1.5ex 
\texttt{public final BooleanProperty {\bf  pickOnBoundsProperty}()
}%end signature
}%end item
\item{\vskip -1.5ex 
\texttt{public double {\bf  prefHeight}(\texttt{double} {\bf  arg0})
}%end signature
}%end item
\item{\vskip -1.5ex 
\texttt{public double {\bf  prefWidth}(\texttt{double} {\bf  arg0})
}%end signature
}%end item
\item{\vskip -1.5ex 
\texttt{public final ReadOnlyBooleanProperty {\bf  pressedProperty}()
}%end signature
}%end item
\item{\vskip -1.5ex 
\texttt{public final void {\bf  pseudoClassStateChanged}(\texttt{javafx.css.PseudoClass} {\bf  arg0},
\texttt{boolean} {\bf  arg1})
}%end signature
}%end item
\item{\vskip -1.5ex 
\texttt{public Object {\bf  queryAccessibleAttribute}(\texttt{AccessibleAttribute} {\bf  arg0},
\texttt{java.lang.Object\lbrack \rbrack } {\bf  arg1})
}%end signature
}%end item
\item{\vskip -1.5ex 
\texttt{public void {\bf  relocate}(\texttt{double} {\bf  arg0},
\texttt{double} {\bf  arg1})
}%end signature
}%end item
\item{\vskip -1.5ex 
\texttt{public final void {\bf  removeEventFilter}(\texttt{javafx.event.EventType} {\bf  arg0},
\texttt{javafx.event.EventHandler} {\bf  arg1})
}%end signature
}%end item
\item{\vskip -1.5ex 
\texttt{public final void {\bf  removeEventHandler}(\texttt{javafx.event.EventType} {\bf  arg0},
\texttt{javafx.event.EventHandler} {\bf  arg1})
}%end signature
}%end item
\item{\vskip -1.5ex 
\texttt{public void {\bf  requestFocus}()
}%end signature
}%end item
\item{\vskip -1.5ex 
\texttt{public void {\bf  resize}(\texttt{double} {\bf  arg0},
\texttt{double} {\bf  arg1})
}%end signature
}%end item
\item{\vskip -1.5ex 
\texttt{public void {\bf  resizeRelocate}(\texttt{double} {\bf  arg0},
\texttt{double} {\bf  arg1},
\texttt{double} {\bf  arg2},
\texttt{double} {\bf  arg3})
}%end signature
}%end item
\item{\vskip -1.5ex 
\texttt{public final DoubleProperty {\bf  rotateProperty}()
}%end signature
}%end item
\item{\vskip -1.5ex 
\texttt{public final ObjectProperty {\bf  rotationAxisProperty}()
}%end signature
}%end item
\item{\vskip -1.5ex 
\texttt{public final DoubleProperty {\bf  scaleXProperty}()
}%end signature
}%end item
\item{\vskip -1.5ex 
\texttt{public final DoubleProperty {\bf  scaleYProperty}()
}%end signature
}%end item
\item{\vskip -1.5ex 
\texttt{public final DoubleProperty {\bf  scaleZProperty}()
}%end signature
}%end item
\item{\vskip -1.5ex 
\texttt{public final ReadOnlyObjectProperty {\bf  sceneProperty}()
}%end signature
}%end item
\item{\vskip -1.5ex 
\texttt{public Bounds {\bf  sceneToLocal}(\texttt{javafx.geometry.Bounds} {\bf  arg0})
}%end signature
}%end item
\item{\vskip -1.5ex 
\texttt{public Bounds {\bf  sceneToLocal}(\texttt{javafx.geometry.Bounds} {\bf  arg0},
\texttt{boolean} {\bf  arg1})
}%end signature
}%end item
\item{\vskip -1.5ex 
\texttt{public Point2D {\bf  sceneToLocal}(\texttt{double} {\bf  arg0},
\texttt{double} {\bf  arg1})
}%end signature
}%end item
\item{\vskip -1.5ex 
\texttt{public Point2D {\bf  sceneToLocal}(\texttt{double} {\bf  arg0},
\texttt{double} {\bf  arg1},
\texttt{boolean} {\bf  arg2})
}%end signature
}%end item
\item{\vskip -1.5ex 
\texttt{public Point3D {\bf  sceneToLocal}(\texttt{double} {\bf  arg0},
\texttt{double} {\bf  arg1},
\texttt{double} {\bf  arg2})
}%end signature
}%end item
\item{\vskip -1.5ex 
\texttt{public Point2D {\bf  sceneToLocal}(\texttt{javafx.geometry.Point2D} {\bf  arg0})
}%end signature
}%end item
\item{\vskip -1.5ex 
\texttt{public Point2D {\bf  sceneToLocal}(\texttt{javafx.geometry.Point2D} {\bf  arg0},
\texttt{boolean} {\bf  arg1})
}%end signature
}%end item
\item{\vskip -1.5ex 
\texttt{public Point3D {\bf  sceneToLocal}(\texttt{javafx.geometry.Point3D} {\bf  arg0})
}%end signature
}%end item
\item{\vskip -1.5ex 
\texttt{public Bounds {\bf  screenToLocal}(\texttt{javafx.geometry.Bounds} {\bf  arg0})
}%end signature
}%end item
\item{\vskip -1.5ex 
\texttt{public Point2D {\bf  screenToLocal}(\texttt{double} {\bf  arg0},
\texttt{double} {\bf  arg1})
}%end signature
}%end item
\item{\vskip -1.5ex 
\texttt{public Point2D {\bf  screenToLocal}(\texttt{javafx.geometry.Point2D} {\bf  arg0})
}%end signature
}%end item
\item{\vskip -1.5ex 
\texttt{public final void {\bf  setAccessibleHelp}(\texttt{java.lang.String} {\bf  arg0})
}%end signature
}%end item
\item{\vskip -1.5ex 
\texttt{public final void {\bf  setAccessibleRole}(\texttt{AccessibleRole} {\bf  arg0})
}%end signature
}%end item
\item{\vskip -1.5ex 
\texttt{public final void {\bf  setAccessibleRoleDescription}(\texttt{java.lang.String} {\bf  arg0})
}%end signature
}%end item
\item{\vskip -1.5ex 
\texttt{public final void {\bf  setAccessibleText}(\texttt{java.lang.String} {\bf  arg0})
}%end signature
}%end item
\item{\vskip -1.5ex 
\texttt{public final void {\bf  setBlendMode}(\texttt{effect.BlendMode} {\bf  arg0})
}%end signature
}%end item
\item{\vskip -1.5ex 
\texttt{public final void {\bf  setCache}(\texttt{boolean} {\bf  arg0})
}%end signature
}%end item
\item{\vskip -1.5ex 
\texttt{public final void {\bf  setCacheHint}(\texttt{CacheHint} {\bf  arg0})
}%end signature
}%end item
\item{\vskip -1.5ex 
\texttt{public final void {\bf  setClip}(\texttt{Node} {\bf  arg0})
}%end signature
}%end item
\item{\vskip -1.5ex 
\texttt{public final void {\bf  setCursor}(\texttt{Cursor} {\bf  arg0})
}%end signature
}%end item
\item{\vskip -1.5ex 
\texttt{public final void {\bf  setDepthTest}(\texttt{DepthTest} {\bf  arg0})
}%end signature
}%end item
\item{\vskip -1.5ex 
\texttt{public final void {\bf  setDisable}(\texttt{boolean} {\bf  arg0})
}%end signature
}%end item
\item{\vskip -1.5ex 
\texttt{protected final void {\bf  setDisabled}(\texttt{boolean} {\bf  arg0})
}%end signature
}%end item
\item{\vskip -1.5ex 
\texttt{public final void {\bf  setEffect}(\texttt{effect.Effect} {\bf  arg0})
}%end signature
}%end item
\item{\vskip -1.5ex 
\texttt{public final void {\bf  setEventDispatcher}(\texttt{javafx.event.EventDispatcher} {\bf  arg0})
}%end signature
}%end item
\item{\vskip -1.5ex 
\texttt{protected final void {\bf  setEventHandler}(\texttt{javafx.event.EventType} {\bf  arg0},
\texttt{javafx.event.EventHandler} {\bf  arg1})
}%end signature
}%end item
\item{\vskip -1.5ex 
\texttt{protected final void {\bf  setFocused}(\texttt{boolean} {\bf  arg0})
}%end signature
}%end item
\item{\vskip -1.5ex 
\texttt{public final void {\bf  setFocusTraversable}(\texttt{boolean} {\bf  arg0})
}%end signature
}%end item
\item{\vskip -1.5ex 
\texttt{protected final void {\bf  setHover}(\texttt{boolean} {\bf  arg0})
}%end signature
}%end item
\item{\vskip -1.5ex 
\texttt{public final void {\bf  setId}(\texttt{java.lang.String} {\bf  arg0})
}%end signature
}%end item
\item{\vskip -1.5ex 
\texttt{public final void {\bf  setInputMethodRequests}(\texttt{input.InputMethodRequests} {\bf  arg0})
}%end signature
}%end item
\item{\vskip -1.5ex 
\texttt{public final void {\bf  setLayoutX}(\texttt{double} {\bf  arg0})
}%end signature
}%end item
\item{\vskip -1.5ex 
\texttt{public final void {\bf  setLayoutY}(\texttt{double} {\bf  arg0})
}%end signature
}%end item
\item{\vskip -1.5ex 
\texttt{public final void {\bf  setManaged}(\texttt{boolean} {\bf  arg0})
}%end signature
}%end item
\item{\vskip -1.5ex 
\texttt{public final void {\bf  setMouseTransparent}(\texttt{boolean} {\bf  arg0})
}%end signature
}%end item
\item{\vskip -1.5ex 
\texttt{public final void {\bf  setNodeOrientation}(\texttt{javafx.geometry.NodeOrientation} {\bf  arg0})
}%end signature
}%end item
\item{\vskip -1.5ex 
\texttt{public final void {\bf  setOnContextMenuRequested}(\texttt{javafx.event.EventHandler} {\bf  arg0})
}%end signature
}%end item
\item{\vskip -1.5ex 
\texttt{public final void {\bf  setOnDragDetected}(\texttt{javafx.event.EventHandler} {\bf  arg0})
}%end signature
}%end item
\item{\vskip -1.5ex 
\texttt{public final void {\bf  setOnDragDone}(\texttt{javafx.event.EventHandler} {\bf  arg0})
}%end signature
}%end item
\item{\vskip -1.5ex 
\texttt{public final void {\bf  setOnDragDropped}(\texttt{javafx.event.EventHandler} {\bf  arg0})
}%end signature
}%end item
\item{\vskip -1.5ex 
\texttt{public final void {\bf  setOnDragEntered}(\texttt{javafx.event.EventHandler} {\bf  arg0})
}%end signature
}%end item
\item{\vskip -1.5ex 
\texttt{public final void {\bf  setOnDragExited}(\texttt{javafx.event.EventHandler} {\bf  arg0})
}%end signature
}%end item
\item{\vskip -1.5ex 
\texttt{public final void {\bf  setOnDragOver}(\texttt{javafx.event.EventHandler} {\bf  arg0})
}%end signature
}%end item
\item{\vskip -1.5ex 
\texttt{public final void {\bf  setOnInputMethodTextChanged}(\texttt{javafx.event.EventHandler} {\bf  arg0})
}%end signature
}%end item
\item{\vskip -1.5ex 
\texttt{public final void {\bf  setOnKeyPressed}(\texttt{javafx.event.EventHandler} {\bf  arg0})
}%end signature
}%end item
\item{\vskip -1.5ex 
\texttt{public final void {\bf  setOnKeyReleased}(\texttt{javafx.event.EventHandler} {\bf  arg0})
}%end signature
}%end item
\item{\vskip -1.5ex 
\texttt{public final void {\bf  setOnKeyTyped}(\texttt{javafx.event.EventHandler} {\bf  arg0})
}%end signature
}%end item
\item{\vskip -1.5ex 
\texttt{public final void {\bf  setOnMouseClicked}(\texttt{javafx.event.EventHandler} {\bf  arg0})
}%end signature
}%end item
\item{\vskip -1.5ex 
\texttt{public final void {\bf  setOnMouseDragEntered}(\texttt{javafx.event.EventHandler} {\bf  arg0})
}%end signature
}%end item
\item{\vskip -1.5ex 
\texttt{public final void {\bf  setOnMouseDragExited}(\texttt{javafx.event.EventHandler} {\bf  arg0})
}%end signature
}%end item
\item{\vskip -1.5ex 
\texttt{public final void {\bf  setOnMouseDragged}(\texttt{javafx.event.EventHandler} {\bf  arg0})
}%end signature
}%end item
\item{\vskip -1.5ex 
\texttt{public final void {\bf  setOnMouseDragOver}(\texttt{javafx.event.EventHandler} {\bf  arg0})
}%end signature
}%end item
\item{\vskip -1.5ex 
\texttt{public final void {\bf  setOnMouseDragReleased}(\texttt{javafx.event.EventHandler} {\bf  arg0})
}%end signature
}%end item
\item{\vskip -1.5ex 
\texttt{public final void {\bf  setOnMouseEntered}(\texttt{javafx.event.EventHandler} {\bf  arg0})
}%end signature
}%end item
\item{\vskip -1.5ex 
\texttt{public final void {\bf  setOnMouseExited}(\texttt{javafx.event.EventHandler} {\bf  arg0})
}%end signature
}%end item
\item{\vskip -1.5ex 
\texttt{public final void {\bf  setOnMouseMoved}(\texttt{javafx.event.EventHandler} {\bf  arg0})
}%end signature
}%end item
\item{\vskip -1.5ex 
\texttt{public final void {\bf  setOnMousePressed}(\texttt{javafx.event.EventHandler} {\bf  arg0})
}%end signature
}%end item
\item{\vskip -1.5ex 
\texttt{public final void {\bf  setOnMouseReleased}(\texttt{javafx.event.EventHandler} {\bf  arg0})
}%end signature
}%end item
\item{\vskip -1.5ex 
\texttt{public final void {\bf  setOnRotate}(\texttt{javafx.event.EventHandler} {\bf  arg0})
}%end signature
}%end item
\item{\vskip -1.5ex 
\texttt{public final void {\bf  setOnRotationFinished}(\texttt{javafx.event.EventHandler} {\bf  arg0})
}%end signature
}%end item
\item{\vskip -1.5ex 
\texttt{public final void {\bf  setOnRotationStarted}(\texttt{javafx.event.EventHandler} {\bf  arg0})
}%end signature
}%end item
\item{\vskip -1.5ex 
\texttt{public final void {\bf  setOnScroll}(\texttt{javafx.event.EventHandler} {\bf  arg0})
}%end signature
}%end item
\item{\vskip -1.5ex 
\texttt{public final void {\bf  setOnScrollFinished}(\texttt{javafx.event.EventHandler} {\bf  arg0})
}%end signature
}%end item
\item{\vskip -1.5ex 
\texttt{public final void {\bf  setOnScrollStarted}(\texttt{javafx.event.EventHandler} {\bf  arg0})
}%end signature
}%end item
\item{\vskip -1.5ex 
\texttt{public final void {\bf  setOnSwipeDown}(\texttt{javafx.event.EventHandler} {\bf  arg0})
}%end signature
}%end item
\item{\vskip -1.5ex 
\texttt{public final void {\bf  setOnSwipeLeft}(\texttt{javafx.event.EventHandler} {\bf  arg0})
}%end signature
}%end item
\item{\vskip -1.5ex 
\texttt{public final void {\bf  setOnSwipeRight}(\texttt{javafx.event.EventHandler} {\bf  arg0})
}%end signature
}%end item
\item{\vskip -1.5ex 
\texttt{public final void {\bf  setOnSwipeUp}(\texttt{javafx.event.EventHandler} {\bf  arg0})
}%end signature
}%end item
\item{\vskip -1.5ex 
\texttt{public final void {\bf  setOnTouchMoved}(\texttt{javafx.event.EventHandler} {\bf  arg0})
}%end signature
}%end item
\item{\vskip -1.5ex 
\texttt{public final void {\bf  setOnTouchPressed}(\texttt{javafx.event.EventHandler} {\bf  arg0})
}%end signature
}%end item
\item{\vskip -1.5ex 
\texttt{public final void {\bf  setOnTouchReleased}(\texttt{javafx.event.EventHandler} {\bf  arg0})
}%end signature
}%end item
\item{\vskip -1.5ex 
\texttt{public final void {\bf  setOnTouchStationary}(\texttt{javafx.event.EventHandler} {\bf  arg0})
}%end signature
}%end item
\item{\vskip -1.5ex 
\texttt{public final void {\bf  setOnZoom}(\texttt{javafx.event.EventHandler} {\bf  arg0})
}%end signature
}%end item
\item{\vskip -1.5ex 
\texttt{public final void {\bf  setOnZoomFinished}(\texttt{javafx.event.EventHandler} {\bf  arg0})
}%end signature
}%end item
\item{\vskip -1.5ex 
\texttt{public final void {\bf  setOnZoomStarted}(\texttt{javafx.event.EventHandler} {\bf  arg0})
}%end signature
}%end item
\item{\vskip -1.5ex 
\texttt{public final void {\bf  setOpacity}(\texttt{double} {\bf  arg0})
}%end signature
}%end item
\item{\vskip -1.5ex 
\texttt{public final void {\bf  setPickOnBounds}(\texttt{boolean} {\bf  arg0})
}%end signature
}%end item
\item{\vskip -1.5ex 
\texttt{protected final void {\bf  setPressed}(\texttt{boolean} {\bf  arg0})
}%end signature
}%end item
\item{\vskip -1.5ex 
\texttt{public final void {\bf  setRotate}(\texttt{double} {\bf  arg0})
}%end signature
}%end item
\item{\vskip -1.5ex 
\texttt{public final void {\bf  setRotationAxis}(\texttt{javafx.geometry.Point3D} {\bf  arg0})
}%end signature
}%end item
\item{\vskip -1.5ex 
\texttt{public final void {\bf  setScaleX}(\texttt{double} {\bf  arg0})
}%end signature
}%end item
\item{\vskip -1.5ex 
\texttt{public final void {\bf  setScaleY}(\texttt{double} {\bf  arg0})
}%end signature
}%end item
\item{\vskip -1.5ex 
\texttt{public final void {\bf  setScaleZ}(\texttt{double} {\bf  arg0})
}%end signature
}%end item
\item{\vskip -1.5ex 
\texttt{public final void {\bf  setStyle}(\texttt{java.lang.String} {\bf  arg0})
}%end signature
}%end item
\item{\vskip -1.5ex 
\texttt{public final void {\bf  setTranslateX}(\texttt{double} {\bf  arg0})
}%end signature
}%end item
\item{\vskip -1.5ex 
\texttt{public final void {\bf  setTranslateY}(\texttt{double} {\bf  arg0})
}%end signature
}%end item
\item{\vskip -1.5ex 
\texttt{public final void {\bf  setTranslateZ}(\texttt{double} {\bf  arg0})
}%end signature
}%end item
\item{\vskip -1.5ex 
\texttt{public void {\bf  setUserData}(\texttt{java.lang.Object} {\bf  arg0})
}%end signature
}%end item
\item{\vskip -1.5ex 
\texttt{public final void {\bf  setVisible}(\texttt{boolean} {\bf  arg0})
}%end signature
}%end item
\item{\vskip -1.5ex 
\texttt{public void {\bf  snapshot}(\texttt{javafx.util.Callback} {\bf  arg0},
\texttt{SnapshotParameters} {\bf  arg1},
\texttt{image.WritableImage} {\bf  arg2})
}%end signature
}%end item
\item{\vskip -1.5ex 
\texttt{public WritableImage {\bf  snapshot}(\texttt{SnapshotParameters} {\bf  arg0},
\texttt{image.WritableImage} {\bf  arg1})
}%end signature
}%end item
\item{\vskip -1.5ex 
\texttt{public Dragboard {\bf  startDragAndDrop}(\texttt{input.TransferMode\lbrack \rbrack } {\bf  arg0})
}%end signature
}%end item
\item{\vskip -1.5ex 
\texttt{public void {\bf  startFullDrag}()
}%end signature
}%end item
\item{\vskip -1.5ex 
\texttt{public final StringProperty {\bf  styleProperty}()
}%end signature
}%end item
\item{\vskip -1.5ex 
\texttt{public void {\bf  toBack}()
}%end signature
}%end item
\item{\vskip -1.5ex 
\texttt{public void {\bf  toFront}()
}%end signature
}%end item
\item{\vskip -1.5ex 
\texttt{public String {\bf  toString}()
}%end signature
}%end item
\item{\vskip -1.5ex 
\texttt{public final DoubleProperty {\bf  translateXProperty}()
}%end signature
}%end item
\item{\vskip -1.5ex 
\texttt{public final DoubleProperty {\bf  translateYProperty}()
}%end signature
}%end item
\item{\vskip -1.5ex 
\texttt{public final DoubleProperty {\bf  translateZProperty}()
}%end signature
}%end item
\item{\vskip -1.5ex 
\texttt{public boolean {\bf  usesMirroring}()
}%end signature
}%end item
\item{\vskip -1.5ex 
\texttt{public final BooleanProperty {\bf  visibleProperty}()
}%end signature
}%end item
\end{itemize}
}
}
\section{\label{vue.VueGraphiqueAideur}Class VueGraphiqueAideur}{
\hypertarget{vue.VueGraphiqueAideur}{}\vskip .1in 
Contient les méthodes permettant d'afficher les éléments dans la zone graphique\vskip .1in 
\subsection{Declaration}{
\begin{lstlisting}[frame=none]
public class VueGraphiqueAideur
 extends java.lang.Object\end{lstlisting}
\subsection{Constructor summary}{
\begin{verse}
\hyperlink{vue.VueGraphiqueAideur(javafx.scene.layout.StackPane, javafx.scene.Group, javafx.scene.control.ScrollPane, javafx.scene.control.Slider)}{{\bf VueGraphiqueAideur(StackPane, Group, ScrollPane, Slider)}} Constructeur de la vue graphique\\
\end{verse}
}
\subsection{Method summary}{
\begin{verse}
\hyperlink{vue.VueGraphiqueAideur.afficherDemande()}{{\bf afficherDemande()}} Affiche les livraisons contenues dans une demande de livraison\\
\hyperlink{vue.VueGraphiqueAideur.afficherPlan()}{{\bf afficherPlan()}} Affiche tous les points du plan et met à jour la taille du canvas graphique.\\
\hyperlink{vue.VueGraphiqueAideur.afficherTournee()}{{\bf afficherTournee()}} Affiche la tournée (les tronçons empreintés entre les lieux de livraisons)\\
\hyperlink{vue.VueGraphiqueAideur.construireDemande(modele.donneesxml.Demande)}{{\bf construireDemande(Demande)}} Stocke la demande\\
\hyperlink{vue.VueGraphiqueAideur.construireGraphe(modele.donneesxml.PlanDeVille)}{{\bf construireGraphe(PlanDeVille)}} Construit et affiche le graphe du plan de la ville sur le canvas graphique de la fenêtre\\
\hyperlink{vue.VueGraphiqueAideur.construireTournee(java.util.List)}{{\bf construireTournee(List)}} Construit et affiche la tournée\\
\hyperlink{vue.VueGraphiqueAideur.desactiverSurbrillance()}{{\bf desactiverSurbrillance()}} Désactive la surbrillance pour toutes les surbrillances\\
\hyperlink{vue.VueGraphiqueAideur.getCanvas()}{{\bf getCanvas()}} \\
\hyperlink{vue.VueGraphiqueAideur.nettoyerAffichage()}{{\bf nettoyerAffichage()}} Supprime tous les éléments sur la partie graphique\\
\hyperlink{vue.VueGraphiqueAideur.setControleurApplication(controleur.ControleurInterface)}{{\bf setControleurApplication(ControleurInterface)}} Met à jour le controleur de l'application pour la vue graphique\\
\hyperlink{vue.VueGraphiqueAideur.surbrillanceLivraison(modele.donneesxml.Livraison)}{{\bf surbrillanceLivraison(Livraison)}} Affiche une livraison en surbrillance sur la partie graphique\\
\hyperlink{vue.VueGraphiqueAideur.surbrillanceLivraisons(java.util.Collection)}{{\bf surbrillanceLivraisons(Collection)}} Mets plusieurs livraisons en surbrillance sur la partie graphique\\
\end{verse}
}
\subsection{Constructors}{
\vskip -2em
\begin{itemize}
\item{ 
\index{VueGraphiqueAideur(StackPane, Group, ScrollPane, Slider)}
\hypertarget{vue.VueGraphiqueAideur(javafx.scene.layout.StackPane, javafx.scene.Group, javafx.scene.control.ScrollPane, javafx.scene.control.Slider)}{{\bf  VueGraphiqueAideur}\\}
\begin{lstlisting}[frame=none]
public VueGraphiqueAideur(javafx.scene.layout.StackPane canvas,javafx.scene.Group group,javafx.scene.control.ScrollPane scrollPane,javafx.scene.control.Slider slider)\end{lstlisting} %end signature
\begin{itemize}
\item{
{\bf  Description}

Constructeur de la vue graphique
}
\item{
{\bf  Parameters}
  \begin{itemize}
   \item{
\texttt{canvas} -- Le canvas sur lequel on dessinera les éléments graphiques}
   \item{
\texttt{group} -- Le group de la partie graphique}
   \item{
\texttt{scrollPane} -- La barre de défilement}
   \item{
\texttt{slider} -- Le slide de zoom}
  \end{itemize}
}%end item
\end{itemize}
}%end item
\end{itemize}
}
\subsection{Methods}{
\vskip -2em
\begin{itemize}
\item{ 
\index{afficherDemande()}
\hypertarget{vue.VueGraphiqueAideur.afficherDemande()}{{\bf  afficherDemande}\\}
\begin{lstlisting}[frame=none]
public void afficherDemande()\end{lstlisting} %end signature
\begin{itemize}
\item{
{\bf  Description}

Affiche les livraisons contenues dans une demande de livraison
}
\end{itemize}
}%end item
\item{ 
\index{afficherPlan()}
\hypertarget{vue.VueGraphiqueAideur.afficherPlan()}{{\bf  afficherPlan}\\}
\begin{lstlisting}[frame=none]
public void afficherPlan()\end{lstlisting} %end signature
\begin{itemize}
\item{
{\bf  Description}

Affiche tous les points du plan et met à jour la taille du canvas graphique. Les points sont toujours affichés par rapport : (leur taille initiale dans le fichier XML / la plus grande taille dans le fichier XML) =\textgreater  (la nouvelle taille / la taille du canvas)
}
\end{itemize}
}%end item
\item{ 
\index{afficherTournee()}
\hypertarget{vue.VueGraphiqueAideur.afficherTournee()}{{\bf  afficherTournee}\\}
\begin{lstlisting}[frame=none]
public void afficherTournee()\end{lstlisting} %end signature
\begin{itemize}
\item{
{\bf  Description}

Affiche la tournée (les tronçons empreintés entre les lieux de livraisons)
}
\end{itemize}
}%end item
\item{ 
\index{construireDemande(Demande)}
\hypertarget{vue.VueGraphiqueAideur.construireDemande(modele.donneesxml.Demande)}{{\bf  construireDemande}\\}
\begin{lstlisting}[frame=none]
public void construireDemande(modele.donneesxml.Demande demande)\end{lstlisting} %end signature
\begin{itemize}
\item{
{\bf  Description}

Stocke la demande
}
\item{
{\bf  Parameters}
  \begin{itemize}
   \item{
\texttt{demande} -- La demande de livraison}
  \end{itemize}
}%end item
\end{itemize}
}%end item
\item{ 
\index{construireGraphe(PlanDeVille)}
\hypertarget{vue.VueGraphiqueAideur.construireGraphe(modele.donneesxml.PlanDeVille)}{{\bf  construireGraphe}\\}
\begin{lstlisting}[frame=none]
public void construireGraphe(modele.donneesxml.PlanDeVille plan)\end{lstlisting} %end signature
\begin{itemize}
\item{
{\bf  Description}

Construit et affiche le graphe du plan de la ville sur le canvas graphique de la fenêtre
}
\item{
{\bf  Parameters}
  \begin{itemize}
   \item{
\texttt{plan} -- Le plan de la ville, chargée par le couche controleur et persistance}
  \end{itemize}
}%end item
\end{itemize}
}%end item
\item{ 
\index{construireTournee(List)}
\hypertarget{vue.VueGraphiqueAideur.construireTournee(java.util.List)}{{\bf  construireTournee}\\}
\begin{lstlisting}[frame=none]
public void construireTournee(java.util.List tournee)\end{lstlisting} %end signature
\begin{itemize}
\item{
{\bf  Description}

Construit et affiche la tournée
}
\item{
{\bf  Parameters}
  \begin{itemize}
   \item{
\texttt{tournee} -- La tournée calculée}
  \end{itemize}
}%end item
\end{itemize}
}%end item
\item{ 
\index{desactiverSurbrillance()}
\hypertarget{vue.VueGraphiqueAideur.desactiverSurbrillance()}{{\bf  desactiverSurbrillance}\\}
\begin{lstlisting}[frame=none]
public void desactiverSurbrillance()\end{lstlisting} %end signature
\begin{itemize}
\item{
{\bf  Description}

Désactive la surbrillance pour toutes les surbrillances
}
\end{itemize}
}%end item
\item{ 
\index{getCanvas()}
\hypertarget{vue.VueGraphiqueAideur.getCanvas()}{{\bf  getCanvas}\\}
\begin{lstlisting}[frame=none]
public javafx.scene.layout.StackPane getCanvas()\end{lstlisting} %end signature
\begin{itemize}
\item{{\bf  Returns} -- 
Retourne la partie graphique pour dessiné 
}%end item
\end{itemize}
}%end item
\item{ 
\index{nettoyerAffichage()}
\hypertarget{vue.VueGraphiqueAideur.nettoyerAffichage()}{{\bf  nettoyerAffichage}\\}
\begin{lstlisting}[frame=none]
public void nettoyerAffichage()\end{lstlisting} %end signature
\begin{itemize}
\item{
{\bf  Description}

Supprime tous les éléments sur la partie graphique
}
\end{itemize}
}%end item
\item{ 
\index{setControleurApplication(ControleurInterface)}
\hypertarget{vue.VueGraphiqueAideur.setControleurApplication(controleur.ControleurInterface)}{{\bf  setControleurApplication}\\}
\begin{lstlisting}[frame=none]
public void setControleurApplication(controleur.ControleurInterface controleurApplication)\end{lstlisting} %end signature
\begin{itemize}
\item{
{\bf  Description}

Met à jour le controleur de l'application pour la vue graphique
}
\end{itemize}
}%end item
\item{ 
\index{surbrillanceLivraison(Livraison)}
\hypertarget{vue.VueGraphiqueAideur.surbrillanceLivraison(modele.donneesxml.Livraison)}{{\bf  surbrillanceLivraison}\\}
\begin{lstlisting}[frame=none]
public void surbrillanceLivraison(modele.donneesxml.Livraison livraison)\end{lstlisting} %end signature
\begin{itemize}
\item{
{\bf  Description}

Affiche une livraison en surbrillance sur la partie graphique
}
\item{
{\bf  Parameters}
  \begin{itemize}
   \item{
\texttt{livraison} -- La livraison à mettre en surbrillance}
  \end{itemize}
}%end item
\end{itemize}
}%end item
\item{ 
\index{surbrillanceLivraisons(Collection)}
\hypertarget{vue.VueGraphiqueAideur.surbrillanceLivraisons(java.util.Collection)}{{\bf  surbrillanceLivraisons}\\}
\begin{lstlisting}[frame=none]
public void surbrillanceLivraisons(java.util.Collection livraisons)\end{lstlisting} %end signature
\begin{itemize}
\item{
{\bf  Description}

Mets plusieurs livraisons en surbrillance sur la partie graphique
}
\item{
{\bf  Parameters}
  \begin{itemize}
   \item{
\texttt{livraisons} -- Les livraisons à mettre en surbrillance}
  \end{itemize}
}%end item
\end{itemize}
}%end item
\end{itemize}
}
}
\section{\label{vue.VuePrincipale}Class VuePrincipale}{
\hypertarget{vue.VuePrincipale}{}\vskip .1in 
Cette classe joue le rôle de binding pour la fenetre principale de l'application. C'est ici qu'on spécifiera les écouteurs et consorts. Remarque : Les écouteurs peuvent être spécifiés directement dans le fichier xml aussi\vskip .1in 
\subsection{Declaration}{
\begin{lstlisting}[frame=none]
public class VuePrincipale
 extends java.lang.Object implements javafx.fxml.Initializable, controleur.observateur.ActivationOuvrirDemandeObservateur, controleur.observateur.ActivationOuvrirPlanObservateur, controleur.observateur.ModeleObservateur, controleur.observateur.AnnulerCommandeObservateur, controleur.observateur.RetablirCommandeObservateur, controleur.observateur.PlanChargeObservateur\end{lstlisting}
\subsection{Constructor summary}{
\begin{verse}
\hyperlink{vue.VuePrincipale()}{{\bf VuePrincipale()}} \\
\end{verse}
}
\subsection{Method summary}{
\begin{verse}
\hyperlink{vue.VuePrincipale.getAideurVueGraphique()}{{\bf getAideurVueGraphique()}} \\
\hyperlink{vue.VuePrincipale.initialiserObserveurs()}{{\bf initialiserObserveurs()}} Initalise les differents obserserveurs de la vue principale\\
\hyperlink{vue.VuePrincipale.initialize(java.net.URL, java.util.ResourceBundle)}{{\bf initialize(URL, ResourceBundle)}} \\
\hyperlink{vue.VuePrincipale.notifierObservateurAnnulerCommande(boolean)}{{\bf notifierObservateurAnnulerCommande(boolean)}} \\
\hyperlink{vue.VuePrincipale.notifierObservateurOuvrirDemande(boolean)}{{\bf notifierObservateurOuvrirDemande(boolean)}} \\
\hyperlink{vue.VuePrincipale.notifierObservateurRetablirCommande(boolean)}{{\bf notifierObservateurRetablirCommande(boolean)}} \\
\hyperlink{vue.VuePrincipale.notifierObservateursModele()}{{\bf notifierObservateursModele()}} \\
\hyperlink{vue.VuePrincipale.notifierObservateursOuvrirPlan(boolean)}{{\bf notifierObservateursOuvrirPlan(boolean)}} \\
\hyperlink{vue.VuePrincipale.notifierObservateursPlanCharge()}{{\bf notifierObservateursPlanCharge()}} \\
\hyperlink{vue.VuePrincipale.setControleurApplication(controleur.ControleurInterface)}{{\bf setControleurApplication(ControleurInterface)}} Met à jour le controleur de l'application\\
\hyperlink{vue.VuePrincipale.setVueGraphiqueControleurApplication(controleur.ControleurInterface)}{{\bf setVueGraphiqueControleurApplication(ControleurInterface)}} + Met à jour le controleur de l'application pour la vue graphique\\
\end{verse}
}
\subsection{Constructors}{
\vskip -2em
\begin{itemize}
\item{ 
\index{VuePrincipale()}
\hypertarget{vue.VuePrincipale()}{{\bf  VuePrincipale}\\}
\begin{lstlisting}[frame=none]
public VuePrincipale()\end{lstlisting} %end signature
}%end item
\end{itemize}
}
\subsection{Methods}{
\vskip -2em
\begin{itemize}
\item{ 
\index{getAideurVueGraphique()}
\hypertarget{vue.VuePrincipale.getAideurVueGraphique()}{{\bf  getAideurVueGraphique}\\}
\begin{lstlisting}[frame=none]
protected VueGraphiqueAideur getAideurVueGraphique()\end{lstlisting} %end signature
\begin{itemize}
\item{{\bf  Returns} -- 
La vue graphique 
}%end item
\end{itemize}
}%end item
\item{ 
\index{initialiserObserveurs()}
\hypertarget{vue.VuePrincipale.initialiserObserveurs()}{{\bf  initialiserObserveurs}\\}
\begin{lstlisting}[frame=none]
protected void initialiserObserveurs()\end{lstlisting} %end signature
\begin{itemize}
\item{
{\bf  Description}

Initalise les differents obserserveurs de la vue principale
}
\end{itemize}
}%end item
\item{ 
\index{initialize(URL, ResourceBundle)}
\hypertarget{vue.VuePrincipale.initialize(java.net.URL, java.util.ResourceBundle)}{{\bf  initialize}\\}
\begin{lstlisting}[frame=none]
void initialize(java.net.URL arg0,java.util.ResourceBundle arg1)\end{lstlisting} %end signature
}%end item
\item{ 
\index{notifierObservateurAnnulerCommande(boolean)}
\hypertarget{vue.VuePrincipale.notifierObservateurAnnulerCommande(boolean)}{{\bf  notifierObservateurAnnulerCommande}\\}
\begin{lstlisting}[frame=none]
void notifierObservateurAnnulerCommande(boolean active)\end{lstlisting} %end signature
\begin{itemize}
\item{
{\bf  Description copied from \hyperlink{controleur.observateur.AnnulerCommandeObservateur}{controleur.observateur.AnnulerCommandeObservateur}{\small \refdefined{controleur.observateur.AnnulerCommandeObservateur}} }

Notifie l'observateur s'il faut activer l'action dans le menu qui permet d'annuler une commande
}
\item{
{\bf  Parameters}
  \begin{itemize}
   \item{
\texttt{active} -- Dit si l'action Annuler doit etre activée ou non}
  \end{itemize}
}%end item
\end{itemize}
}%end item
\item{ 
\index{notifierObservateurOuvrirDemande(boolean)}
\hypertarget{vue.VuePrincipale.notifierObservateurOuvrirDemande(boolean)}{{\bf  notifierObservateurOuvrirDemande}\\}
\begin{lstlisting}[frame=none]
void notifierObservateurOuvrirDemande(boolean activer)\end{lstlisting} %end signature
\begin{itemize}
\item{
{\bf  Description copied from \hyperlink{controleur.observateur.ActivationOuvrirDemandeObservateur}{controleur.observateur.ActivationOuvrirDemandeObservateur}{\small \refdefined{controleur.observateur.ActivationOuvrirDemandeObservateur}} }

Notifie les observateurs qu'il faut ou pas activer l'élément du menu qui permet d'ouvrir une demande
}
\item{
{\bf  Parameters}
  \begin{itemize}
   \item{
\texttt{activer} -- Vrai s'il faut s'activer ou se désactiver lors de cette modification}
  \end{itemize}
}%end item
\end{itemize}
}%end item
\item{ 
\index{notifierObservateurRetablirCommande(boolean)}
\hypertarget{vue.VuePrincipale.notifierObservateurRetablirCommande(boolean)}{{\bf  notifierObservateurRetablirCommande}\\}
\begin{lstlisting}[frame=none]
void notifierObservateurRetablirCommande(boolean active)\end{lstlisting} %end signature
\begin{itemize}
\item{
{\bf  Description copied from \hyperlink{controleur.observateur.RetablirCommandeObservateur}{controleur.observateur.RetablirCommandeObservateur}{\small \refdefined{controleur.observateur.RetablirCommandeObservateur}} }

Notifie l'observateur s'il faut activer l'action dans le menu qui permet de retablir une commande une commande
}
\item{
{\bf  Parameters}
  \begin{itemize}
   \item{
\texttt{active} -- Vrai s'il faut activer suite au rétablissement de la commande}
  \end{itemize}
}%end item
\end{itemize}
}%end item
\item{ 
\index{notifierObservateursModele()}
\hypertarget{vue.VuePrincipale.notifierObservateursModele()}{{\bf  notifierObservateursModele}\\}
\begin{lstlisting}[frame=none]
void notifierObservateursModele()\end{lstlisting} %end signature
\begin{itemize}
\item{
{\bf  Description copied from \hyperlink{controleur.observateur.ModeleObservateur}{controleur.observateur.ModeleObservateur}{\small \refdefined{controleur.observateur.ModeleObservateur}} }

Notifie les observeurs que le modèle a changé
}
\end{itemize}
}%end item
\item{ 
\index{notifierObservateursOuvrirPlan(boolean)}
\hypertarget{vue.VuePrincipale.notifierObservateursOuvrirPlan(boolean)}{{\bf  notifierObservateursOuvrirPlan}\\}
\begin{lstlisting}[frame=none]
void notifierObservateursOuvrirPlan(boolean activer)\end{lstlisting} %end signature
\begin{itemize}
\item{
{\bf  Description copied from \hyperlink{controleur.observateur.ActivationOuvrirPlanObservateur}{controleur.observateur.ActivationOuvrirPlanObservateur}{\small \refdefined{controleur.observateur.ActivationOuvrirPlanObservateur}} }

Notifie les observateurs qu'il faut ou pas activer l'élément du menu qui permet d'ouvrir un plan
}
\item{
{\bf  Parameters}
  \begin{itemize}
   \item{
\texttt{activer} -- Le plan a été charger ou désactivé}
  \end{itemize}
}%end item
\end{itemize}
}%end item
\item{ 
\index{notifierObservateursPlanCharge()}
\hypertarget{vue.VuePrincipale.notifierObservateursPlanCharge()}{{\bf  notifierObservateursPlanCharge}\\}
\begin{lstlisting}[frame=none]
void notifierObservateursPlanCharge()\end{lstlisting} %end signature
\begin{itemize}
\item{
{\bf  Description copied from \hyperlink{controleur.observateur.PlanChargeObservateur}{controleur.observateur.PlanChargeObservateur}{\small \refdefined{controleur.observateur.PlanChargeObservateur}} }

Notifie la vue qu'un nouveau plan a été chargé.
}
\end{itemize}
}%end item
\item{ 
\index{setControleurApplication(ControleurInterface)}
\hypertarget{vue.VuePrincipale.setControleurApplication(controleur.ControleurInterface)}{{\bf  setControleurApplication}\\}
\begin{lstlisting}[frame=none]
public void setControleurApplication(controleur.ControleurInterface controleurApplication)\end{lstlisting} %end signature
\begin{itemize}
\item{
{\bf  Description}

Met à jour le controleur de l'application
}
\item{
{\bf  Parameters}
  \begin{itemize}
   \item{
\texttt{controleurApplication} -- Le nouveau controleur d'interface}
  \end{itemize}
}%end item
\end{itemize}
}%end item
\item{ 
\index{setVueGraphiqueControleurApplication(ControleurInterface)}
\hypertarget{vue.VuePrincipale.setVueGraphiqueControleurApplication(controleur.ControleurInterface)}{{\bf  setVueGraphiqueControleurApplication}\\}
\begin{lstlisting}[frame=none]
public void setVueGraphiqueControleurApplication(controleur.ControleurInterface controleurApplication)\end{lstlisting} %end signature
\begin{itemize}
\item{
{\bf  Description}

+ Met à jour le controleur de l'application pour la vue graphique
}
\end{itemize}
}%end item
\end{itemize}
}
}
\section{\label{vue.VueTextuelle}Class VueTextuelle}{
\hypertarget{vue.VueTextuelle}{}\vskip .1in 
Cette classe gère les livraisons et leurs horaires. Elle s'occupe de la vue textuelle qui se trouve à gauche dans la fenêtre principale.\vskip .1in 
\subsection{Declaration}{
\begin{lstlisting}[frame=none]
public class VueTextuelle
 extends java.lang.Object implements javafx.fxml.Initializable, controleur.observateur.ModeleObservateur, controleur.observateur.PlanChargeObservateur\end{lstlisting}
\subsection{Constructor summary}{
\begin{verse}
\hyperlink{vue.VueTextuelle()}{{\bf VueTextuelle()}} \\
\end{verse}
}
\subsection{Method summary}{
\begin{verse}
\hyperlink{vue.VueTextuelle.construireVueTableLivraion(modele.donneesxml.Demande)}{{\bf construireVueTableLivraion(Demande)}} Contruit la table des livraisons\\
\hyperlink{vue.VueTextuelle.initialiserObserveurs()}{{\bf initialiserObserveurs()}} Ajoute la vue textuelle comme observeurs au près du controleur\\
\hyperlink{vue.VueTextuelle.initialize(java.net.URL, java.util.ResourceBundle)}{{\bf initialize(URL, ResourceBundle)}} Méthode appelée automatiquement au chargement du fichier XML\\
\hyperlink{vue.VueTextuelle.notifierObservateursModele()}{{\bf notifierObservateursModele()}} Notification déclenchée lors d'un changement dans le model.\\
\hyperlink{vue.VueTextuelle.notifierObservateursPlanCharge()}{{\bf notifierObservateursPlanCharge()}} Efface le contenu de la table à chaque chargement d'un plan.\\
\hyperlink{vue.VueTextuelle.setAideurVueGraphique(vue.VueGraphiqueAideur)}{{\bf setAideurVueGraphique(VueGraphiqueAideur)}} Met à jour la référence vers la vue graphique\\
\hyperlink{vue.VueTextuelle.setControleurApplication(controleur.ControleurInterface)}{{\bf setControleurApplication(ControleurInterface)}} Met à jour le controleur de l'application pour la vue textuelle\\
\end{verse}
}
\subsection{Constructors}{
\vskip -2em
\begin{itemize}
\item{ 
\index{VueTextuelle()}
\hypertarget{vue.VueTextuelle()}{{\bf  VueTextuelle}\\}
\begin{lstlisting}[frame=none]
public VueTextuelle()\end{lstlisting} %end signature
}%end item
\end{itemize}
}
\subsection{Methods}{
\vskip -2em
\begin{itemize}
\item{ 
\index{construireVueTableLivraion(Demande)}
\hypertarget{vue.VueTextuelle.construireVueTableLivraion(modele.donneesxml.Demande)}{{\bf  construireVueTableLivraion}\\}
\begin{lstlisting}[frame=none]
protected void construireVueTableLivraion(modele.donneesxml.Demande demande)\end{lstlisting} %end signature
\begin{itemize}
\item{
{\bf  Description}

Contruit la table des livraisons
}
\item{
{\bf  Parameters}
  \begin{itemize}
   \item{
\texttt{demande} -- La demande de livraison chargée à partir d'un fichier XML. La demande ne doit pas être null}
  \end{itemize}
}%end item
\end{itemize}
}%end item
\item{ 
\index{initialiserObserveurs()}
\hypertarget{vue.VueTextuelle.initialiserObserveurs()}{{\bf  initialiserObserveurs}\\}
\begin{lstlisting}[frame=none]
protected void initialiserObserveurs()\end{lstlisting} %end signature
\begin{itemize}
\item{
{\bf  Description}

Ajoute la vue textuelle comme observeurs au près du controleur
}
\end{itemize}
}%end item
\item{ 
\index{initialize(URL, ResourceBundle)}
\hypertarget{vue.VueTextuelle.initialize(java.net.URL, java.util.ResourceBundle)}{{\bf  initialize}\\}
\begin{lstlisting}[frame=none]
public void initialize(java.net.URL location,java.util.ResourceBundle resources)\end{lstlisting} %end signature
\begin{itemize}
\item{
{\bf  Description}

Méthode appelée automatiquement au chargement du fichier XML
}
\end{itemize}
}%end item
\item{ 
\index{notifierObservateursModele()}
\hypertarget{vue.VueTextuelle.notifierObservateursModele()}{{\bf  notifierObservateursModele}\\}
\begin{lstlisting}[frame=none]
public void notifierObservateursModele()\end{lstlisting} %end signature
\begin{itemize}
\item{
{\bf  Description}

Notification déclenchée lors d'un changement dans le model. Cette peut etre notification est déclenchée que si on a déja chargé un plan et une demande de livraison dans l'application
}
\end{itemize}
}%end item
\item{ 
\index{notifierObservateursPlanCharge()}
\hypertarget{vue.VueTextuelle.notifierObservateursPlanCharge()}{{\bf  notifierObservateursPlanCharge}\\}
\begin{lstlisting}[frame=none]
public void notifierObservateursPlanCharge()\end{lstlisting} %end signature
\begin{itemize}
\item{
{\bf  Description}

Efface le contenu de la table à chaque chargement d'un plan.
}
\end{itemize}
}%end item
\item{ 
\index{setAideurVueGraphique(VueGraphiqueAideur)}
\hypertarget{vue.VueTextuelle.setAideurVueGraphique(vue.VueGraphiqueAideur)}{{\bf  setAideurVueGraphique}\\}
\begin{lstlisting}[frame=none]
protected void setAideurVueGraphique(VueGraphiqueAideur vueGraphique)\end{lstlisting} %end signature
\begin{itemize}
\item{
{\bf  Description}

Met à jour la référence vers la vue graphique
}
\end{itemize}
}%end item
\item{ 
\index{setControleurApplication(ControleurInterface)}
\hypertarget{vue.VueTextuelle.setControleurApplication(controleur.ControleurInterface)}{{\bf  setControleurApplication}\\}
\begin{lstlisting}[frame=none]
protected void setControleurApplication(controleur.ControleurInterface controleurApplication)\end{lstlisting} %end signature
\begin{itemize}
\item{
{\bf  Description}

Met à jour le controleur de l'application pour la vue textuelle
}
\item{
{\bf  Parameters}
  \begin{itemize}
   \item{
\texttt{controleurApplication} -- controleur initialisé (non null)}
  \end{itemize}
}%end item
\end{itemize}
}%end item
\end{itemize}
}
}
}
\chapter{Package }{
\label{<none>}\hypertarget{<none>}{}
\hskip -.05in
\hbox to \hsize{\textit{ Package Contents\hfil Page}}
\vskip .13in
\hbox{{\bf  Classes}}
\entityintro{DevOO}{DevOO}{Classe d'entrée de l'application}
\vskip .1in
\vskip .1in
\section{\label{DevOO}Class DevOO}{
\hypertarget{DevOO}{}\vskip .1in 
Classe d'entrée de l'application\vskip .1in 
\subsection{Declaration}{
\begin{lstlisting}[frame=none]
public class DevOO
 extends java.lang.Object\end{lstlisting}
\subsection{Constructor summary}{
\begin{verse}
\hyperlink{DevOO()}{{\bf DevOO()}} \\
\end{verse}
}
\subsection{Method summary}{
\begin{verse}
\hyperlink{DevOO.main(java.lang.String[])}{{\bf main(String\lbrack \rbrack )}} Point d'entrée de l'application\\
\end{verse}
}
\subsection{Constructors}{
\vskip -2em
\begin{itemize}
\item{ 
\index{DevOO()}
\hypertarget{DevOO()}{{\bf  DevOO}\\}
\begin{lstlisting}[frame=none]
public DevOO()\end{lstlisting} %end signature
}%end item
\end{itemize}
}
\subsection{Methods}{
\vskip -2em
\begin{itemize}
\item{ 
\index{main(String\lbrack \rbrack )}
\hypertarget{DevOO.main(java.lang.String[])}{{\bf  main}\\}
\begin{lstlisting}[frame=none]
public static void main(java.lang.String[] args)\end{lstlisting} %end signature
\begin{itemize}
\item{
{\bf  Description}

Point d'entrée de l'application
}
\item{
{\bf  Parameters}
  \begin{itemize}
   \item{
\texttt{args} -- Les arguments en ligne de commande}
  \end{itemize}
}%end item
\end{itemize}
}%end item
\end{itemize}
}
}
}
\end{document}
